\usepackage[utf8]{inputenc}

% \usepackage[
% HomeHTMLFilename=index, % Filename of the homepage.
% HTMLFilename={node-}, % Filename prefix of other pages.
% % IndexLanguage=english, % Language for xindy index, glossary.
% % latexmk, % Use latexmk to compile.
% % mathjax, % Use MathJax to display math.
% ]{lwarp}

\usepackage{amsmath, amsthm, amssymb, amsfonts}
\usepackage{thmtools}
\usepackage{graphicx}
\usepackage{setspace}
\usepackage{geometry}
\usepackage{mathrsfs}

\usepackage{float}
\usepackage{amsthm}
\usepackage{hyperref}




\usepackage{cleveref}
\crefname{equation}{式}{式}
\crefname{figure}{图}{图}
\crefname{table}{表}{表}
\crefname{page}{页}{页}
\crefname{chapter}{章}{章}
\crefname{section}{节}{节}
\crefname{appendix}{附录}{附录}
\crefname{theorem}{定理}{定理}
\crefname{lemma}{引理}{引理}
\crefname{corollary}{推论}{推论}
\crefname{proposition}{命题}{命题}
\crefname{definition}{定义}{定义}
\crefname{example}{例}{例}
\crefname{algorithm}{算法}{算法}
\crefname{listing}{列表}{列表}
\crefname{line}{行}{行}

\crefformat{chapter}{#2第#1章#3}
\crefformat{section}{#2第#1节#3}
\crefformat{subsection}{#2第#1节#3}
\crefformat{subsubsection}{#2第#1节#3}

\crefrangeformat{chapter}{#3第#1章#4至#5第#2章#6}
\crefrangeformat{section}{#3第#1节#4至#5第#2节#6}
\crefrangeformat{subsection}{#3第#1节#4至#5第#2节#6}
\crefrangeformat{subsubsection}{#3第#1节#4至#5第#2节#6}

\crefmultiformat{chapter}{#2第#1章#3}{和#2第#1章#3}{, #2第#1章#3}{和#2第#1章#3}
\crefmultiformat{section}{#2第#1节#3}{和#2第#1节#3}{, #2第#1节#3}{和#2第#1节#3}
\crefmultiformat{subsection}{#2第#1节#3}{和#2第#1节#3}{, #2第#1节#3}{和#2第#1节#3}
\crefmultiformat{subsubsection}{#2第#1节#3}{和#2第#1节#3}{, #2第#1节#3}{和#2第#1节#3}

\crefrangemultiformat{chapter}{#3第#1章#4至#5第#2章#6}{和#3第#1章#4至#5第#2章#6}{, #3第#1章#4至#5第#2章#6}{和#3第#1章#4至#5第#2章#6}
\crefrangemultiformat{section}{#3第#1节#4至#5第#2节#6}{和#3第#1节#4至#5第#2节#6}{, #3第#1节#4至#5第#2节#6}{和#3第#1节#4至#5第#2节#6}
\crefrangemultiformat{subsection}{#3第#1节#4至#5第#2节#6}{和#3第#1节#4至#5第#2节#6}{, #3第#1节#4至#5第#2节#6}{和#3第#1节#4至#5第#2节#6}
\crefrangemultiformat{subsubsection}{#3第#1节#4至#5第#2节#6}{和#3第#1节#4至#5第#2节#6}{, #3第#1节#4至#5第#2节#6}{和#3第#1节#4至#5第#2节#6}

\newcommand{\crefpairconjunction}{~和~}
\newcommand{\crefmiddleconjunction}{, }
\newcommand{\creflastconjunction}{~和~}
\newcommand{\crefpairgroupconjunction}{~和~}
\newcommand{\crefmiddlegroupconjunction}{, }
\newcommand{\creflastgroupconjunction}{~和~}
\newcommand{\crefrangeconjunction}{~至~}

% LISTINGS
\usepackage{listings}
\usepackage{xcolor}
\definecolor{codegreen}{rgb}{0,0.6,0}
\definecolor{codegray}{rgb}{0.5,0.5,0.5}
\definecolor{codepurple}{rgb}{0.58,0,0.82}
\definecolor{backcolour}{rgb}{0.95,0.95,0.92}

\lstdefinestyle{mystyle}{
    backgroundcolor=\color{backcolour},   
    commentstyle=\color{codegreen},
    keywordstyle=\color{magenta},
    numberstyle=\tiny\color{codegray},
    stringstyle=\color{codepurple},
    basicstyle=\ttfamily\footnotesize,
    breakatwhitespace=false,         
    breaklines=true,                 
    captionpos=b,                    
    keepspaces=true,                 
    numbers=left,                    
    numbersep=5pt,                  
    showspaces=false,                
    showstringspaces=false,
    showtabs=false,                  
    tabsize=2
}

\lstset{style=mystyle}

\usepackage{algorithm}
\usepackage[noend]{algpseudocode}

% Make SS at the beginning of a section

\makeatletter
%% See pp. 26f. of 'The LaTeX Companion,' 2nd. ed.
\def\@seccntformat#1{\@ifundefined{#1@cntformat}%
    {\csname the#1\endcsname\quad}%      default
    {\csname #1@cntformat\endcsname}}%   individual control
\newcommand{\section@cntformat}{\S\thesection\quad}
\newcommand{\subsection@cntformat}{\S\thesubsection\quad}
\makeatother % changes @ back to a special character

\usepackage{titlesec}

\CTEXsetup[format={\raggedright\large\bfseries}]{section}
\titleformat{\subsection}[runin]{\normalfont\bfseries}{\thesubsection.}{0.5em}{}[.]
\titleformat{\subsubsection}[runin]{\normalfont\bfseries}{\alph{subsubsection})}{0.5em}{}


\usepackage{annotate-equations}



\theoremstyle{definition}
\newtheorem{example}{例子}[section]
\newtheorem{definition}{定义}[section]
\newtheorem{theorem}{定理}[section]
\newtheorem{proposition}[theorem]{命题}
\newtheorem{prop}[theorem]{性质}
\newtheorem{corollary}[theorem]{推论}
\newtheorem{exercise}{练习}


\newenvironment{remark}{%
  \par\medskip
  \noindent
  \textbf{注:}
}{%
  \par\medskip
}

\newenvironment{chk}{%
  \par\medskip
  \noindent
  \textbf{检查理解:}
}{%
  \par\medskip
}

\newenvironment{solution}{%
  \par\medskip
  \noindent
  \textbf{解答:}
}{%
  \par\medskip
}

\newenvironment{solution*}{%
  \par\medskip
  \noindent 
  \color{gray}\small\textbf{提示或解答:}
}{%
  \par\medskip
}

\newenvironment{definition*}{%
  \par\medskip
  \noindent
  \textbf{定义:}
}{%
  \par\medskip
}

\newenvironment{lemma}{%
  \par\medskip
  \noindent
  \textbf{引理:}
}{%
  \par\medskip
}

\newenvironment{proposition*}{%
  \par\medskip
  \noindent
  \textbf{性质: }
}{%
  \par\medskip
}




\usepackage{enumitem}

\setlist{nosep}

\setstretch{1.2}
\geometry{
    textheight=9in,
    textwidth=5.5in,
    top=1in,
    headheight=12pt,
    headsep=25pt,
    footskip=30pt, 
    margin=1in,
    rmargin=2.5in
}
\setlength{\marginparwidth}{2in}
\usepackage{environ}



\newcommand{\lecture}[6]{
   \begin{center}
   \framebox{
      \vbox{\vspace{2mm}
    \hbox to 6.28in { {\bf #1
		\hfill #6} }
       \vspace{4mm}
       \hbox to 6.28in { {\Large \hfill 第 #2 节: #3  \hfill} }
       \vspace{2mm}
       \hbox to 6.28in { {\it Lecturer: #4 \hfill Scribes: #5} }
      \vspace{2mm}}
   }
   \end{center}
   
   \vspace*{4mm}
}

\usepackage{caption}
\captionsetup{
  justification=raggedright,
  font=small}
% \setlength{\marginparwidth}{100pt}

\usepackage{marginnote}
\renewcommand*{\marginfont}{\color{gray}\ttfamily\small}
\usepackage{setspace}
\newcounter{paranum}[section]
\newcommand{\Par}[1]{\vspace{10pt}\noindent\textbf{\refstepcounter{paranum}\theparanum. }\textbf{#1}~~}
\newcommand{\lec}[1]{\reversemarginpar\marginnote{{\textbf{#1}}}}
\newcommand{\mn}[1]{\marginnote{{#1}}}
% \renewcommand{\algorithmcfname}{算法}
\usepackage[abspath]{currfile}

\newcommand{\incfig}[1]{\begin{center}\includegraphics[width=.4\textwidth]{figs/\jobname/#1}\end{center}}
\newcommand{\incfigw}[1]{\begin{center}\includegraphics[width=.8\textwidth]{figs/\jobname/#1}\end{center}}
\newcommand{\incfigside}[3]{\marginpar{\includegraphics[width=\marginparwidth]{figs/\jobname/#1}
  \captionof{figure}{#2}\label{fig:#3}}}

