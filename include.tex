
\usepackage{amsmath, amsthm, amssymb, amsfonts}
\usepackage{thmtools}
\usepackage{graphicx}
\usepackage{setspace}
\usepackage{geometry}

\usepackage{float}
\usepackage{amsthm}
\usepackage{hyperref}
\usepackage{cleveref}
% \usepackage{mathabx}
\usepackage[utf8]{inputenc}
\usepackage[english]{babel}
\usepackage{framed}
\usepackage[dvipsnames]{xcolor}
\usepackage[skins,breakable]{tcolorbox}
\usepackage{awesomebox}
\usepackage{mathrsfs}  
\usepackage{xcolor}
\usepackage{wrapfig}
\usepackage{algorithm2e}
\RestyleAlgo{ruled}

\usepackage{pstricks-add}
\usepackage{epsfig}
\usepackage{pst-grad} % For gradients
\usepackage{pst-plot} % For axes
\usepackage[space]{grffile} % For spaces in paths
\usepackage{etoolbox} % For spaces in paths
\makeatletter % For spaces in paths
\patchcmd\Gread@eps{\@inputcheck#1 }{\@inputcheck"#1"\relax}{}{}
\makeatother

% Make SS at the beginning of a section

\makeatletter
%% See pp. 26f. of 'The LaTeX Companion,' 2nd. ed.
\def\@seccntformat#1{\@ifundefined{#1@cntformat}%
    {\csname the#1\endcsname\quad}%      default
    {\csname #1@cntformat\endcsname}}%   individual control
\newcommand{\section@cntformat}{\S\thesection\quad}
\newcommand{\subsection@cntformat}{\S\thesubsection\quad}
\makeatother % changes @ back to a special character

\usepackage{titlesec}

\CTEXsetup[format={\raggedright\large\bfseries}]{section}
\titleformat{\subsection}[runin]{\normalfont\bfseries}{\thesubsection.}{0.5em}{}[.]
\titleformat{\subsubsection}[runin]{\normalfont\bfseries}{\alph{subsubsection})}{0.5em}{}





\theoremstyle{definition}
\newtheorem{example}{例子}[section]
\newtheorem{definition}{定义}[section]
\newtheorem{theorem}{定理}[section]
\newtheorem{proposition}[theorem]{命题}
\newtheorem{prop}[theorem]{性质}
\newtheorem{corollary}[theorem]{推论}

\newenvironment{remark}{%
  \par\medskip
  \noindent
  \textbf{注:}
}{%
  \par\medskip
}

\newenvironment{solution}{%
  \par\medskip
  \noindent
  \textbf{解答:}
}{%
  \par\medskip
}

\newenvironment{solution*}{%
  \par\medskip
  \noindent 
  \color{gray}\small\textbf{提示或解答:}
}{%
  \par\medskip
}

\newenvironment{definition*}{%
  \par\medskip
  \noindent
  \textbf{定义:}
}{%
  \par\medskip
}

\newenvironment{lemma}{%
  \par\medskip
  \noindent
  \textbf{引理:}
}{%
  \par\medskip
}

\newenvironment{proposition*}{%
  \par\medskip
  \noindent
  \textbf{性质: }
}{%
  \par\medskip
}


\newtcolorbox{asidebox}{
  colback=gray!10,
  colframe=gray!60,
  fonttitle=\bfseries,
  title={Aside},
  breakable=true
}

\newtcolorbox{webaside}{
  colback=cyan!10,
  colframe=cyan!60,
  fonttitle=\bfseries,
  title={Web Demonstrate Aside},
  breakable=true
}

\usepackage{enumitem}

\setlist{nosep}

\setstretch{1.2}
\geometry{
    textheight=9in,
    textwidth=5.5in,
    top=1in,
    headheight=12pt,
    headsep=25pt,
    footskip=30pt
}

\usepackage{environ}
\usepackage[tikz]{bclogo}
\usepackage{tikz}
\usetikzlibrary{calc}
\NewEnviron{takeaway}
  {\par\medskip\noindent
  \begin{tikzpicture}
    \node[inner sep=0pt] (box) {\parbox[t]{.99\textwidth}{%
      \begin{minipage}{.3\textwidth}
      \centering\tikz[scale=5]\node[scale=3,rotate=30]{\bclampe};
      \end{minipage}%
      \begin{minipage}{.65\textwidth}
      \textbf{Takeaway Message}\par\smallskip
      \BODY
      \end{minipage}\hfill}%
    };
    \draw[red!75!black,line width=3pt] 
      ( $ (box.north east) + (-5pt,3pt) $ ) -- ( $ (box.north east) + (0,3pt) $ ) -- ( $ (box.south east) + (0,-3pt) $ ) -- + (-5pt,0);
    \draw[red!75!black,line width=3pt] 
      ( $ (box.north west) + (5pt,3pt) $ ) -- ( $ (box.north west) + (0,3pt) $ ) -- ( $ (box.south west) + (0,-3pt) $ ) -- + (5pt,0);
  \end{tikzpicture}\par\medskip%
}


\newcommand{\lecture}[6]{
   \begin{center}
   \framebox{
      \vbox{\vspace{2mm}
    \hbox to 6.28in { {\bf #1
		\hfill #6} }
       \vspace{4mm}
       \hbox to 6.28in { {\Large \hfill 第 #2 节: #3  \hfill} }
       \vspace{2mm}
       \hbox to 6.28in { {\it Lecturer: #4 \hfill Scribes: #5} }
      \vspace{2mm}}
   }
   \end{center}
   
   \vspace*{4mm}
}

\usepackage{marginnote}
\renewcommand*{\marginfont}{\color{gray}\ttfamily\small}
\usepackage{setspace}
\newcounter{paranum}[section]
\newcommand{\Par}[1]{\vspace{10pt}\noindent\textbf{\refstepcounter{paranum}\theparanum. }\textbf{#1}~~}
\newcommand{\lec}[1]{\reversemarginpar\marginnote{{\textbf{#1}}}}
\newcommand{\mn}[1]{\marginnote{{#1}}}
\renewcommand{\algorithmcfname}{算法}
\usepackage[abspath]{currfile}

\newcommand{\incfig}[1]{\begin{center}\includegraphics[width=.4\textwidth]{figs/\currfilename/#1}\end{center}}
\newcommand{\incfigw}[1]{\begin{center}\includegraphics[width=.8\textwidth]{figs/\currfilename/#1}\end{center}}


