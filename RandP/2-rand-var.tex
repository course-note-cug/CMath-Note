%!TEX TS-program = xelatex
\documentclass{ctexart}
\usepackage{cancel}
\usepackage[utf8]{inputenc}

% \usepackage[
% HomeHTMLFilename=index, % Filename of the homepage.
% HTMLFilename={node-}, % Filename prefix of other pages.
% % IndexLanguage=english, % Language for xindy index, glossary.
% % latexmk, % Use latexmk to compile.
% % mathjax, % Use MathJax to display math.
% ]{lwarp}

\usepackage{amsmath, amsthm, amssymb, amsfonts}
\usepackage{thmtools}
\usepackage{graphicx}
\usepackage{setspace}
\usepackage{geometry}

\usepackage{float}
\usepackage{amsthm}
\usepackage{hyperref}


\usepackage{cleveref}
\crefname{equation}{式}{式}
\crefname{figure}{图}{图}
\crefname{table}{表}{表}
\crefname{page}{页}{页}
\crefname{chapter}{章}{章}
\crefname{section}{节}{节}
\crefname{appendix}{附录}{附录}
\crefname{theorem}{定理}{定理}
\crefname{lemma}{引理}{引理}
\crefname{corollary}{推论}{推论}
\crefname{proposition}{命题}{命题}
\crefname{definition}{定义}{定义}
\crefname{example}{例}{例}
\crefname{algorithm}{算法}{算法}
\crefname{listing}{列表}{列表}
\crefname{line}{行}{行}

\crefformat{chapter}{#2第#1章#3}
\crefformat{section}{#2第#1节#3}
\crefformat{subsection}{#2第#1节#3}
\crefformat{subsubsection}{#2第#1节#3}

\crefrangeformat{chapter}{#3第#1章#4至#5第#2章#6}
\crefrangeformat{section}{#3第#1节#4至#5第#2节#6}
\crefrangeformat{subsection}{#3第#1节#4至#5第#2节#6}
\crefrangeformat{subsubsection}{#3第#1节#4至#5第#2节#6}

\crefmultiformat{chapter}{#2第#1章#3}{和#2第#1章#3}{, #2第#1章#3}{和#2第#1章#3}
\crefmultiformat{section}{#2第#1节#3}{和#2第#1节#3}{, #2第#1节#3}{和#2第#1节#3}
\crefmultiformat{subsection}{#2第#1节#3}{和#2第#1节#3}{, #2第#1节#3}{和#2第#1节#3}
\crefmultiformat{subsubsection}{#2第#1节#3}{和#2第#1节#3}{, #2第#1节#3}{和#2第#1节#3}

\crefrangemultiformat{chapter}{#3第#1章#4至#5第#2章#6}{和#3第#1章#4至#5第#2章#6}{, #3第#1章#4至#5第#2章#6}{和#3第#1章#4至#5第#2章#6}
\crefrangemultiformat{section}{#3第#1节#4至#5第#2节#6}{和#3第#1节#4至#5第#2节#6}{, #3第#1节#4至#5第#2节#6}{和#3第#1节#4至#5第#2节#6}
\crefrangemultiformat{subsection}{#3第#1节#4至#5第#2节#6}{和#3第#1节#4至#5第#2节#6}{, #3第#1节#4至#5第#2节#6}{和#3第#1节#4至#5第#2节#6}
\crefrangemultiformat{subsubsection}{#3第#1节#4至#5第#2节#6}{和#3第#1节#4至#5第#2节#6}{, #3第#1节#4至#5第#2节#6}{和#3第#1节#4至#5第#2节#6}

\newcommand{\crefpairconjunction}{~和~}
\newcommand{\crefmiddleconjunction}{, }
\newcommand{\creflastconjunction}{~和~}
\newcommand{\crefpairgroupconjunction}{~和~}
\newcommand{\crefmiddlegroupconjunction}{, }
\newcommand{\creflastgroupconjunction}{~和~}
\newcommand{\crefrangeconjunction}{~至~}


% \usepackage{mathabx}

\usepackage[english]{babel}
\usepackage{framed}
\usepackage[dvipsnames]{xcolor}
\usepackage[skins,breakable]{tcolorbox}
\usepackage{awesomebox}
\usepackage{mathrsfs}  
\usepackage{xcolor}
\usepackage{wrapfig}
% \usepackage{algorithm2e}
\usepackage{algorithm}
\usepackage[noend]{algpseudocode}
% \RestyleAlgo{ruled}

\usepackage{pstricks-add}
\usepackage{epsfig}
\usepackage{pst-grad} % For gradients
\usepackage{pst-plot} % For axes
\usepackage[space]{grffile} % For spaces in paths
\usepackage{etoolbox} % For spaces in paths
\makeatletter % For spaces in paths
\patchcmd\Gread@eps{\@inputcheck#1 }{\@inputcheck"#1"\relax}{}{}
\makeatother

% Make SS at the beginning of a section

\makeatletter
%% See pp. 26f. of 'The LaTeX Companion,' 2nd. ed.
\def\@seccntformat#1{\@ifundefined{#1@cntformat}%
    {\csname the#1\endcsname\quad}%      default
    {\csname #1@cntformat\endcsname}}%   individual control
\newcommand{\section@cntformat}{\S\thesection\quad}
\newcommand{\subsection@cntformat}{\S\thesubsection\quad}
\makeatother % changes @ back to a special character

\usepackage{titlesec}

\CTEXsetup[format={\raggedright\large\bfseries}]{section}
\titleformat{\subsection}[runin]{\normalfont\bfseries}{\thesubsection.}{0.5em}{}[.]
\titleformat{\subsubsection}[runin]{\normalfont\bfseries}{\alph{subsubsection})}{0.5em}{}


\usepackage{annotate-equations}



\theoremstyle{definition}
\newtheorem{example}{例子}[section]
\newtheorem{definition}{定义}[section]
\newtheorem{theorem}{定理}[section]
\newtheorem{proposition}[theorem]{命题}
\newtheorem{prop}[theorem]{性质}
\newtheorem{corollary}[theorem]{推论}

\newenvironment{remark}{%
  \par\medskip
  \noindent
  \textbf{注:}
}{%
  \par\medskip
}

\newenvironment{solution}{%
  \par\medskip
  \noindent
  \textbf{解答:}
}{%
  \par\medskip
}

\newenvironment{solution*}{%
  \par\medskip
  \noindent 
  \color{gray}\small\textbf{提示或解答:}
}{%
  \par\medskip
}

\newenvironment{definition*}{%
  \par\medskip
  \noindent
  \textbf{定义:}
}{%
  \par\medskip
}

\newenvironment{lemma}{%
  \par\medskip
  \noindent
  \textbf{引理:}
}{%
  \par\medskip
}

\newenvironment{proposition*}{%
  \par\medskip
  \noindent
  \textbf{性质: }
}{%
  \par\medskip
}


\newtcolorbox{asidebox}{
  colback=gray!10,
  colframe=gray!60,
  fonttitle=\bfseries,
  title={Aside},
  breakable=true
}

\newtcolorbox{webaside}{
  colback=cyan!10,
  colframe=cyan!60,
  fonttitle=\bfseries,
  title={Web Demonstrate Aside},
  breakable=true
}

\usepackage{enumitem}

\setlist{nosep}

\setstretch{1.2}
\geometry{
    textheight=9in,
    textwidth=5.5in,
    top=1in,
    headheight=12pt,
    headsep=25pt,
    footskip=30pt
}

\usepackage{environ}
\usepackage[tikz]{bclogo}
\usepackage{tikz}
\usetikzlibrary{calc}
\NewEnviron{takeaway}
  {\par\medskip\noindent
  \begin{tikzpicture}
    \node[inner sep=0pt] (box) {\parbox[t]{.99\textwidth}{%
      \begin{minipage}{.3\textwidth}
      \centering\tikz[scale=5]\node[scale=3,rotate=30]{\bclampe};
      \end{minipage}%
      \begin{minipage}{.65\textwidth}
      \textbf{Takeaway Message}\par\smallskip
      \BODY
      \end{minipage}\hfill}%
    };
    \draw[red!75!black,line width=3pt] 
      ( $ (box.north east) + (-5pt,3pt) $ ) -- ( $ (box.north east) + (0,3pt) $ ) -- ( $ (box.south east) + (0,-3pt) $ ) -- + (-5pt,0);
    \draw[red!75!black,line width=3pt] 
      ( $ (box.north west) + (5pt,3pt) $ ) -- ( $ (box.north west) + (0,3pt) $ ) -- ( $ (box.south west) + (0,-3pt) $ ) -- + (5pt,0);
  \end{tikzpicture}\par\medskip%
}


\newcommand{\lecture}[6]{
   \begin{center}
   \framebox{
      \vbox{\vspace{2mm}
    \hbox to 6.28in { {\bf #1
		\hfill #6} }
       \vspace{4mm}
       \hbox to 6.28in { {\Large \hfill 第 #2 节: #3  \hfill} }
       \vspace{2mm}
       \hbox to 6.28in { {\it Lecturer: #4 \hfill Scribes: #5} }
      \vspace{2mm}}
   }
   \end{center}
   
   \vspace*{4mm}
}

\usepackage{marginnote}
\renewcommand*{\marginfont}{\color{gray}\ttfamily\small}
\usepackage{setspace}
\newcounter{paranum}[section]
\newcommand{\Par}[1]{\vspace{10pt}\noindent\textbf{\refstepcounter{paranum}\theparanum. }\textbf{#1}~~}
\newcommand{\lec}[1]{\reversemarginpar\marginnote{{\textbf{#1}}}}
\newcommand{\mn}[1]{\marginnote{{#1}}}
% \renewcommand{\algorithmcfname}{算法}
\usepackage[abspath]{currfile}

\newcommand{\incfig}[1]{\begin{center}\includegraphics[width=.4\textwidth]{figs/\currfilename/#1}\end{center}}
\newcommand{\incfigw}[1]{\begin{center}\includegraphics[width=.8\textwidth]{figs/\currfilename/#1}\end{center}}



\newcommand{\set}[1]{\{#1\}}
\newcommand{\stirling}[2]{\left\{{#1 \atop #2}\right\}}
\newcommand{\binomt}[2]{\left(\left({#1 \atop #2}\right)\right)}
\newcommand{\pf}[4]{#1_{#2}^{#3_{#4}}}
\newcommand{\pl}[4]{#1_{#2}{#3^{#4}}}
\newcommand{\ty}[3]{{#1} \equiv {#2} ~(\bmod {#3})}
\newcommand{\Z}{{\mathbb Z}}
\newcommand{\one}{\mathbf{1}}
\newcommand{\varsub}[2]{\stackrel{#1}{\stackrel{\rule{#2}{0.4pt}}{\rule{#2}{0.4pt}}}}
\newcommand{\dd}{\mathrm{d}}
\newcommand{\Ep}[1]{\mathbb E\left(#1\right)}

\renewcommand{\red}[1]{{{\color{red}#1}}}
\newcommand{\teal}[1]{{{\color{teal}#1}}}
\renewcommand{\blue}[1]{{{\color{blue}#1}}}
\newcommand{\purple}[1]{{{\color{purple}#1}}}
\DeclareMathOperator{\var}{Var}
\newcommand{\E}{\mathbb E}
\newcommand{\like}{$\blacktriangleright$}
\newcommand{\exrate}[1]{{[#1]~}}
\newcommand{\newword}[2]{{\textbf{#1(#2)}\index{#1}}}
\newcommand{\newenword}[1]{{\textbf{#1}\index{#1}}}


\begin{document}
\lecture{概率论}{2}{随机变量~期望}{尹⼀通}{张桄玮}{Spring 2024}
\section{随机变量}

随机变量一开始希望刻画那些``取值看似是随机''的变量. 但是, 在更加严格地定义它们之前, 我们就无法定义什么叫相同的随机变量. 比如$X,Y$是两个掷骰子的结果, 请问$X^2, XY$是相同的随机变量吗? $2X, X+Y$呢? 在给出定义之前, 我们不好回答! 

随机变量还可能是
\begin{itemize}
    \item 连续抛硬币, 直到正面朝上为止的次数;
    \item 从$M$个白球和$N-M$个黑球中(有/无放回)取出$n$个球的白球数目;
    \item $n$个顶点, 任意两点之间以概率$p$产生一条边的随机图的最小染色数;
    \item $[0,1]$中随机取一个数它的值. 
\end{itemize}

我们先从最简单的例子看一看. 

\begin{example}[掷骰子]
    投掷一枚骰子, 定义$X \in \Omega$为掷出来的结果, $Y \in \{ 0, 1 \}$表示它的奇偶性. 

    我们有如下的观察:
    \begin{center}
        $$
\begin{array}{|c|c|c|}
    \hline  \Omega \text{中的样本} &X \text{的值}& Y\text{的值}\\
    \hline \text{1点} & 1 & 1 \\
\hline \text{2点} & 2 & 0 \\
\hline \text{3点} & 3 & 1 \\
\hline \text { 4点 } & 4 & 0 \\
\hline \text { 5点 } & 5 & 1 \\
\hline \text { 6点 } & 6 & 0 \\
\hline
\end{array}
$$
    \end{center}
    可以发现, 随机变量只不过是把样本空间里面的元素映射到了实数. 

    好比说投掷两枚骰子, 样本空间是$\Omega \times \Omega$, 如果这随机变量是两次骰子的和, 那就是把这空间的每一个元素和一个实数相对应.
\end{example}

刚刚我们从样本空间$\Omega$来理解了这映射. 那么这映射应当满足一些约束条件. 具体地, 需要与事件集合$\Sigma$以及概率律$\Pr$有些联系. 所以我们给出下面的定义. 

\begin{definition}[随机变量]
    给出一概率空间$(\Omega, \Sigma, \Pr)$, 其上的\newword{随机变量}{random variable, r.v.}是一个函数$X:\Omega \to \R$, 满足$\forall x \in \R, \{ w \in \Omega: X(\omega) \leq  x \} \in \Sigma$. 
\end{definition}

这定义实际上说的是, 任给我一个实数值, 样本空间中那些满足$X(\omega)\leq x$的$\omega$构成的集合要在事件集合中. 比如, 我们的$\Sigma$-可测的事件集天然满足这一性质.\mn{当定义看上去很难懂的时候, 想办法把它读出来.} 

后续, 为了方便起见, 我们引入一些简化记号: 

\begin{itemize}
    \item $X\leq x(x \in \R)$表示事件集合$\{ \omega \in \Omega:X(\omega)\leq x \}$;
    \item $X > x(x \in \R)$表示事件集合$\{ \omega \in \Omega:X(\omega)> x \}$;
    \item $X \in S(S \subset \R$, 且是由有限个交、并生成的) 表示事件$\{ \omega \in \Omega: X(\omega) \in S \}$. 
\end{itemize}


由于实数的复杂性, 我们才特意规定了上面的$S$是由有限个交、并生成出来的集合. (不然你可能构造出千奇百怪的东西). 对于离散的随机变量$X:\Omega \to \Z$而言, $X \in S$中的限制条件就变成了$S \subset Z$了. 

\subsection{随机变量的分布}

既然随机变量是一个从样本空间到实数上的映射, 自然需要一个方法来直观地表述这一映射. 由于随机变量的值域通常是可以观测的, 在下面看到表示随机变量的过程中, 通常拿这一值域当做自变量. 因变量就是对应值域的可能的$\Omega$中元素的集合. 

令$X$为两个独立的掷骰子得到的点数之和. 于是我们可以画出这样的图像. 

\incfigw{distr-2-dice.jpg}

这些集合如果这样画上去还是有些太麻烦了. 我们干脆把这些集合(根据定义保证会在概率空间的事件集里面)塞到概率律$\Pr$中, 这样我们又得到了一个实数. 于是, 便可以用中学学过的处理$f:\R \to \R$的手段来直观表示(也就是上图右侧的数字).  

上述考量自然地揭示了分布的想法. 在离散的情形下, 我们可以使用概率质量函数定义分布. 

\begin{definition}[概率质量函数]
    随机变量$X:\Omega \to \R$被叫做离散的, 如果$X(\Omega)$是可数的. 

    对于一离散的随机变量$X$, 其\newword{概率质量函数}{probability mass function, pmf}$p_X:\R\to[0,1]$定义做
    \[
        P_X(x)=\Pr(X=x).
    \]
\end{definition}


但是, 在连续的情形, 单单盯着一个点谈论概率往往是没有意义的(因为任何一个点发生的概率都是0, 而$0 \cdot \infty$是未定式, 不会违反加起来为1的限制). 因此, 我们通常使用累积分布(看$x\leq X$的概率)来描述一个分布.

\begin{definition}[累积分布函数]
    随机变量$X$的\newword{累计分布函数}{cumulative distribution function}(或简单叫做分布函数)是一个映射: $F_X:\R \to [0, 1]$, 对应法则为
    \[
        F_X(x):=\Pr(X\leq x).
    \]
\end{definition}

关于$X$的一切概率都可以从$F_X$中得到. 因此, 一旦能够确定累积分布函数$F_X(x)$在后续中, 我们实际上便不再需要概率空间. 

现在就可以讨论什么叫两个随机变量相同了. 

\begin{definition}
    称两个随机变量$X, Y$\newword{相同分布}{identically distributed}. 如果两个变量具有相同分布, 那么$F_X = F_Y$.     
\end{definition}

从上面的定义来看, 对于离散的情况, 随机变量$X$的CDF就是$F_X(y):=\sum_{x\leq y}p_X(x)$. 如果我们能把CDF表示成某个可积的积分\mn{实际上这积分并不一定Riemann可积, 有可能是Lebesgue可积}的形式: $F_X(y)=\operatorname{Pr}(X \leq y)=\int_{-\infty}^y f_X(x) d x$, 我们就称它为连续随机变量. 

上面的叙述, 实际上暗示了有些随机变量既不是离散的, 也不是连续的. 但是这里先不谈这些. 

累积分布函数还满足一些(显而易见的)性质. 

\begin{prop}[累积分布函数的性质]
    累积分布函数满足如下的性质: 
    \begin{enumerate}
        \item 单调性: $\forall x, y\in \R$, 如果$x\leq y$, 那么$F_X(x)\leq F_Y(y)$.
        \item 有界性: $\lim_{n \to -\infty} F_X(x)=0$, 而且$\lim_{x \to \infty} F_X(x)=1$. 
    \end{enumerate}
\end{prop}

可以说, 这两条分布的性质直接继承了概率律函数$\Pr$的属性. 

\subsection{独立性}

我们上次在介绍概率空间的时候说了事件的独立性. 描述随机变量号称可以``不用再考虑概率空间''的分布函数当然也要提一提. 

\begin{definition}[随机变量的独立性]
    对于随机变量$X_1, X_2, ..., X_n$, 我们说它们(互相)独立, 当且仅当
    任意的$x_1, x_2, ..., x_n$, 有
    $$
p_{\left(X_1, \ldots, X_n\right)}\left(x_1, \ldots, x_n\right)=\operatorname{Pr}\left(X_1\leq x_1 \cap \cdots \cap X_n\leq x_n\right)=p_{X_1}\left(x_1\right) \cdots p_{X_n}\left(x_n\right)
$$
\end{definition}

当然上述定义并不好用(!) 对于接下来我们要考虑的离散型随机变量, 定义就简化成了
\begin{itemize}
    \item 两个离散随机变量$X, Y$是独立的, 当且仅当对于任何的两个数值$x, y$, 有$X=x, Y=y$两个事件也是独立的. 
    \item 如果有一组离散随机变量$X_1, X_2, ..., X_n$,称它们是随机的, 当且仅当对于任意的$x_1,x_2,..., x_n$而言, 事件$X_1=x_1, X_2=x_2, ..., X_n = x_n$是独立的. 换言之, 就是PMF可以直接乘起来. 即
        $$
p_{\left(X_1, \ldots, X_n\right)}\left(x_1, \ldots, x_n\right)=\operatorname{Pr}\left(X_1=x_1 \cap \cdots \cap X_n=x_n\right)=p_{X_1}\left(x_1\right) \cdots p_{X_n}\left(x_n\right)
$$
\end{itemize}

\subsection{随机向量}

有了一个随机变量, 可否考虑一系列随机变量? 即, 我们有$X_1, X_2, X_3, ...$, 他们都是随机变量. 为了方便起见, 干脆把它们放在一起作为一个整体来考虑. 叫做随机向量. 

\begin{definition}[随机向量]
    给定一个概率空间$(\Omega, \Sigma, \operatorname{Pr})$, 某\newword{随机向量}{random vector}$\boldsymbol X$记作$\boldsymbol{X}:=\left(X_1, \ldots, X_n\right)$, 其中每一个元素$X_i$都是定义在概率空间$(\Omega, \Sigma, \operatorname{Pr})$上的随机变量. 

\end{definition}

\begin{example}
    比如我们有两个随机变量, $X, Y$. $X$可以取$x_1, x_2, x_3, x_4$; $Y$可以取$y_1, y_2, y_3$. 那么$(X,Y)$就是一个随机向量. 
\end{example}

如需寻求一个直观的表示, 对于离散的随机变量而言, 当然可以枚举每一种可能的组合. 并且定义联合质量函数.

\begin{definition}[联合质量函数]
    对于离散的随机变量而言, 其\newword{联合质量函数}{joint mass function}被定义作
    $$
p_X\left(x_1, \ldots, x_n\right)=\operatorname{Pr}\left(X_1=x_1 \cap \cdots \cap X_n=x_n\right)
$$
\end{definition}

\begin{example}
    对于上例子, 若把全部的可能组合写出来, 实际上可以画出一张表. 比如
    \begin{center}
        \begin{tabular}{|c|c|c|c|c|}
\hline $Y \backslash X$ & ${x}_1$ & ${x}_2$ & ${x}_3$ & ${x}_4$ \\
\hline$y_1$ & $\frac{4}{32}$ & $\frac{2}{32}$ & $\frac{1}{32}$ & $\frac{1}{32}$ \\
\hline$y_2$ & $\frac{3}{32}$ & $\frac{6}{32}$ & $\frac{3}{32}$ & $\frac{3}{32}$ \\
\hline$y_3$ & $\frac{9}{32}$ & 0 & 0 & 0 \\
\hline
\end{tabular}
    \end{center}
\end{example}

同样对于非离散的情形. 仍然可以通过联合累计密度函数表示. 

\begin{definition}[联合累积密度函数]
    一个随机向量的\newword{联合累积密度函数}{joint CDF}是$F_X:\R^n \to [0,1]$, 其对应法则为
    $$
F_X\left(x_1, \ldots, x_n\right)=\operatorname{Pr}\left(X_1 \leq x_1 \cap \cdots \cap X_n \leq x_n\right)
$$

\end{definition}

如果有选择性地忽略某一变量, 即按照行列相加, 会得到边缘分布. 如下所示. 
\begin{center}
\begin{tabular}{|c|c|c|c|c|c|}
\hline ${Y}\backslash {X}$ & ${x}_1$ & $x_2$ & $x_3$ & $x_4$ & $p_y(y) \downarrow$ \\
\hline$y_1$ & $\frac{4}{32}$ & $\frac{2}{32}$ & $\frac{1}{32}$ & $\frac{1}{32}$ & $\frac{8}{32}$ \\
\hline$y_2$ & $\frac{3}{32}$ & $\frac{6}{32}$ & $\frac{3}{32}$ & $\frac{3}{32}$ & $\frac{15}{32}$ \\
\hline$y_3$ & $\frac{9}{32}$ & 0 & 0 & 0 & $\frac{9}{32}$ \\
\hline$p_X(x) \rightarrow$ & $\frac{16}{32}$ & $\frac{8}{32}$ & $\frac{4}{32}$ & $\frac{4}{32}$ & $\frac{32}{32}$ \\
\hline
\end{tabular}
\end{center}

这样便得到了一个只含有$X$(或$Y$)的随机变量的质量函数, 将二维化为了一维. 像这样的分布叫做边缘分布, 因为其总是在表格的边上. 

\begin{definition}[边缘分布]
    对于一个随机向量$\left(X_1, \ldots, X_n\right)$, 其边缘分布由$$
p_{X_i}\left(x_i\right):=\sum_{x_1, \ldots, x_{i-1}, x_{i+1}, \ldots, x_n} p_{\left(X_1, \ldots, X_n\right)}\left(x_1, \ldots, x_n\right)
$$
定义. 

\end{definition}

\section{离散随机变量}

接下来我们主要考虑整数值的随机变量$X: \Omega \to \Z$. 根据先前的定义, 其PMF $p_X:\Z\to[0,1]$由$p_X(k)=\operatorname{Pr}(X=k)$给出. 我们可以把这样的表示解读做
\begin{itemize}
    \item 柱状图: $p_X$描绘了概率分布的直方图.
\item 向量: 如果$R:=$随机变量$X$的值域, 那么$p_X\in [0,1]^R$可以看做一个向量, 满足$||p_X(x)||_1=1$. 
\end{itemize}

一个随机变量的函数也是一个随机变量. 就像我们创造复合函数一样. 譬如我们有$Y=f(X),$ 那么$$
p_Y(y)=\sum_{x: f(x)=y} p_X(x)
$$
\subsection{若干离散随机变量及其分布} 下面来看若干离散随机变量及其分布. 

\subsubsection{Bernoulli随机变量} Bernoulli试验只可能产生两个结果. 这个实验的结果只可能是是``成功''或者``失败''. 倘若将``成功''记作1, ``失败''记作0, 并用随机变量表示之, 便得到了Bernouli随机变量. 

\begin{definition}[Bernouli随机变量]
    Bernouli随机变量$X$是在$\{ 0, 1 \}$中取值的随机变量, 满足
    $$
p_X(k)=\operatorname{Pr}(X=k)= \begin{cases}p & \text { if } k=1 \\ 1-p & \text { if } k=0\end{cases}
$$
其中$p\in [0,1]$是参数. 
\end{definition}

Bernouli随机变量通常用于一个事件发生与否的\newword{指示器}{indicator}. 例如
$$
X=I(A)=\left\{\begin{array}{ll}
1 & \text { 如果 } A \text { 发生 } \\
0 & \text { 否则 }
\end{array} \quad \text {. 这是一个参数为$\Pr(A)$的Bernouli r.v.} \operatorname{Pr}(A)\right.
$$


\subsubsection{几何分布} 不断地投一枚硬币, 第一次抛出正面的时候, 我们抛出了几次? 这也是一个随机变量. 其分布类似几何分布. 

\begin{definition}[几何分布随机变量]
    随机变量$X$是第一次独立同分布的Bernoulli实验成功的时候, 做实验的次数. 其取值范围是$\{ 1,2,3, ...,  \}$, 其分布是
    $$
p_X(k)=\operatorname{Pr}(X=k)=(1-p)^{k-1} p, \quad k=1,2, \ldots
$$
其中$p$是一次Bernouli实验成功的概率, 是一个参数. 
\end{definition}

若一随机变量服从这样的分布, 简单记作$X \sim \operatorname{Geo}(p)$.

需要注意的是, 几何分布没有记忆性. 也就是说, 无论从何时开始, 都和从第一次开始的分布无异.  

\begin{prop}
   几何随机变量$X \sim \operatorname{Geo}(p)$不具有记忆性. 即对$k \geq 1, n \geq 0$, 
   $$\operatorname{Pr}(X=k+n \mid X>n)=\operatorname{Pr}(X=k).$$


\end{prop}

\begin{proof}
    只要回到条件概率的定义证明即可.
    $$
\begin{gathered}
\operatorname{Pr}(X=k+n \mid X>n)=\frac{\operatorname{Pr}(X=k+n)}{\operatorname{Pr}(X>n)}=\frac{(1-p)^{n+k-1} p}{\sum_{k=n}^{\infty}(1-p)^k p} \\
=\frac{(1-p)^{k-1} p}{\sum_{k=0}^{\infty}(1-p)^k p}=(1-p)^{k-1} p
\end{gathered}
$$
\end{proof}


顺带一提, 几何分布是值域为$\{ 1,2, ... \}$的离散随机变量中\red{唯一}一个没有记忆性的分布. 

\subsubsection{二项分布} 二项分布是$n$次抛硬币中, 抛出正面的个数的分布. 

\begin{definition}[二项随机变量]
    在$n$次独立的参数为$p$的Bernouli实验中, 成功的次数称为\newword{二项随机变量}{Binomial r.v.}. 其分布称为\newword{二项分布}{binomial distribution}, 若一随机变量$X$服从二项分布, 则可简记为$X\sim B ( n, p)$(或$X\sim \text{Bin}(n,p)$).
\end{definition}

二项分布的取值范围在$\{ 0,1, ..., n \}$, 并且有
$$
p_X(k)=\operatorname{Pr}(X=k)=\binom{n}{k} p^k(1-p)^k, \quad k=0,1, \ldots, n
$$





\subsubsection{Poisson分布} 在实际生活中, 通常面对大部分的情况有二项分布的$n \to \infty$, 但是$np=\lambda$是一个常数. 通过计算, 我们知道
$$
p_{\operatorname{Bin}(n, \lambda / n)}(k)=\binom{n}{k}\left(\frac{\lambda}{n}\right)^k\left(1-\frac{\lambda}{n}\right)^{n-k}=\frac{n}{n} \frac{n-1}{n} \ldots \frac{n-k+1}{n} \cdot \frac{\lambda^k}{k!}\left(1-\frac{\lambda}{n}\right)^n\left(1-\frac{\lambda}{n}\right)^{-k} \approx \frac{\lambda^k}{k!} \mathrm{e}^{-\lambda}
$$
这就是Poisson分布. 

\begin{definition}[Possion分布]
    某一Possion随机变量$X$取值范围为$\{ 0,1,2 ... \}$, 其服从的分布为
    $$
p_X(k)=\operatorname{Pr}(X=k)=\mathrm{e}^{-\lambda} \frac{\lambda^k}{k!}, \quad k=0,1,2, \ldots
$$
\end{definition}

可以验证, 这是一个良定义(没有与定义中描述的相冲突)的分布. 因为它满足$\sum_{k=0}^{\infty} \mathrm{e}^{-\lambda} \frac{\lambda^k}{k!}=1$. 如果某个随机变量服从Poisson分布, 可以简单记作$X \sim \operatorname{Pois}(\lambda)$. 


另外指出, 独立的Possion随机变量的和也是Possion随机变量. 这是由于二项分布的类比: $X \sim \operatorname{Bin}\left(n_1, p\right), Y \sim \operatorname{Bin}\left(n_2, p\right) \Longrightarrow X+Y \sim \operatorname{Bin}\left(n_1+n_2, p\right)$

我们来证明相互独立的$X, Y$, $X \sim \operatorname{Pois}\left(\lambda_1\right), Y \sim \operatorname{Pois}\left(\lambda_2\right) \Longrightarrow X+Y \sim \operatorname{Pois}\left(\lambda_1+\lambda_2\right)$.

\begin{proof}
    $$
\begin{aligned}
& p_{X+Y}(k)=\operatorname{Pr}(X+Y=k)=\sum_{i=0}^k \operatorname{Pr}(X=i \cap Y=k-i)=\sum_{i=0}^k p_X(i) p_Y(k-i) \\
& =\sum_{i=0}^k \frac{\mathrm{e}^{-\lambda_1} \lambda_1^i}{i!} \frac{\mathrm{e}^{-\lambda_2} \lambda_2^{k-i}}{(k-i)!}=\frac{\mathrm{e}^{-\left(\lambda_1+\lambda_2\right)}}{k!} \sum_{i=0}^k\binom{k}{i} \lambda_1^i \lambda_2^{k-i}=\frac{\mathrm{e}^{-\left(\lambda_1+\lambda_2\right)}\left(\lambda_1+\lambda_2\right)^k}{k!}
\end{aligned}
$$
\end{proof}

\subsubsection{多项式分布} 这是对二项分布的推广. 假设有$n$个球扔进了$m$个桶里面, 每个球扔进哪个桶里面是随机的, 满足第$i$个桶接收到这个球的概率是$p_i$($p_1+p_2+ ...+p_m=1$). 我们用一组随机变量$(X_1, X_2, ..., X_n)$表示第$i$个桶恰好收到的$X_i$个球. 那么$(X_1, X_2, ..., X_m)$从$(k_1, k_2, ..., k_m) \in \{ 0,1, ..., n \}^m$中取值, 且$k_1+k_2+ ... + k_m = n$. 且其分布$p_{\left(X_1, \ldots, X_m\right)}\left(k_1, \ldots, k_m\right)$应该满足什么样的规律呢? 

我们首先从$n$个球里面选出$k_1$个(每个的概率为$p_1$), 然后从$n-k_1$当中选出$k_2$个(每个的概率是$p_2$), ..., 就得到了如下的式子: 

$$
\begin{aligned}
 & \underbrace{{n \choose k_{1}}p_{1}^{k_{1}}}_{\text{第一个桶放入}k_{1}\text{个球的概率}}\underbrace{{n-k_{1} \choose k_{2}}p_{2}^{k_{2}}}_{\text{第二个桶放入}k_{2}\text{个球的概率}}{n-k_{1}-k_{2} \choose k_{3}}p_3^{k_3}\cdots{n-k_{1}-k_{2}-\cdots-k_{m-1} \choose k_{m}}p_m^{k_m}\\
 &\varsub{\text{组合数定义}}{1cm}  \frac{n!}{k_{1}!(n-k_{1})!}\frac{(n-k_{1})!}{k_{2}!(n-k_{1}-k_{2})}\cdots\frac{(n-k_{1}-\cdots-k_{m-1})!}{k_{m}!(n-k_{1}-k_{2}-\cdots-k_{m-1}-k_{m})!}p_{1}^{k_{1}}\cdots p_{m}^{k^{m}}\\
 &  \varsub{\text{约分}}{1cm}  \frac{n!}{k_{1}!\cancel{(n-k_{1})!}}\frac{\cancel{(n-k_{1})!}}{k_{2}!\cancel{(n-k_{1}-k_{2})}}\cdots\frac{\cancel{(n-k_{1}-\cdots-k_{m-1})!}}{k_{m}!\underbrace{(n-k_{1}-k_{2}-\cdots-k_{m-1}-k_{m})!}_{\text{等于0}}}p_{1}^{k_{1}}\cdots p_{m}^{k^{m}}\\
 &=  \frac{n!}{k_{1}!k_{2}!\cdots k_{m}!}p_{1}^{k_{1}}\cdots p_{m}^{k^{m}}
\end{aligned}
$$

所以我们得到: 
$$
p_{\left(X_1, \ldots, X_m\right)}\left(k_1, \ldots, k_m\right)=\operatorname{Pr}\left(\bigcap_{i=1}^m\left(X_i=k_i\right)\right)=\frac{n!}{k_{1}!k_{2}!\cdots k_{m}!} p_1^{k_1} p_2^{k_2} \cdots p_m^{k_m}
$$

实际上, 刚刚推导的正是多重组合数. 对于多重组合数, 有时候也可以记${n \choose k_{1},k_{2},k_{3},\cdots,k_{m}}$作``把$k_1+\cdots+k_m$个球放进$m$个桶里面, 第一个桶里面有$k_1$个球, 第二个桶里有$k_2$个球, ..., 第$m$个桶里有$k_m$个球的方案数. '' 

其计算公式是
$$
{n \choose k_{1},k_{2},k_{3},\cdots,k_{m}}=\frac{n!}{k_{1}!k_{2}!\cdots k_{m}!}.
$$
如果考察这分布的边缘分布, 任意的$1\leq i\leq m$, 边缘分布$X_i$服从$\text{Bin}(n, p_i)$. 


\subsubsection{Possion高维分布} 将多项分布推向极限, 如果$(Y_1, ..., Y_m)$中的每个随机变量都服从独立的Possion分布($Y_i \sim \text{Pois}(\lambda_i)$), 并且$\lambda_i = np_i$, 那么实际上高维的Possion分布和多项分布是同分布的. 

\begin{prop}
    $\boldsymbol X=(X_1, ..., X_n)$遵循参数为$m, n, p_1+p_2+ ... +p_m$的多维分布. $\boldsymbol Y=(Y_1, Y_2, ..., Y_n)$满足独立的$\forall i, Y_i \sim \text{Pois}(\lambda_i), \lambda_i=np_i$. 如果$\sum_{i=1}^mY_i = n$, 那么$\boldsymbol X$和$\boldsymbol Y$同分布. 
\end{prop}

\begin{proof}
    $$
\begin{aligned}
& \operatorname{Pr}\left[\left(Y_1, \ldots, Y_m\right)=\left(k_1, \ldots, k_m\right) \mid Y_1+\cdots+Y_m=n\right]=\left(\prod_{i=1}^m \frac{\mathrm{e}^{-n p_i\left(n p_i\right)^{k_i}}}{k_{i}!}\right) /\left(\frac{\mathrm{e}^{-n} n^n}{n!}\right) \\
& =\frac{n!}{k_{1}!\cdots k_{m}!} p_1^{k_1} \cdots p_m^{k_m}=\operatorname{Pr}\left[\left(X_1, \ldots, X_m\right)=\left(k_1, \ldots, k_m\right)\right] \\
&
\end{aligned}
$$
\end{proof}


\subsection{构造随机变量的方法}

事实上, 从上面可以看出, 一般有两种方法构造随机变量. 第一种是\blue{对已有的随机变量做一个函数映射}. 如已经知道了$X_1, X_2, ..., X_n$, 可以通过$Y=f(X_1, X_2, \cdots, X_n)$构造这一新的随机变量$Y$. 如二项分布随机变量就是$n$个独立的Bernoulli随机变量求和得到的. 

另一个方法是为考虑随机变量序列$X_1, X_2, ..., X_T$的\blue{\newword{停止时间}{stopping time}$T$}, 把$T$当做随机变量. 如几何分布中考虑的``Bernoulli试验的第一次成功的时候, 做实验的次数. ''

\begin{definition}[随机变量的停止时间]
    对于随机变量序列$X_1, X_2, ...$, 我们说这组随机变量的\newword{停止时间}{stopping time}是$T$(也是一个随机变量), 当且仅当对于所有的$t\geq 1, T=t$仅由$X_1, X_2, ..., X_t$决定.  
\end{definition}

上述定义表示我们不再考虑停止时间$X_t$之后的那些元素造成的影响. 自然有在某一随机变量之后停止之意. 

\subsubsection{独立随机变量和的分布} 如上例, 独立随机变量的和的分布非常常见. 我们可以为它导出一个通用的公式. 比如$X, Y$是两个独立的离散随机变量, 我们需要得到$X+Y$的分布. 

只要求出$X+Y=1$的概率, $X+Y=2$的概率, ..., 我们就可以得到这一随机变量的分布. 我们若要求出$X+Y=z$的概率, 只需要使用全概率公式, 先固定住$X=x$, 然后使用独立性, 最后把所有可能的$x$加起来.  

$$
\begin{aligned}
 p_{X+Y}(z)&=\operatorname{Pr}(X+Y=z)=\sum_x \operatorname{Pr}(X=x \cap Y=z-x) \\
& =\sum_x p_X(x) p_Y(z-x)=\sum_y p_X(z-y) p_Y(y)
\end{aligned}
$$

如果我们把这个函数看做一个整体, 而不是看做输入某个固定的值之后如何计算, 我们便说: ``$p_{X+Y}$的PMF就是$p_X$的PMF和$p_Y$的PMF做了一个\newword{卷积}{convolution}''. 记作
$$
p_{X+Y}=p_X * p_Y
$$

\begin{example}
    对于二项分布而言是若干个i.i.d. Bernouli随机变量的和. 假设一次成功概率为$p$, 那么根据上述的公式照样可以推出二项分布: 
    $$
\begin{aligned}
& p_{X_1+\cdots+X_n}(k)=p \cdot p_{X_1+\cdots+X_{n-1}}(k-1)+(1-p) \cdot p_{X_1+\cdots+X_{n-1}}(k) \\
= & \binom{n-1}{k-1} p^k(1-p)^{n-k}+\binom{n-1}{k} p^k(1-p)^{n-k}=\binom{n}{k} p^k(1-p)^{n-k}
\end{aligned}
$$
\end{example}


\subsubsection{另一个停止时间随机变量的例子} 我们来考虑负二项分布. 它考虑的是成功概率为$p$的iid. Bernouli实验中成功$r$次这一段时间中失败的次数. 先记为随机变量$X$. 

\begin{definition}[负二项分布]
    负二项随机变量$X$在$\{ 0,1,2 ,... \}$取值, 其分布为
    $$
p_X(k)=\operatorname{Pr}(X=k)=\binom{k+r-1}{k}(1-p)^k p^r=(-1)^k\binom{-r}{k}(1-p)^k p^r, k=0,1,2, ...
$$

其中$r, p$是参数. 
\end{definition}

\section{期望及其计算} 回顾中学定义的期望, 用随机变量的语言重述如下.
\begin{definition}[期望]
    一个离散型随机变量的\newword{期望}{expectation}定义做
    $$
\mathbb{E}[X]=\sum_x x p_X(x)
$$
其中, $p_X$表示$X$的pmf, 求和指标$x$取遍$p_X(x)>0$的$x$. 
\end{definition}

需要注意的是, 期望不总是存在. 如对于正整数$k$, 定义$p_X\left(2^k\right)=2^{-k}$. 其期望并不是一个有限数! 

下面将介绍三种常见的期望计算方法. 如果我们发现期望很难算, 那么我们可以采取估计的方式. 在后续的内容中, 将说明计算的期望有什么含义. 

\subsection{直接计算} 根据期望的定义, 我们可以直接计算. 

\begin{example}[指示器变量的期望]
    对于一个成功概率为$p$的Bernouli随机变量, 其期望为
    $$
\mathbb{E}[X]=0 \cdot(1-p)+1 \cdot p=p
$$

如果指示某事件$A$发生与否的随机变量$I(A)$, $I(A)$取得1表示事件$A$发生, 否则表示事件没有发生. 那么
\[
    \E[X]=0\cdot \Pr(A^c)+1\cdot \Pr(A).
\]
这例子看上去简单, 但是后续一些性质可以把复杂的任务拆成如此简单的求期望的内容.     

\end{example}

\begin{example}[Possion随机变量]
    如果$X \sim \operatorname{Pois}(\lambda)$, 那么
    $$
\begin{aligned}
\mathbb{E}[X] & =\sum_{k \geq 0} k \frac{\mathrm{e}^{-\lambda} \lambda^k}{k!} \\
& =\sum_{k \geq 1} \frac{\mathrm{e}^{-\lambda} \lambda^k}{(k-1)!} \\
& =\sum_{k \geq 0} \frac{\mathrm{e}^{-\lambda} \lambda^{k+1}}{k!}=\lambda \sum_{k \geq 0} \frac{\mathrm{e}^{-\lambda} \lambda^k}{k!} \\
& =\lambda
\end{aligned}
$$    
\end{example}

既然随机变量经过某个函数之后也是随机变量, 那么这个新的随机变量的期望是什么? 下面的命题展示了随机变量的函数的期望怎么求.

\begin{prop}
对于$f:\R \to \R$, $X$和$\boldsymbol X=\{ X_1,X_2,..., X_n \}$, 我们有
\begin{itemize}
    \item $\mathbb{E}[f(X)]=\sum_x f(x) p_X(x)$
    \item $\mathbb{E}\left[f\left(X_1, \ldots, X_n\right)\right]=\sum_{\left(x_1, \ldots, x_n\right)} f\left(x_1, \ldots, x_n\right) p_X\left(x_1, \ldots, x_n\right)$
\end{itemize}
    
\end{prop}

\begin{proof}
    令$Y=f(X_1, X_2, ..., X_n)$, 那么
    $$
\begin{aligned}
\mathbb{E}\left[f\left(X_1, \ldots, X_n\right)\right]= & \sum_y y \operatorname{Pr}(Y=y)=\sum_y y \sum_{\left(x_1, \ldots, x_n\right) \in f^{-1}(y)} \operatorname{Pr}\left(\left(X_1, \ldots, X_1\right)=\left(x_1, \ldots, x_n\right)\right) \\
& =\sum_{\left(x_1, \ldots, x_n\right)} f\left(x_1, \ldots, x_n\right) \operatorname{Pr}\left(\left(X_1, \ldots, X_1\right)=\left(x_1, \ldots, x_n\right)\right) \\
& =\sum_{\left(x_1, \ldots, x_n\right)} f\left(x_1, \ldots, x_n\right) p_X\left(x_1, \ldots, x_n\right)
\end{aligned}
$$
\end{proof}

\subsection{期望的线性性}

下面的定理叙述的是, \red{无论两个随机变量是否独立}, 期望的和等于和的期望. 

\begin{theorem}[期望的线性性]
    对于任意的$a,b\in R$, 以及随机变量$X, Y$有
    \begin{itemize}
        \item $\mathbb{E}[a X+b]=a \mathbb{E}[X]+b$;
        \item $\mathbb{E}[X+Y]=\mathbb{E}[X]+\mathbb{E}[Y]$
    \end{itemize}
    
\end{theorem}

\begin{proof}
    $$
\mathbb{E}[a X+b]=\sum_x(a x+b) p_X(x)=a \sum_x x p_X(x)+b \sum_x p_X(x)=a \mathbb{E}[X]+b
$$

$$
\begin{aligned}
\mathbb{E}[X+Y] & =\sum_{x, y}(x+y) \operatorname{Pr}((X, Y)=(x, y)) \\
& =\sum_x^{x, y} x \sum_y \operatorname{Pr}((X, Y)=(x, y))+\sum_y y \sum_x \operatorname{Pr}((X, Y)=(x, y)) \\
& =\sum_x x \operatorname{Pr}(X=x)+\sum_y y \operatorname{Pr}(Y=y)=\mathbb{E}[X]+\mathbb{E}[Y]
\end{aligned}
$$
\end{proof}

这可以推广到线性函数$f$和一组随机变量$X_1, ..., X_n$, 同样有$\mathbb{E}\left[f\left(X_1, \ldots, X_n\right)\right]=f\left(\mathbb{E}\left[X_1\right], \ldots, \mathbb{E}\left[X_n\right]\right)$, 而且不用关心它们这些随机变量的相关性.

\begin{example}[二项分布的期望]
上文提到, 要从定义计算二项随机变量$X \sim \operatorname{Bin}(n, p)$的期望, 必须计算$$\mathbb{E}[X]=\sum_{k=0}^n k\binom{n}{k} p^k(1-p)^{n-k}.$$ 

但是, 由于二项分布可以被看做一系列随机变量的和$X=X_1+\cdots+X_n$, 其中每一个$X_i$是iid. Bernouli随机变量. 于是根据期望的线性性有: 

$$\mathbb{E}[X]=\mathbb{E}\left[X_1\right]+\cdots+\mathbb{E}\left[X_n\right]=n p$$
    
\end{example}

\begin{example}[几何分布的期望]
    要从定义计算几何分布随机变量$X \sim \operatorname{Geo}(p)$, 必须计算
    $$\mathbb{E}[X]=\sum_{k \geq 1} k(1-p)^{k-1} p.$$ 

    但是, 可以用另一种方法看这问题: 令$X=\sum_{k>1} I_k$,其中$I_k\in \{ 0,1 \}$表示\blue{头$k-1$次试验是否都失败了}, 那么
$$
\mathbb{E}[X]=\sum_{k \geq 1} \mathbb{E}\left[I_k\right]=\sum_{k \geq 1}(1-p)^{k-1}=\frac{1}{p}
$$

\end{example}

\begin{example}[负二项分布的期望]
    按照定义, 对于服从参数为$r,p$的负二项分布的随机变量$X$, 应该计算
    $$
\mathbb{E}[X]=\sum_{k \geq 1} k\binom{k+r-1}{k}(1-p)^k p^r
$$

但是, $X$可以看做由若干个iid.几何随机变量组成. 即$X_1, X_2, ..., X_r$个参数为$p$的iid. 几何随机变量. 那么$X=\left(X_1-1\right)+\cdots+\left(X_r-1\right)$. 因此根据期望的线性性, 有
$$
\mathbb{E}[X]=\mathbb{E}\left[X_1\right]+\cdots+\mathbb{E}\left[X_r\right]-r=r(1-p) / p
$$
    
\end{example}

\begin{example}[超几何分布的期望]
    一个盒子中总共有$N$个球. 其中有$M$个是红球, $N-M$个是蓝球. 从中无放回地抽取$n$个球. 假设抽取红球的个数记为随机变量$X$, 那么我们说$X$服从参数为$N,M,n$的超几何分布. 

    对于一个服从参数为$N,M,n$的超几何分布的随机变量$X$, 按照定义, 我们应该计算
    $$
\mathbb{E}[X]=\sum_{k=0}^n k\binom{M}{k}\binom{N-M}{n-k} /\binom{N}{n}
$$
但是, 定义指示变量$X_i\in \{ 0, 1 \}$表示\blue{第$i$个红球被抽出来了}. 那么, $X=X_1+\cdots+X_M$. 每一个红球被抽出来的概率是$\binom{N-1}{n-1} /\binom{N}{n}=\frac{n}{N}$.因此使用期望的线性性, 有
$$
\mathbb{E}[X]=\mathbb{E}\left[X_1\right]+\cdots+\mathbb{E}\left[X_M\right]=\frac{n M}{N}
$$    
\end{example}


\begin{example}
    一只猴子在胡乱敲键盘(假设敲击每个字母的概率是均等的). 请问它敲击$m$次键盘之后, 期望出现了多少次``hamlet''?

    更加具体地, 假设我们有一个字母表$Q$, 其大小$|Q|=q$. 定义$s:=\left(s_1, \ldots, s_n\right) \in Q^n$表示字母表中由$n$个字母组成的串的集合. 对于我们要匹配的字符串是$\pi \in Q^k$, 定义随机变量$X$表示$\pi$在$s$中作为子串出现的次数. 
    要计算这个问题, 我们可以设置指示变量$I_i \in \{ 0, 1 \}$, 表示$\pi=\left(s_i, s_{i+1}, \ldots, s_{i+k-1}\right)$. 那么, 我们的随机变量就可以变为$X=\sum_{i=1}^{n-k+1} I_i$. 

    根据期望的线性性, 
    $$
\mathbb{E}[X]=\sum_{i=1}^{n-k+1} \mathbb{E}\left[I_i\right]=(n-k+1) q^{-k}
$$
可以发现, 出现的次数仅仅与要匹配的字符串的长度以及敲了几次键盘有关. 

如果这只猴子不断地敲击键盘, 直到最后的几个字符是``hamlet''停止. $X$是截止停止时间的时候敲击键盘的次数. 那么这样, 这期望的次数就要和这个串的性质决定了. 

\end{example}

\begin{example}[邮票收集者]
    现在去集邮. 假设现在有$n$种邮票, 每次去集邮的时候他会随机等概率给你一张. 在集齐的时候去集邮的次数作为随机变量$X$. 请问, 你期望要去多少次, 才能把所有的邮票集齐? 

    实际上, 这问题如果用球和盒子的比喻, 可以说作: ``向$n$个篮子里面不断一个一个地扔球, 直到所有的篮子都填满, 扔的球的个数作为随机变量$X$. ''

    可以设随机变量$X_i$表示当现在\red{有且仅有}$i-1$个非空的篮子的时候扔的球的数目. 实际上, $X_i$是参数为$p_i=1-\frac{i-1}{n}$的几何随机变量, 且$X=\sum_{i=1}^n X_i$. 

    那么根据期望的线性性, 有
    $$
\mathbb{E}[X]=\sum_{i=1}^n \mathbb{E}\left[X_i\right]=\sum_{i=1}^n \frac{n}{n-i+1}=n \sum_{i=1}^n \frac{1}{i}=n H(n)
$$
其中$H(n)$是调和级数. 
    
\end{example}

由于期望是概率的加权平均和, 我们可以枚举每一个事件把它拆做一个一个小事情. 

\begin{theorem}
    对于非负取值在$\{ 0,1,2, ... \}$的随机变量$X$, 有
    $$
\mathbb{E}[X]=\sum_{k=0}^{\infty} \operatorname{Pr}[X>k]
$$
成立. 
\end{theorem}

\begin{proof}
    这是一个双射. 可以交换求和记号证明. 
    $$
\mathbb{E}[X]=\sum_{x \geq 0} x \operatorname{Pr}[X=x]=\sum_{x \geq 0} \sum_{k=0}^{x-1} \operatorname{Pr}[X=x]=\sum_{k \geq 0} \sum_{x>k} \operatorname{Pr}[X=x]=\sum_{k \geq 0} \operatorname{Pr}[X>k]
$$
\end{proof}

\begin{proof}
    也可以使用期望的线性性. 定义$I_k \in \{ 0, 1 \}$表示随机变量$X>k$是否成立. 那么, $X=\sum_{k \geq 0} I_k$. 根据期望的线性性, 有
    $$\mathbb{E}[X]=\sum_{k \geq 0} \mathbb{E}\left[I_k\right]=\sum_{k \geq 0} \operatorname{Pr}[X>k]$$

\end{proof}

面对复杂的事件, 难免使用容斥原理. 假设我们有指示事件$A$发生与否的变量$I(A)\in \{ 0, 1 \}$, 那么可以知道:
\begin{itemize}
    \item $I\left(A^c\right)=1-I(A)$;
    \item $I(A \cap B)=I(A) \cdot I(B)$.
\end{itemize}

我们化并为交的容斥原理就可以使用了. 在上一节中说到的证明方法同样适用于这里. 

\begin{theorem}[容斥原理]
    对于$n$个事件$A_1, A_2, \ldots, A_n$, 有
    $$
I\left(\bigcup_{i=1}^n A_i\right)=\sum_{\varnothing \neq S \subseteq\{1, \ldots, n\}}(-1)^{|S|-1} I\left(\bigcap_{i \in S} A_i\right)
$$
\end{theorem}

期望的线性性帮助我们省去了很多很繁杂甚至不可能的计算. 但是, 在面对无穷多个随机变量的时候, 期望的线性性也有其限制和条件. 

假设有无穷多个随机变量$X_1, X_2, \ldots$, 满足级数$\sum_{i=1}^{\infty} \mathbb{E}\left[\left|X_i\right|\right]$绝对收敛的时候, 式子$\mathbb{E}\left[\sum_{i=1}^{\infty} X_i\right]=\sum_{i=1}^{\infty} \mathbb{E}\left[X_i\right]$才会成立. 

一个更加有趣的情况是随机个随机变量的和, 即, 假设$N$是一个非负整值随机变量, 以及有随机变量$X_1, X_2, \ldots, X_N$, 那么, $\mathbb{E}\left[\sum_{i=1}^N X_i\right]=\mathbb{E}[N] \mathbb{E}\left[X_1\right]$成立吗?

\subsection{一些不等式}

\subsubsection{Jensen不等式} 对于一般的非线性函数$f$, 以及随机变量$X$, 通常不满足$\mathbb{E}[f(X)]=f(\mathbb{E}[X])$. 但是, 如果我们知道了$f$是凸函数, 根据Jensen不等式, 有

$$
\begin{aligned}
f \text {是凸函数 } & \Longleftrightarrow f(\lambda x+(1-\lambda) y) \leq \lambda f(x)+(1-\lambda) f(y) \\
& \Longleftrightarrow \mathbb{E}[f(X)] \geq f(\mathbb{E}[X])
\end{aligned}
$$
这可以使用Taylor展开以及中值定理证明. 

\subsubsection{期望的单调性} 对于随机变量$X, Y$, 以及$c \in \R$, 
\begin{itemize}
    \item 如果$X\leq Y$ a.s.(almost surely, 即$\Pr(X<Y)=1$), 那么$\mathbb{E}[X] \leq \mathbb{E}[Y]$.
    \item 如果$X\leq c(X\geq c)$ a.s, 那么$\mathbb{E}[X] \leq c(\mathbb{E}[X] \geq c)$.
    \item $\mathbb{E}[|X|] \geq|\mathbb{E}[X]| \geq 0$.
\end{itemize}

\begin{proof}
    $$
\begin{aligned}
\mathbb{E}[X] & =\sum_x x \operatorname{Pr}(X=x)=\sum_x x \sum_y \operatorname{Pr}((X, Y)=(x, y)) \\
& =\sum_x x \sum_{y \geq x} \operatorname{Pr}((X, Y)=(x, y))=\sum_y \sum_{x \leq y} x \operatorname{Pr}((X, Y)=(x, y)) \\
& \leq \sum_y \sum_{x \leq y} y \operatorname{Pr}((X, Y)=(x, y)) \leq \sum_y y \operatorname{Pr}(Y=y)=\mathbb{E}[Y]
\end{aligned}
$$
\end{proof}

\subsubsection{平均原理} 平均原理说的是肯定有元素大于等于均值的元素. 即$\operatorname{Pr}(X \geq \mathbb{E}[X])>0$. 不然均值就无法维持. 

\section{条件分布与条件期望}

在取条件的时候, 实际上是更换我们讨论的样本空间. 因此, 当然可以在这个新的空间上面再次考察其分布. 比如得到其pmf(概率质量函数). 

\begin{definition}[条件分布]
    离散型随机变量$X$在$A$发生的条件下的概率密度函数(pmf)记作$p_{X|A}:\Z\to [0,1]$. 其对应法则为
    $$
p_{X \mid A}(x)=\operatorname{Pr}(X=x \mid A)
$$

\end{definition}

需要注意的是, $(X|A)$也是一个随机变量. 其分布完全由pmf $p_{X|A}$表示. 既然如此, 它自然可以用于期望的计算. 如$\mathbb{E}[X \mid A]=\sum_x x \operatorname{Pr}(X=x \mid A)$. 以及满足期望的各种性质(如最常用的, 线性性). 

\begin{definition}[条件期望]
    离散型随机变量$X$在\red{事件}$A$发生的条件下的\newword{条件期望}{conditional expectation}定义做
    $$
\mathbb{E}[X \mid A]=\sum_x x \operatorname{Pr}(X=x \mid A)
$$
为了满足概率空间的定义, 还需满足
\begin{itemize}
    \item $\operatorname{Pr}(A)>0$;
    \item 级数 $\sum_x x \operatorname{Pr}(X=x \mid A)$ 绝对收敛. 
\end{itemize}
\end{definition}

\begin{definition}[条件期望]
    对于两个\red{随机变量}$X, Y$, 条件期望$\mathbb{E}[X \mid Y]$是一个随机变量$f(Y)$, 其分布为当$Y=y$的时候
    $$
f(y)=\mathbb{E}[X \mid Y=y]
$$
\end{definition}

这样的定义自然可以推广到多于2个变量的情况. 

为什么要引入条件期望? 许多时候我们希望对样本空间做划分, 在那些划分过后的样本空间中, 往往就好求解期望了. 如果把每一个小块的期望值加起来, 结果上就等于全空间的期望了. 这就是下一个定理阐述的全期望定理. 

\begin{theorem}[全期望定理]
    假设随机变量$X$的期望存在, $B_1, B_2, \ldots, B_n$是样本空间$\Omega$的一个划分, 满足$\operatorname{Pr}\left(B_i\right)>0, \forall i$. 
    那么有
    $$
\mathbb{E}[X]=\sum_{i=1}^n \mathbb{E}\left[X \mid B_i\right] \operatorname{Pr}\left(B_i\right)
$$

\end{theorem}

\begin{proof}
    $$
\begin{aligned}
\mathbb{E}[X] & =\sum_x x \operatorname{Pr}(X=x)=\sum_x x \sum_{i=1}^n \operatorname{Pr}\left(X=x \mid B_i\right) \operatorname{Pr}\left(B_i\right) \\
& =\sum_{i=1}^n \operatorname{Pr}\left(B_i\right) \sum_x x \operatorname{Pr}\left(X=x \mid B_i\right)=\sum_{i=1}^n \mathbb{E}\left[X \mid B_i\right] \operatorname{Pr}\left(B_i\right)
\end{aligned}
$$
\end{proof}

有了这样的定理就会得出一个有趣的结论: $\mathbb{E}[\mathbb{E}[X \mid Y]]=\mathbb{E}[X]$. 这是因为
$$
\begin{array}{rlr}
\mathbb{E}[\mathbb{E}[X \mid Y]] & =\sum_y \mathbb{E}[X \mid Y=y] & \operatorname{Pr}(Y=y) \\
& =\mathbb{E}[X] \end{array}
$$


\subsection{快速排序的期望运行时间}

我们在《算法导论》的课程上了解了如下的随机快速排序算法. 使用自然语言大概可以看做\cref{algo:quick-sort}. 
\begin{algorithm}
    \caption{随机快速排序算法}
    \label{algo:quick-sort}
    {输入: 待排序的数组$S=[x_1, x_2, \cdots, x_n]$}
    
    {输出: 排序后的数组$S$.}
    \begin{itemize}
        \item 如果$S$只有一个或者零个元素, 返回$S$. 否则继续. 
        \item 随机选择$S$中的元素$s$作为基准元素. 
        \item 把$S$分为两个小的列表$S_1, S_2$. 其中, 任何一个$S_1$中的元素比$s$要小, 任何一个$S_2$中的元素比$s$要大.
        \item 对$S_1, S_2$进行快速排序. 
        \item 返回列表$[S_1, x, S_2]$.
    \end{itemize}
\end{algorithm}

我们声称: 每一次从元素中独立地随机选取基准元素, 那么对于任意的输入, 快速排序比较的期望次数为$2n\lg n + O(n)$. 

\begin{proof}
    设$y_1, y_2, \cdots, y_n$是输入值$x_1, x_2, \cdots, x_n$按照升序排列的结果. 我们定义$X_{ij}(i<j)$是一个随机变量. 如果在算法执行的某一时刻$y_i$和$y_j$发生了比较, $X_{ij}$取值为1, 否则为0. 那么比较的总次数满足
    $$
    X=\sum_{i=1}^{n-1}\sum_{j=i+1}^n X_{ij}
    $$
    根据期望的线性性, $\E {X}=\E {\sum_{i=1}^{n-1}\sum_{j=i+1}^n X_{ij}}=\sum_{i=1}^{n-1}\sum_{j=i+1}^n \E {X_{ij}}$. 

    由于$X_{ij}$只能取0和1, 是指示变量, $\E {X_{ij}}$是$X_{ij}$等于1的概率. 

    什么时候$y_i$和$y_j$会发生比较呢? 我们发现$y_i$和$y_j$发生比较, 当且仅当$y_i$或$y_j$是从集合$Y_{ij}=\{y_i, y_{i+1}, \cdots, y_{j-1},y_j\}$中选取的一个基准元素. 否则, 他们会被分在不同的子列表中, 因而不会比较. 

    由于我们的基准元素是独立选取的, 因此$y_i$和$y_j$是从$Y_{ij}$中选取的一个基准元素的概率, 也就是$X_{ij}$取1的概率, 是$2/(j-i+1)$. 也就是
    $$
    \begin{aligned}
        \E{X}&=\sum_{i=1}^{n-1} \sum_{j=i+1}^{n} \frac 2{j-i+1}\\
        &\varsub{k:=j-i+1}{1.5cm} \sum_{i=1}^{n-1}\sum_{k=2}^{n-i+1} \frac 2k = \sum_{k=2}^n \sum_{i=1}^{n+1-k} \frac 2k \\ 
        &= \sum_{k=2}^n (n+1-k) \frac 2k = \left((n+1)\sum_{k=2}^n \frac 2k\right)-2(n-1) \\
        &= (2n+2) \sum_{k=1}^n \frac 1k - 4n.
    \end{aligned} 
    $$
\end{proof}

\subsection{随机个随机变量的和的期望}

假设某种生物在生命的最后自动繁殖, 且每一代的存活时间相同. 每一个个体会繁殖的个数的均值为$\mu$. 现在请问你$n$代之后存在的生物量均值有多少. 

上述问题可以转换为一列随机变量$X_0, X_1, X_2, \ldots$由
$$
\begin{cases}
X_{0}=1\\
X_{n+1}=\sum_{j=1}^{X_{n}}\xi_{j}^{(n)}
\end{cases}
$$
定义. 其中$\xi_j^{(n)} \in \mathbb{Z}_{\geq 0}$是iid. rv. 服从均值$\mu=\mathbb{E}\left[\xi_j^{(n)}\right]$. 

这问题奇怪的地方在于, 随机变量的个数是随机的. 但是没关系, 我们可以把当前的随机变量$X_n$依据$X_{n-1}$的值划分成若干份: 即在$X_{n-1}=k$的条件下的期望. 最后把它们加起来. 
$$
\mathbb{E}\left[X_n \mid X_{n-1}=k\right]=\mathbb{E}\left[\sum_{j=1}^k \xi_j^{(n-1)} \mid X_{n-1}=k\right]=k \mu \Longrightarrow \mathbb{E}\left[X_n \mid X_{n-1}\right]=X_{n-1} \mu
$$

那么根据期望的期望等于它本身, 就得到了
$$
\mathbb{E}\left[X_n\right]=\mathbb{E}\left[\mathbb{E}\left[X_n \mid X_{n-1}\right]\right]=\mathbb{E}\left[X_{n-1} \mu\right]=\mathbb{E}\left[X_{n-1}\right] \cdot \mu=\mu^n
$$

这表明, 在第$n$代后代数量期望有$\mu^n$个.



% \section{练习题}

% 求解期望很多时候需要一些新奇的想法. 尤其是使用指示器变量的时候, 需要对问题做合理的拆分. 

% \begin{exercise}
% 一矿工被困在有三个门的矿井中, 第一个门通一隧道, 沿此隧道走2小时可到达安全
% 区; 第二个门通一隧道,  沿此隧道走3小时可回到原矿井中; 第三个门通一隧道, 沿此隧
% 道走5小时可回到原矿井中. 假定此矿工总是等可能地在三个门中选择一个, 用X表示矿
% 工到达安全区所用的时间, 求X的均值. 
% \end{exercise}
% \begin{solution}
%     假设$Y$表示矿工第一次选择的门. $\{Y=i\}$ 表示第一次选择第 $i$ 个门. 根据题意, $\Pr\{Y=1\}=\Pr\{Y=2\}=\Pr\{Y=3\}=\frac{1}{3}$. 
%     根据全期望公式, 有
%     $$
% \begin{aligned}
% \E(X) & \left.=\E[ \E(X \mid Y)\right] \\
% & =\frac{1}{3}(\E(X \mid Y=1)+\E(X \mid Y=2)+\E(X \mid Y=3)) \\
% & =\frac{1}{3}(2+3+\E(X)+5+\E(X))
% \end{aligned}
% $$
% 解得$\E(X)=10$. 
% \end{solution}


% \begin{exercise}
%     某个大楼有10层, 某次有25人在一楼搭乘电梯上楼. 假设每人都等可能地在2到10层中的任何一层出电梯, 并且出电梯与否互相独立. 同时在2到10层中没有人上电梯. 并且电梯只有在有人要出电梯的时候才停止. 求电梯停下的总次数的数学期望. 
% \end{exercise}
% \begin{solution}
%     每个人在第$i$层下的概率为$P_i=1/9, i=1,2,\cdots, 10$. 记第$k$个人在第$i$层下电梯记作$A_{k,i}$. 那么对于任意的$i$, $\Pr(A_{k,i})=1/9, \Pr({A_{k,i}}^c)=8/9. (k=1,2,\cdots, 25)$. 又因为$A_{1, i}, \cdots, A_{25, i}$相互独立, 那么第$i$层无人下电梯的概率为
% $$
% P\left(\prod_{i=1}^{25} {A_{k,i}}^c\right) = \prod_{i=1}^{25}P\left({A_{k,i}}^c\right) = \left(\frac89\right)^{25}, (k=1,2,\cdots, 25).
% $$

% 设$X_i$是指示第$i$层有没有人下的指示函数, 那么, $\Pr(X=0)=(8/9)^{25}$, $\Pr(X=1)=1-(8/9)^{25}$.

% 因此电梯停的总次数期望为$\E[X]=\sum_{i=2}^{10}X_i=9\times\left(1-(8/9)^{25}\right)$
% \end{solution}



\end{document}
