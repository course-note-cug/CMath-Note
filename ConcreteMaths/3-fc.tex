%!TEX TS-program = xelatex

\documentclass{ctexart}

\usepackage{amsmath, amsthm, amssymb, amsfonts}
\usepackage{thmtools}
\usepackage{graphicx}
\usepackage{setspace}
\usepackage{geometry}

\usepackage{float}
\usepackage{amsthm}
\usepackage{hyperref}
\usepackage{cleveref}
% \usepackage{mathabx}
\usepackage[utf8]{inputenc}
\usepackage[english]{babel}
\usepackage{framed}
\usepackage[dvipsnames]{xcolor}
\usepackage[skins,breakable]{tcolorbox}
\usepackage{awesomebox}
\usepackage{mathrsfs}  
\usepackage{xcolor}
\usepackage{wrapfig}
\usepackage{algorithm2e}
\RestyleAlgo{ruled}

\usepackage{pstricks-add}
\usepackage{epsfig}
\usepackage{pst-grad} % For gradients
\usepackage{pst-plot} % For axes
\usepackage[space]{grffile} % For spaces in paths
\usepackage{etoolbox} % For spaces in paths
\makeatletter % For spaces in paths
\patchcmd\Gread@eps{\@inputcheck#1 }{\@inputcheck"#1"\relax}{}{}
\makeatother

% Make SS at the beginning of a section

\makeatletter
%% See pp. 26f. of 'The LaTeX Companion,' 2nd. ed.
\def\@seccntformat#1{\@ifundefined{#1@cntformat}%
    {\csname the#1\endcsname\quad}%      default
    {\csname #1@cntformat\endcsname}}%   individual control
\newcommand{\section@cntformat}{\S\thesection\quad}
\newcommand{\subsection@cntformat}{\S\thesubsection\quad}
\makeatother % changes @ back to a special character

\usepackage{titlesec}

\CTEXsetup[format={\raggedright\large\bfseries}]{section}
\titleformat{\subsection}[runin]{\normalfont\bfseries}{\thesubsection.}{0.5em}{}[.]
\titleformat{\subsubsection}[runin]{\normalfont\bfseries}{\alph{subsubsection})}{0.5em}{}





\theoremstyle{definition}
\newtheorem{example}{例子}[section]
\newtheorem{definition}{定义}[section]
\newtheorem{theorem}{定理}[section]
\newtheorem{proposition}[theorem]{命题}
\newtheorem{prop}[theorem]{性质}
\newtheorem{corollary}[theorem]{推论}

\newenvironment{remark}{%
  \par\medskip
  \noindent
  \textbf{注:}
}{%
  \par\medskip
}

\newenvironment{solution}{%
  \par\medskip
  \noindent
  \textbf{解答:}
}{%
  \par\medskip
}

\newenvironment{solution*}{%
  \par\medskip
  \noindent 
  \color{gray}\small\textbf{提示或解答:}
}{%
  \par\medskip
}

\newenvironment{definition*}{%
  \par\medskip
  \noindent
  \textbf{定义:}
}{%
  \par\medskip
}

\newenvironment{lemma}{%
  \par\medskip
  \noindent
  \textbf{引理:}
}{%
  \par\medskip
}

\newenvironment{proposition*}{%
  \par\medskip
  \noindent
  \textbf{性质: }
}{%
  \par\medskip
}


\newtcolorbox{asidebox}{
  colback=gray!10,
  colframe=gray!60,
  fonttitle=\bfseries,
  title={Aside},
  breakable=true
}

\newtcolorbox{webaside}{
  colback=cyan!10,
  colframe=cyan!60,
  fonttitle=\bfseries,
  title={Web Demonstrate Aside},
  breakable=true
}

\usepackage{enumitem}

\setlist{nosep}

\setstretch{1.2}
\geometry{
    textheight=9in,
    textwidth=5.5in,
    top=1in,
    headheight=12pt,
    headsep=25pt,
    footskip=30pt
}

\usepackage{environ}
\usepackage[tikz]{bclogo}
\usepackage{tikz}
\usetikzlibrary{calc}
\NewEnviron{takeaway}
  {\par\medskip\noindent
  \begin{tikzpicture}
    \node[inner sep=0pt] (box) {\parbox[t]{.99\textwidth}{%
      \begin{minipage}{.3\textwidth}
      \centering\tikz[scale=5]\node[scale=3,rotate=30]{\bclampe};
      \end{minipage}%
      \begin{minipage}{.65\textwidth}
      \textbf{Takeaway Message}\par\smallskip
      \BODY
      \end{minipage}\hfill}%
    };
    \draw[red!75!black,line width=3pt] 
      ( $ (box.north east) + (-5pt,3pt) $ ) -- ( $ (box.north east) + (0,3pt) $ ) -- ( $ (box.south east) + (0,-3pt) $ ) -- + (-5pt,0);
    \draw[red!75!black,line width=3pt] 
      ( $ (box.north west) + (5pt,3pt) $ ) -- ( $ (box.north west) + (0,3pt) $ ) -- ( $ (box.south west) + (0,-3pt) $ ) -- + (5pt,0);
  \end{tikzpicture}\par\medskip%
}

\usepackage{marginnote}
\renewcommand*{\marginfont}{\color{gray}\ttfamily\small}
\usepackage{setspace}
\newcounter{paranum}[section]
\newcommand{\Par}[1]{\vspace{10pt}\noindent\textbf{\refstepcounter{paranum}\theparanum. }\textbf{#1}~~}
\newcommand{\lec}[1]{\reversemarginpar\marginnote{{\textbf{#1}}}}
\newcommand{\mn}[1]{\marginnote{{#1}}}
\renewcommand{\algorithmcfname}{算法}
\usepackage[abspath]{currfile}

\newcommand{\incfig}[1]{\begin{center}\includegraphics[width=.4\textwidth]{figs/#1}\end{center}}
\newcommand{\incfigw}[1]{\begin{center}\includegraphics[width=.8\textwidth]{figs/#1}\end{center}}
\newcommand{\set}[1]{\{#1\}}
\newcommand{\stirling}[2]{\left\{{#1 \atop #2}\right\}}
\newcommand{\binomt}[2]{\left(\left({#1 \atop #2}\right)\right)}
\newcommand{\pf}[4]{#1_{#2}^{#3_{#4}}}
\newcommand{\pl}[4]{#1_{#2}{#3^{#4}}}
\newcommand{\ty}[3]{{#1} \equiv {#2} ~(\bmod {#3})}
\newcommand{\Z}{{\mathbb Z}}
\newcommand{\one}{\mathbf{1}}
\newcommand{\varsub}[2]{\stackrel{#1}{\stackrel{\rule{#2}{0.4pt}}{\rule{#2}{0.4pt}}}}
\newcommand{\dd}{\mathrm{d}}
\newcommand{\Ep}[1]{\mathbb E\left(#1\right)}

\renewcommand{\red}[1]{{{\color{red}#1}}}
\newcommand{\teal}[1]{{{\color{teal}#1}}}
\renewcommand{\blue}[1]{{{\color{blue}#1}}}
\newcommand{\purple}[1]{{{\color{purple}#1}}}
\DeclareMathOperator{\var}{Var}
\newcommand{\E}{\mathbb E}
\newcommand{\like}{$\blacktriangleright$}
\newcommand{\exrate}[1]{{[#1]~}}
\newcommand{\newword}[2]{{\textbf{#1(#2)}\index{#1}}}
\newcommand{\newenword}[1]{{\textbf{#1}\index{#1}}}


\begin{document}

\section{上取整和下取整} 

\subsection{基本定义} 

\begin{definition}[上取整和下取整]
    $\forall x \in \mathbb{R}$, 定义下取整函数和上取整函数
    \begin{align*}
    \lfloor x\rfloor:= \text{最大的} \leqslant x \text{的整数}\\
    \lceil x\rceil := \text{最小的} \geqslant x\text{ 的整数}
    \end{align*}
    分别称他们为``上取整函数''和``下取整函数 ''.
    
\end{definition}

\begin{example}绘制出下取整函数的图像
    \incfig{floor}
   上取整函数的图像也是同理.  
\end{example}

\begin{prop}
    取整函数有如下的三个性质: 

    \begin{enumerate}
        \item $\lfloor x\rfloor=x \iff$ $x$是正数 $\iff $ $\lceil x\rceil=x$.
        \item (放缩) $x-1<\lfloor x\rfloor \leqslant x \leqslant\lceil x\rceil<x+1$.
        \item (对称)$\lfloor-x\rfloor=-\lceil x\rceil, \quad F x\rceil=-\lfloor x\rfloor$.
    \end{enumerate}
    
\end{prop}

\begin{prop}
    实际上, 上下取整函数是对一类不等式的缩写

    $$
\lfloor x\rfloor=n\left\{\begin{array}{lll}
    \iff & x-n \leqslant x & \text {(固定}x\text{,考察 } n \text { 的范围) } \\
\iff & n \leqslant x \leqslant n+1 & \text { (固定}n\text{, 考察 } x \text { 的范围 })
\end{array}\right.
$$

$$
\lceil x\rceil=n \begin{cases}\iff & x \leqslant n \leqslant x+1 \\ \iff & n-1<x \leqslant n .\end{cases}
$$

    
\end{prop}

\begin{corollary}
    若$n \in \mathbb{Z},\lfloor x+n\rfloor=\lfloor x\rfloor+n$.
    
\end{corollary}

\begin{remark}
    实际上, 这一个性质经常用于将整数(也许是配凑出来的)移入或者移出取整记号中. 
\end{remark}

为了更加彰显不等式的地位, 特别有下面的一个不等式.

\begin{prop}[上下取整的不等式]

$$ 
\begin{gathered}
x<n \Leftrightarrow\lfloor x\rfloor<n ; \quad n<x \Leftrightarrow n<\lceil x\rceil . \\
x \leqslant n \Leftrightarrow\lceil x\rceil \leqslant n ; \quad n \leqslant x \Leftrightarrow n \leqslant\lfloor x\rfloor .
\end{gathered}
$$ 

\end{prop}

这是一个重要的式子. 可以通过分类讨论的方法证明之. 

\begin{definition}[分数部分] 定义一个数的分数部分为$x-\lfloor x\rfloor$. 如果与集合的记号不相冲突的话, 可以记作$\set{x}$. 
    
\end{definition}

\begin{example}
    $\lfloor x+y\rfloor$是否永远等于$\lfloor x\rfloor+\lfloor y\rfloor$?
    实际上不是这样的. 将$x,y$写作$x=\lfloor x\rfloor+\{x\}, \quad y=\lfloor y\rfloor+\{y\}$, 那么
\[
    \lfloor x+y\rfloor=\lfloor x\rfloor+\lfloor y\rfloor+\lfloor\{x\}+\{y\}\rfloor
\]
    我们发现当且仅当$\{x\}+\{y\}<1$ 时, $\lfloor x+y\rfloor=\lfloor x\rfloor+\lfloor y\rfloor$. 否则由于$0 \leqslant\{x\}+\{y\}<2$, 其为$\lfloor x\rfloor+\lfloor y\rfloor+1$. 
\end{example}

\subsection{上取整, 下取整的复合} 我们首先考察一些基本例子得到的结果. 

\begin{example}
    $\lceil\lfloor x\rfloor\rceil=\lfloor x\rfloor ;\lfloor\lceil x\rceil\rfloor=\lceil x\rceil$. 直接将两者复合得到的结果是平凡的. 
    
\end{example}

\begin{example}
    证明或推翻: $\lfloor\sqrt{\lfloor x\rfloor}\rfloor=\lfloor\sqrt{x}\rfloor$.

首先尝试举反例, 但是$\pi, e, \phi, 1,2 \cdots$都是正确的, 于是考虑证明. 

我们的目标为想办法除去$\sqrt{ }$下的$\lfloor\rfloor$. 假设$m:=\lfloor\sqrt{\lfloor x\rfloor}\rfloor$, 解掉最外层的底可以得到
$$
m \leqslant \sqrt{\lfloor x\rfloor}<m+1
$$

$$
\begin{aligned}
& \Rightarrow m^2 \leqslant\lfloor x\rfloor<(m+1)^2 \\
& \Rightarrow m^2 \leqslant x<(m+1)^2 . \Longrightarrow m \leqslant \sqrt{x}<(m+1)^2 \\
& \Rightarrow m \leqslant\lfloor\sqrt{x}\rfloor<(m+1)^2 .
\end{aligned}
$$
    
\end{example}

考虑泛化上述的例子. 

\begin{theorem}
    令$f(x)$为一个连续, 单调递增的函数, 满足
    \[
        f(x) \text{是整数} \implies x\text{是整数}. 
    \]

    那么有
    $$
\lfloor f(x)\rfloor=\lfloor f(\lfloor x\rfloor)\rfloor,\lceil f(x)\rceil=\lceil f(\lceil x \rceil)\rceil .
$$
只要$f(x), f(\lfloor x\rfloor), f(\lceil x\rceil)$有定义. 
\end{theorem}

\begin{proof}
    $1^\circ$ 如果$x$是整数, 那么显然成立. 

    $2^\circ$ 若$x>\lfloor x\rfloor$, 由于$f$单调递增, 那么$f(x)>f(\lfloor x\rfloor)$. 然后两端同时取下取整符号, 由于下取整记号不减, 那么$\lfloor f(x)\rfloor \geqslant\lfloor f(\lfloor x\rfloor)\rfloor$. 我们接下来分类讨论, 以确定大于号的情形不成立. 
    \begin{itemize}
        \item 如果$\lfloor f(x)\rfloor =\lfloor f(\lfloor x\rfloor)\rfloor$, 那么这就是我们想要的. 
        \item 如果$\lfloor f(x)\rfloor>\lfloor f(\lfloor x\rfloor)\rfloor$, 那么由于$f$是一个连续的函数, 一定存在$y$, 使得$\lfloor x\rfloor \leqslant y<x$, 并且$f(y)=\lfloor f(\lfloor x\rfloor)\rfloor$. 根据$f$的性质, $y$一定也是整数. 但是$\lfloor x\rfloor$与$x$之间不存在另一个整数, 矛盾! 所以我们的假设不成立.  
    \end{itemize}
\end{proof}

\begin{corollary}
    $$
\left\lfloor\frac{x+m}{n}\right\rfloor=\left\lfloor\frac{\lfloor x\rfloor+m}{n}\right\rfloor,\left\lceil\frac{x+m}{n}\right\rceil=\left\lceil\frac{\lceil x\rceil+m}{n}\right\rceil .
$$
    
\end{corollary}

比如, $$
\lfloor\lfloor\lfloor x / 10\rfloor / 10\rfloor / 10\rfloor=\lfloor x / 1000\rfloor .
$$

\subsection{区间计数问题} 

如果我们用如下的简写记后面的区间:
$$
\begin{cases}
    [\alpha..\beta] & \alpha \leqslant x \leqslant \beta\\
    (\alpha..\beta]& \alpha<x \leqslant \beta \\
    [\alpha,\beta) & \alpha \leq x<\beta\\
    (\alpha,\beta)& \alpha<x<\beta
\end{cases}$$

我们的问题是这个集合里面包含了多少个整数. 

\end{document}
