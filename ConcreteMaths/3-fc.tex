%!TEX TS-program = xelatex

\documentclass{ctexart}

\usepackage{amsmath, amsthm, amssymb, amsfonts}
\usepackage{thmtools}
\usepackage{graphicx}
\usepackage{setspace}
\usepackage{geometry}

\usepackage{float}
\usepackage{amsthm}
\usepackage{hyperref}
\usepackage{cleveref}
% \usepackage{mathabx}
\usepackage[utf8]{inputenc}
\usepackage[english]{babel}
\usepackage{framed}
\usepackage[dvipsnames]{xcolor}
\usepackage[skins,breakable]{tcolorbox}
\usepackage{awesomebox}
\usepackage{mathrsfs}  
\usepackage{xcolor}
\usepackage{wrapfig}
\usepackage{algorithm2e}
\RestyleAlgo{ruled}

\usepackage{pstricks-add}
\usepackage{epsfig}
\usepackage{pst-grad} % For gradients
\usepackage{pst-plot} % For axes
\usepackage[space]{grffile} % For spaces in paths
\usepackage{etoolbox} % For spaces in paths
\makeatletter % For spaces in paths
\patchcmd\Gread@eps{\@inputcheck#1 }{\@inputcheck"#1"\relax}{}{}
\makeatother

% Make SS at the beginning of a section

\makeatletter
%% See pp. 26f. of 'The LaTeX Companion,' 2nd. ed.
\def\@seccntformat#1{\@ifundefined{#1@cntformat}%
    {\csname the#1\endcsname\quad}%      default
    {\csname #1@cntformat\endcsname}}%   individual control
\newcommand{\section@cntformat}{\S\thesection\quad}
\newcommand{\subsection@cntformat}{\S\thesubsection\quad}
\makeatother % changes @ back to a special character

\usepackage{titlesec}

\CTEXsetup[format={\raggedright\large\bfseries}]{section}
\titleformat{\subsection}[runin]{\normalfont\bfseries}{\thesubsection.}{0.5em}{}[.]
\titleformat{\subsubsection}[runin]{\normalfont\bfseries}{\alph{subsubsection})}{0.5em}{}





\theoremstyle{definition}
\newtheorem{example}{例子}[section]
\newtheorem{definition}{定义}[section]
\newtheorem{theorem}{定理}[section]
\newtheorem{proposition}[theorem]{命题}
\newtheorem{prop}[theorem]{性质}
\newtheorem{corollary}[theorem]{推论}

\newenvironment{remark}{%
  \par\medskip
  \noindent
  \textbf{注:}
}{%
  \par\medskip
}

\newenvironment{solution}{%
  \par\medskip
  \noindent
  \textbf{解答:}
}{%
  \par\medskip
}

\newenvironment{solution*}{%
  \par\medskip
  \noindent 
  \color{gray}\small\textbf{提示或解答:}
}{%
  \par\medskip
}

\newenvironment{definition*}{%
  \par\medskip
  \noindent
  \textbf{定义:}
}{%
  \par\medskip
}

\newenvironment{lemma}{%
  \par\medskip
  \noindent
  \textbf{引理:}
}{%
  \par\medskip
}

\newenvironment{proposition*}{%
  \par\medskip
  \noindent
  \textbf{性质: }
}{%
  \par\medskip
}


\newtcolorbox{asidebox}{
  colback=gray!10,
  colframe=gray!60,
  fonttitle=\bfseries,
  title={Aside},
  breakable=true
}

\newtcolorbox{webaside}{
  colback=cyan!10,
  colframe=cyan!60,
  fonttitle=\bfseries,
  title={Web Demonstrate Aside},
  breakable=true
}

\usepackage{enumitem}

\setlist{nosep}

\setstretch{1.2}
\geometry{
    textheight=9in,
    textwidth=5.5in,
    top=1in,
    headheight=12pt,
    headsep=25pt,
    footskip=30pt
}

\usepackage{environ}
\usepackage[tikz]{bclogo}
\usepackage{tikz}
\usetikzlibrary{calc}
\NewEnviron{takeaway}
  {\par\medskip\noindent
  \begin{tikzpicture}
    \node[inner sep=0pt] (box) {\parbox[t]{.99\textwidth}{%
      \begin{minipage}{.3\textwidth}
      \centering\tikz[scale=5]\node[scale=3,rotate=30]{\bclampe};
      \end{minipage}%
      \begin{minipage}{.65\textwidth}
      \textbf{Takeaway Message}\par\smallskip
      \BODY
      \end{minipage}\hfill}%
    };
    \draw[red!75!black,line width=3pt] 
      ( $ (box.north east) + (-5pt,3pt) $ ) -- ( $ (box.north east) + (0,3pt) $ ) -- ( $ (box.south east) + (0,-3pt) $ ) -- + (-5pt,0);
    \draw[red!75!black,line width=3pt] 
      ( $ (box.north west) + (5pt,3pt) $ ) -- ( $ (box.north west) + (0,3pt) $ ) -- ( $ (box.south west) + (0,-3pt) $ ) -- + (5pt,0);
  \end{tikzpicture}\par\medskip%
}

\usepackage{marginnote}
\renewcommand*{\marginfont}{\color{gray}\ttfamily\small}
\usepackage{setspace}
\newcounter{paranum}[section]
\newcommand{\Par}[1]{\vspace{10pt}\noindent\textbf{\refstepcounter{paranum}\theparanum. }\textbf{#1}~~}
\newcommand{\lec}[1]{\reversemarginpar\marginnote{{\textbf{#1}}}}
\newcommand{\mn}[1]{\marginnote{{#1}}}
\renewcommand{\algorithmcfname}{算法}
\usepackage[abspath]{currfile}

\newcommand{\incfig}[1]{\begin{center}\includegraphics[width=.4\textwidth]{figs/#1}\end{center}}
\newcommand{\incfigw}[1]{\begin{center}\includegraphics[width=.8\textwidth]{figs/#1}\end{center}}
\newcommand{\set}[1]{\{#1\}}
\newcommand{\stirling}[2]{\left\{{#1 \atop #2}\right\}}
\newcommand{\binomt}[2]{\left(\left({#1 \atop #2}\right)\right)}
\newcommand{\pf}[4]{#1_{#2}^{#3_{#4}}}
\newcommand{\pl}[4]{#1_{#2}{#3^{#4}}}
\newcommand{\ty}[3]{{#1} \equiv {#2} ~(\bmod {#3})}
\newcommand{\Z}{{\mathbb Z}}
\newcommand{\one}{\mathbf{1}}
\newcommand{\varsub}[2]{\stackrel{#1}{\stackrel{\rule{#2}{0.4pt}}{\rule{#2}{0.4pt}}}}
\newcommand{\dd}{\mathrm{d}}
\newcommand{\Ep}[1]{\mathbb E\left(#1\right)}

\renewcommand{\red}[1]{{{\color{red}#1}}}
\newcommand{\teal}[1]{{{\color{teal}#1}}}
\renewcommand{\blue}[1]{{{\color{blue}#1}}}
\newcommand{\purple}[1]{{{\color{purple}#1}}}
\DeclareMathOperator{\var}{Var}
\newcommand{\E}{\mathbb E}
\newcommand{\like}{$\blacktriangleright$}
\newcommand{\exrate}[1]{{[#1]~}}
\newcommand{\newword}[2]{{\textbf{#1(#2)}\index{#1}}}
\newcommand{\newenword}[1]{{\textbf{#1}\index{#1}}}


\begin{document}

\section{上取整和下取整} 

实际上, 上取整和下取整实际上英语是对应着房子的``天花板''和``地板''. 

\incfig{fl-illus}

\subsection{基本定义} 

\begin{definition}[上取整和下取整]
    $\forall x \in \mathbb{R}$, 定义下取整函数和上取整函数
    \begin{align*}
    \lfloor x\rfloor:= \text{最大的} \leqslant x \text{的整数}\\
    \lceil x\rceil := \text{最小的} \geqslant x\text{ 的整数}
    \end{align*}
    分别称他们为``上取整函数''和``下取整函数 ''.
    
\end{definition}

\begin{example}绘制出下取整函数的图像
    \incfig{floor}
   上取整函数的图像也是同理.  
\end{example}

\begin{prop}
    取整函数有如下的三个性质: 

    \begin{enumerate}
        \item $\lfloor x\rfloor=x \iff$ $x$是正数 $\iff $ $\lceil x\rceil=x$.
        \item (放缩) $x-1<\lfloor x\rfloor \leqslant x \leqslant\lceil x\rceil<x+1$.
        \item (对称)$\lfloor-x\rfloor=-\lceil x\rceil, \quad F x\rceil=-\lfloor x\rfloor$.
    \end{enumerate}
    
\end{prop}

\begin{prop}
    实际上, 上下取整函数是对一类不等式的缩写

    $$
\lfloor x\rfloor=n\left\{\begin{array}{lll}
    \iff & x-n \leqslant x & \text {(固定}x\text{,考察 } n \text { 的范围) } \\
\iff & n \leqslant x \leqslant n+1 & \text { (固定}n\text{, 考察 } x \text { 的范围 })
\end{array}\right.
$$

$$
\lceil x\rceil=n \begin{cases}\iff & x \leqslant n \leqslant x+1 \\ \iff & n-1<x \leqslant n .\end{cases}
$$

    
\end{prop}

\begin{corollary}
    若$n \in \mathbb{Z},\lfloor x+n\rfloor=\lfloor x\rfloor+n$.
    
\end{corollary}

\begin{remark}
    实际上, 这一个性质经常用于将整数(也许是配凑出来的)移入或者移出取整记号中. 
\end{remark}

为了更加彰显不等式的地位, 特别有下面的一个不等式.

\begin{prop}[上下取整的不等式]

$$ 
\begin{gathered}
x<n \Leftrightarrow\lfloor x\rfloor<n ; \quad n<x \Leftrightarrow n<\lceil x\rceil . \\
x \leqslant n \Leftrightarrow\lceil x\rceil \leqslant n ; \quad n \leqslant x \Leftrightarrow n \leqslant\lfloor x\rfloor .
\end{gathered}
$$ 

\end{prop}

这是一个重要的式子. 可以通过分类讨论的方法证明之. 

\begin{definition}[分数部分] 定义一个数的分数部分为$x-\lfloor x\rfloor$. 如果与集合的记号不相冲突的话, 可以记作$\set{x}$. 
    
\end{definition}

\begin{example}
    $\lfloor x+y\rfloor$是否永远等于$\lfloor x\rfloor+\lfloor y\rfloor$?
    实际上不是这样的. 将$x,y$写作$x=\lfloor x\rfloor+\{x\}, \quad y=\lfloor y\rfloor+\{y\}$, 那么
\[
    \lfloor x+y\rfloor=\lfloor x\rfloor+\lfloor y\rfloor+\lfloor\{x\}+\{y\}\rfloor
\]
    我们发现当且仅当$\{x\}+\{y\}<1$ 时, $\lfloor x+y\rfloor=\lfloor x\rfloor+\lfloor y\rfloor$. 否则由于$0 \leqslant\{x\}+\{y\}<2$, 其为$\lfloor x\rfloor+\lfloor y\rfloor+1$. 
\end{example}

\subsection{上取整, 下取整的复合} 我们首先考察一些基本例子得到的结果. 

\begin{example}
    $\lceil\lfloor x\rfloor\rceil=\lfloor x\rfloor ;\lfloor\lceil x\rceil\rfloor=\lceil x\rceil$. 直接将两者复合得到的结果是平凡的. 
    
\end{example}

\begin{example}
    证明或推翻: $\lfloor\sqrt{\lfloor x\rfloor}\rfloor=\lfloor\sqrt{x}\rfloor$.

首先尝试举反例, 但是$\pi, e, \phi, 1,2 \cdots$都是正确的, 于是考虑证明. 

我们的目标为想办法除去$\sqrt{ }$下的$\lfloor\rfloor$. 假设$m:=\lfloor\sqrt{\lfloor x\rfloor}\rfloor$, 解掉最外层的底可以得到
$$
m \leqslant \sqrt{\lfloor x\rfloor}<m+1
$$

$$
\begin{aligned}
& \Rightarrow m^2 \leqslant\lfloor x\rfloor<(m+1)^2 \\
& \Rightarrow m^2 \leqslant x<(m+1)^2 . \Longrightarrow m \leqslant \sqrt{x}<(m+1)^2 \\
& \Rightarrow m \leqslant\lfloor\sqrt{x}\rfloor<(m+1)^2 .
\end{aligned}
$$
    
\end{example}

考虑泛化上述的例子. 

\begin{theorem}
    令$f(x)$为一个连续, 单调递增的函数, 满足
    \[
        f(x) \text{是整数} \implies x\text{是整数}. 
    \]

    那么有
    $$
\lfloor f(x)\rfloor=\lfloor f(\lfloor x\rfloor)\rfloor,\lceil f(x)\rceil=\lceil f(\lceil x \rceil)\rceil .
$$
只要$f(x), f(\lfloor x\rfloor), f(\lceil x\rceil)$有定义. 
\end{theorem}

\begin{proof}
    $1^\circ$ 如果$x$是整数, 那么显然成立. 

    $2^\circ$ 若$x>\lfloor x\rfloor$, 由于$f$单调递增, 那么$f(x)>f(\lfloor x\rfloor)$. 然后两端同时取下取整符号, 由于下取整记号不减, 那么$\lfloor f(x)\rfloor \geqslant\lfloor f(\lfloor x\rfloor)\rfloor$. 我们接下来分类讨论, 以确定大于号的情形不成立. 
    \begin{itemize}
        \item 如果$\lfloor f(x)\rfloor =\lfloor f(\lfloor x\rfloor)\rfloor$, 那么这就是我们想要的. 
        \item 如果$\lfloor f(x)\rfloor>\lfloor f(\lfloor x\rfloor)\rfloor$, 那么由于$f$是一个连续的函数, 一定存在$y$, 使得$\lfloor x\rfloor \leqslant y<x$, 并且$f(y)=\lfloor f(\lfloor x\rfloor)\rfloor$. 根据$f$的性质, $y$一定也是整数. 但是$\lfloor x\rfloor$与$x$之间不存在另一个整数, 矛盾! 所以我们的假设不成立.  
    \end{itemize}
\end{proof}

\begin{corollary}
    $$
\left\lfloor\frac{x+m}{n}\right\rfloor=\left\lfloor\frac{\lfloor x\rfloor+m}{n}\right\rfloor,\left\lceil\frac{x+m}{n}\right\rceil=\left\lceil\frac{\lceil x\rceil+m}{n}\right\rceil .
$$
    
\end{corollary}

比如, $$
\lfloor\lfloor\lfloor x / 10\rfloor / 10\rfloor / 10\rfloor=\lfloor x / 1000\rfloor .
$$

\subsection{区间计数问题} 

如果我们用如下的简写记后面的区间:
$$
\begin{cases}
    [\alpha..\beta] & \alpha \leqslant x \leqslant \beta\\
    (\alpha..\beta]& \alpha<x \leqslant \beta \\
    [\alpha..\beta) & \alpha \leq x<\beta\\
    (\alpha..\beta)& \alpha<x<\beta
\end{cases}$$

我们的问题是这个集合里面包含了多少个整数.($\alpha,\beta$不一定是整数). 

首先, 简化考虑的问题情形. 如果$\alpha,\beta$均为整数, 那么数量就是
\[
    \beta-\alpha+\text{``[''的个数}-1
\]

然后, 考虑转换, 因为
$$
\begin{aligned}
& \alpha \leqslant n<\beta \quad \Leftrightarrow\lceil\alpha\rceil \leqslant n<\lceil\beta\rceil \\
& \alpha<n \leqslant \beta \quad \Leftrightarrow\lfloor\alpha\rfloor<n \leqslant[\beta\rfloor
\end{aligned}
$$
所以余下二者仅需要补充上``$+1$''或者$``-1''$. 也就是
$$
\begin{array}{ll}
{[\alpha .. \beta]} & \lfloor\beta\rfloor-\lceil \alpha\rceil+1 \\
(\alpha . . \beta) & \lceil\beta \rceil-\lfloor\alpha\rfloor-1 .
\end{array}
$$
\begin{remark}
    \begin{enumerate}
        \item 我们习惯使用左闭右开的括号序列, 因为其具有可加性.
        \item 上述记号在关于求和取整时非常有用, 因为可以迅速的把这些技术的内容收缩下来. 
    \end{enumerate}
    
\end{remark}

\begin{example} 抽奖游戏. 在$[1,1000]$中选一整数. 记抽出数为$n$. 若$\lfloor\sqrt[3]{n}\rfloor \backslash n$(记号$a \backslash b$表示$a$是$b$的正因子), 那么他就赢取了5元. 否则, 他就输了1元. 请求出它玩这个游戏得到的期望. 


        \begin{align*}
            W&= \sum_{n=1}^{1000}[\text{抽到$n$会让我们赢}]\\
             &= \sum_{1 \leqslant n \leqslant 1000}[\lfloor\sqrt[3]{n}\rfloor \backslash n] \varsub{k:=\sqrt[3]{n}}{1.0cm} \sum_{k, n}[k=\lfloor\sqrt[3]{n}\rfloor][k \backslash n][1 \leqslant n \leqslant 1000] \\
             &\varsub{k \backslash n \text { 变形为 }n=km}{2.3cm}\sum_{k, m, n}\left[(k)^3 \leqslant n<(k+1)^3\right][n=k m][1 \leqslant n \leqslant 1000]\\
             &\varsub{\text{换掉}n:=km}{1.5cm} \sum_{k, m}\left[k^3 \leqslant k m<(k+1)^3\right][1 \leqslant k m \leqslant 1000]\\
             &\varsub{m=n/k=n/\lfloor\sqrt[3]n\rfloor}{2.5cm} 1+\sum_{k, m}\left[k^3 \leqslant k m<(k+1)^3\right][1 \leqslant k<10]\\
             &= 1+\sum_{1\leq k\leq 10}^{}(\lceil k^2+3 k+3 + \frac1k\rceil -\left\lceil k^2\right\rceil)\\
            &= 1+\sum_{1 \leqslant k<10}(3 k+4)=172
        \end{align*}
        
        拓展问题: 在$[1..n]$中呢?  记$K$为最大的满足$K^3\leq n$的数字. 也就是我们先求出较为整体的部分. 

        \begin{align*}
            W&=\sum_{1 \leqslant k<K}(3 k+4)+\sum_m\left[K^3 \leqslant K_m<N\right]
             &= =\frac{3}{2} K^2+\frac{5}{2} K-4+\underbrace{\sum_m\left[m \in\left[K^2, N / K\right]\right]}_{\left\lfloor\frac{N}{K}\right\rfloor-K^2+1}
        \end{align*}

        然后可以着手求零碎的部分. 较为零碎的部分并没有很好的封闭表达式, 因此我们只能够讨论它的渐进特性. 这个特性会在后续展开. 
        
\end{example}

\begin{example}
    对于$\alpha \in \mathbb{R}$, $\alpha$的谱是一个多重集, 定义做
    $$
\operatorname{spec}(\alpha):=\{\lfloor\alpha\rfloor,\lfloor 2 \alpha\rfloor, \cdots\}
$$

我们证明两个事实: 

第一, $\alpha \neq \beta \Rightarrow \operatorname{Spec}(\alpha) \neq \operatorname{Spec}(\beta)$. 

证明说: 不妨设$\alpha<\beta$. 我们发现$\exists m$, s.t. $m(\beta-\alpha) \geqslant 1$.(可以取 $\left\lceil\frac{1}{\beta-\alpha}\right\rceil$ ). 由此$m \beta-m \alpha \geqslant 1, \quad m \beta \geqslant 1+m \alpha \Rightarrow\lfloor m \beta\rfloor>\lfloor m \alpha\rfloor$. 因而$\text{Spec}(\beta)$中$\leqslant\lfloor m \alpha\rfloor$
的元素小于$m$个. 而$\operatorname{Spec}(\alpha)$中至少有$m$个$\leqslant\lfloor m \alpha\rfloor$个元素. 这就表示这两个不同的谱序列永远不可能相等.  

第二, $\operatorname{Spec}(\sqrt{2}) \cup \operatorname{Spec}(2+\sqrt{2})={\mathbb{N}}, \quad \operatorname{Spec}(\sqrt{2}) \cap \operatorname{Spec}(2+\sqrt{2})=\varnothing$. 

证明说: 考虑有多少个$\leq n$个元素在$\operatorname{spec}(\sqrt{2})$里面, 有多少个$\leqslant n$的元素在$\operatorname{Spec}(2+\sqrt{2})$里面. 定义$N(\alpha, n)$ 表示 $\operatorname{Spec}(\alpha)$ 中 $\leqslant n$ 的之元素个数.
$$
\begin{aligned}
N(\alpha, n) & =\sum_{k>0}[\lfloor k \alpha\rfloor \leqslant n] \\
& =\sum_{k>0}[\lfloor k \alpha\rfloor<n+1] \\
& =\sum_{k>0}[k \alpha<n+1]=\sum_k\left[0<k<\frac{n+1}{\alpha}\right] \\
& =\left\lceil\frac{n+1}{\alpha}\right\rceil-1 .
\end{aligned}
$$

然后求 $N(\sqrt{2}, n)+N(2+\sqrt{2}, n) \stackrel{?}{=} n$

$$
\begin{aligned}
& \left\lceil\frac{n+1}{\sqrt{2}}\right\rceil-1+\left\lceil\frac{n+1}{2+\sqrt{2}}\right\rceil-1=n \\
& \Leftrightarrow\left\lfloor\frac{n+1}{\sqrt{2}}\right\rfloor+\left\lfloor\frac{n+1}{2+\sqrt{2}}\right\rfloor=n \quad&(\lfloor x\rfloor-\lfloor x\rfloor=[x \text { is not int }]) \\
& \Leftrightarrow \underbrace{\frac{n+1}{\sqrt{2}}+\frac{n+1}{2+\sqrt{2}}}_{n+1}-(\underbrace{\left\{\frac{n+1}{\sqrt{2}}\right\}+\left\{\frac{n+1}{2+\sqrt{2}}\right\}}_{1\text{, 因为两个不是整数的加起来等于整数一定进位了}})=n \\
&
\end{aligned}
$$

这样的结论可推广到 $\frac{1}{\alpha}+\frac{1}{\beta}=1 \Leftrightarrow \operatorname{Spec}( \alpha)  \cup \operatorname{Spec} (\beta)= \mathbb{N}$.

\end{example}


\section{二元运算: 模(mod)运算}

\subsection{模} 我们已经知道了$n/m$的商为$\lfloor n / m\rfloor$, 我们下面定义$n/m$的余数. 

\begin{definition}
$n \bmod m$ 为 $n / m$ 的余数. 即     
$$
n \bmod m:=n-m\left\lfloor\frac{n}{m}\right\rfloor, \forall n, m \in \mathbb{R} ., m \neq 0 .
$$
特别地, 定义$n \bmod 0=n($ 方使起见 $)$. 

\end{definition}

\begin{example}
    $$
\begin{aligned}
& 5 \bmod 3=2 \\
& 5 \bmod (-3)=5-(-3)\left\lfloor\frac{5}{-3}\right\rfloor=5-3 \times 2=-1 \\
& -5 \bmod 3=-5-3\left\lfloor\frac{-5}{3}\right\rfloor=1 \\
& -5 \bmod (-3)=-5-(-3) \times\left\lfloor\frac{-5}{-3}\right\rfloor=-2 . \\
&
\end{aligned}
$$

可见, 对 $x \bmod y$ 而言,若 $y>0 , 0 \leqslant x \bmod y<y$; 

$y<0, \quad 0 \geqslant x \bmod y>y .$. 
\end{example}

\begin{example}
    $$
x=\lfloor x\rfloor+\{x\}=\lfloor x\rfloor+x \bmod 1 .
$$
    
\end{example}

同样可以使用上取整定义``不足近似量''.

\begin{definition}
    $$
n \text { mumble }:=m\left\lfloor\frac{n}{m}\right\rceil-n ., \forall n, m \in \mathbb{R}, m \neq 0 \text {. }
$$
    
\end{definition}


\begin{example}
    $$
\begin{aligned}
5 \text { mumble } 3=3\left[\frac{5}{3}\right\rceil-5 & =1 . \\
5 \text { mumble }(-3)=-3\left\lceil\frac{5}{-3}\right\rceil-5 & =-2 \\
-5 \text { mumble } 3 & =2 \\
-5 \text { mumble }-3 \quad & =-1 .
\end{aligned}
$$
    
\end{example}

我们首先注意一个性质: 

\begin{prop}
    $$
c(x \bmod y)=c x \bmod c y
$$
    
\end{prop}

\begin{proof}
    首先验证一般情况: 
    $$
\begin{aligned}
c(x \bmod y)=c\left(x-y\left\lfloor\frac{x}{y}\right\rfloor\right) & =c x-c y\left\lfloor\frac{x}{y}\right\rfloor \\
& =c x-c y\left\lfloor\frac{c x}{c y}\right\rfloor \\
& =c x \bmod c y .
\end{aligned}
$$

然后对于0的情况, 有
$$
c(x \bmod 0)=c x \bmod 0=c x
$$
也成立, 所以上面的性质成立. 
\end{proof}

\begin{example}
    我们希望尽可能平均地将$n$个物品分为$m$份. 如$37=8+8+8+8+5=8+8+7+7+7$. 

    \incfigw{distr}

    可以考虑横着放
    \begin{itemize}
        \item 若 $n \not\backslash m$ ,“长的行”将会容纳 $\left\lceil\frac{n}{m}\right\rceil$ 个物品. $(n \bmod m)$
        \item 若 $n \not\backslash m$ ,“短的行”将会容纳 $\left\lfloor\frac{n}{m}\right\rfloor$ 个物品. $(n \operatorname{mumble} m)$
    \end{itemize}
    并且两种行最多仅仅相差1. 

    我们考虑另一种方法. 要把$n$个物品“尽可能平均”分为$m$个组时(干脆使用$f(n,m)$表示之),  我们有如下的情况: 
    \begin{enumerate}
        \item 把 $\lceil n / m\rceil$ 个物品放到当前组中
        \item 递归执行$f(n-\lceil n / m\rceil, m-1)$. 
    \end{enumerate}

    比如$m=314, n=6$, 那么有下表格的过程: 

    \begin{center}
        \begin{tabular}{|ccc|}
\hline$n$ & $m$ & $\lceil n / m\rceil$ \\
314 & 6 & 53 \\
261 & 5 & 53 \\
208 & 4 & 52 \\
156 & 3 & 52 \\
104 & 2 & 52 \\
52 & 1 & 52 \\
\hline
\end{tabular}
    \end{center}


    我们来说明这个算法的正确性. 假设$n=\underbrace{q}_{\lfloor n/m \rfloor} m+\underbrace{r}_{n\bmod m}$. 那么
    \begin{enumerate}
        \item 若 $r=0$, 放 $\lceil n / m\rceil=q, f(n-q, n-1)$仍为 $r=0$ 的情形. 
        \item 考 $r>0, f(n-(q+1), n-1)$, 使下一次调用 $r$ 减少 1 .
    \end{enumerate}

    这样一来, 最后一定会到达$r=0$的情形. 

    接下来我们考虑第$k$组有多少物品. 答案是$\left\{\begin{array}{l}\lceil n / m\rceil, 1<k \leqslant n \bmod m \\ \lfloor n / m\rfloor, n \geqslant k > n \bmod m\end{array}\right.$

    稍加改写, 有
    $$
\begin{aligned}
&=q+[k \leqslant r]=q+\left\lceil\frac{r-k+1}{m}\right\rceil \\
&=\left\lceil\frac{r-k+1}{m}\right\rceil+q=\left\lceil\frac{\left.m q+r-k+1\right.}{m}\right\rceil \\
&=\left\lceil\frac{n-k+1}{m}\right\rceil .
\end{aligned}
$$

\end{example}

根据上述的例子, 我们就得到了一个恒等式: 

$$
n=\left\lceil\frac{n}{m}\right\rceil+\left\lceil\frac{n-1}{m}\right\rceil+\cdots+\left\lceil\frac{n-m+1}{m}\right\rceil .
$$

同时我们可以证明下面的命题: 

\begin{prop}

    $$
\begin{aligned}
& n=\underbrace{\left\lfloor\frac{n}{m}\right\rfloor+\left\lfloor\frac{n+1}{m}\right\rfloor+\cdots+\left\lfloor\frac{n+m}{m}\right\rfloor}_{m \text { 项 }} \\
& n=\overbrace{\left\lceil\frac{n}{m}\right\rceil+\left\lceil\frac{n-1}{m}\right\rceil+\cdots+\left\lceil\frac{n-m+1}{m}\right\rceil} \\
&
\end{aligned}
$$
    
\end{prop}

要证明这两个对称的形式, 只要注意到$n=\lceil n / 2\rceil+\lfloor n / 2\rfloor$即可.

将$n$代换为$mx$可以得到如下推论: 

\begin{corollary}
    $$
\lfloor m x\rfloor=\lfloor x\rfloor+\left\lfloor x+\frac{1}{m}\right\rfloor+\cdots+\left\lfloor x+\frac{m-1}{m}\right\rfloor
$$
    
\end{corollary}

\begin{proof}
    \begin{align*}
        \lfloor m x\rfloor&=\left\lfloor\frac{\lfloor m x\rfloor}{m}\right\rfloor+\left\lfloor\frac{\lfloor m x\rfloor+1}{m}\right\rfloor+\cdots+\left\lfloor\frac{\lfloor m x\rfloor+m-1}{m}\right\rfloor \\
                          &\varsub{\text{去掉下取整}}{1.5cm} \lfloor x\rfloor+\left\lfloor x+\frac{1}{m}\right\rfloor+\cdots+\left\lfloor x+\frac{m-1}{m}\right\rfloor
    \end{align*}
    
\end{proof}

\end{document}
