%!TEX TS-program = xelatex
\documentclass{ctexart}

\usepackage{amsmath, amsthm, amssymb, amsfonts}
\usepackage{thmtools}
\usepackage{graphicx}
\usepackage{setspace}
\usepackage{geometry}

\usepackage{float}
\usepackage{amsthm}
\usepackage{hyperref}
\usepackage{cleveref}
% \usepackage{mathabx}
\usepackage[utf8]{inputenc}
\usepackage[english]{babel}
\usepackage{framed}
\usepackage[dvipsnames]{xcolor}
\usepackage[skins,breakable]{tcolorbox}
\usepackage{awesomebox}
\usepackage{mathrsfs}  
\usepackage{xcolor}
\usepackage{wrapfig}
\usepackage{algorithm2e}
\RestyleAlgo{ruled}

\usepackage{pstricks-add}
\usepackage{epsfig}
\usepackage{pst-grad} % For gradients
\usepackage{pst-plot} % For axes
\usepackage[space]{grffile} % For spaces in paths
\usepackage{etoolbox} % For spaces in paths
\makeatletter % For spaces in paths
\patchcmd\Gread@eps{\@inputcheck#1 }{\@inputcheck"#1"\relax}{}{}
\makeatother

% Make SS at the beginning of a section

\makeatletter
%% See pp. 26f. of 'The LaTeX Companion,' 2nd. ed.
\def\@seccntformat#1{\@ifundefined{#1@cntformat}%
    {\csname the#1\endcsname\quad}%      default
    {\csname #1@cntformat\endcsname}}%   individual control
\newcommand{\section@cntformat}{\S\thesection\quad}
\newcommand{\subsection@cntformat}{\S\thesubsection\quad}
\makeatother % changes @ back to a special character

\usepackage{titlesec}

\CTEXsetup[format={\raggedright\large\bfseries}]{section}
\titleformat{\subsection}[runin]{\normalfont\bfseries}{\thesubsection.}{0.5em}{}[.]
\titleformat{\subsubsection}[runin]{\normalfont\bfseries}{\alph{subsubsection})}{0.5em}{}





\theoremstyle{definition}
\newtheorem{example}{例子}[section]
\newtheorem{definition}{定义}[section]
\newtheorem{theorem}{定理}[section]
\newtheorem{proposition}[theorem]{命题}
\newtheorem{prop}[theorem]{性质}
\newtheorem{corollary}[theorem]{推论}

\newenvironment{remark}{%
  \par\medskip
  \noindent
  \textbf{注:}
}{%
  \par\medskip
}

\newenvironment{solution}{%
  \par\medskip
  \noindent
  \textbf{解答:}
}{%
  \par\medskip
}

\newenvironment{solution*}{%
  \par\medskip
  \noindent 
  \color{gray}\small\textbf{提示或解答:}
}{%
  \par\medskip
}

\newenvironment{definition*}{%
  \par\medskip
  \noindent
  \textbf{定义:}
}{%
  \par\medskip
}

\newenvironment{lemma}{%
  \par\medskip
  \noindent
  \textbf{引理:}
}{%
  \par\medskip
}

\newenvironment{proposition*}{%
  \par\medskip
  \noindent
  \textbf{性质: }
}{%
  \par\medskip
}


\newtcolorbox{asidebox}{
  colback=gray!10,
  colframe=gray!60,
  fonttitle=\bfseries,
  title={Aside},
  breakable=true
}

\newtcolorbox{webaside}{
  colback=cyan!10,
  colframe=cyan!60,
  fonttitle=\bfseries,
  title={Web Demonstrate Aside},
  breakable=true
}

\usepackage{enumitem}

\setlist{nosep}

\setstretch{1.2}
\geometry{
    textheight=9in,
    textwidth=5.5in,
    top=1in,
    headheight=12pt,
    headsep=25pt,
    footskip=30pt
}

\usepackage{environ}
\usepackage[tikz]{bclogo}
\usepackage{tikz}
\usetikzlibrary{calc}
\NewEnviron{takeaway}
  {\par\medskip\noindent
  \begin{tikzpicture}
    \node[inner sep=0pt] (box) {\parbox[t]{.99\textwidth}{%
      \begin{minipage}{.3\textwidth}
      \centering\tikz[scale=5]\node[scale=3,rotate=30]{\bclampe};
      \end{minipage}%
      \begin{minipage}{.65\textwidth}
      \textbf{Takeaway Message}\par\smallskip
      \BODY
      \end{minipage}\hfill}%
    };
    \draw[red!75!black,line width=3pt] 
      ( $ (box.north east) + (-5pt,3pt) $ ) -- ( $ (box.north east) + (0,3pt) $ ) -- ( $ (box.south east) + (0,-3pt) $ ) -- + (-5pt,0);
    \draw[red!75!black,line width=3pt] 
      ( $ (box.north west) + (5pt,3pt) $ ) -- ( $ (box.north west) + (0,3pt) $ ) -- ( $ (box.south west) + (0,-3pt) $ ) -- + (5pt,0);
  \end{tikzpicture}\par\medskip%
}

\usepackage{marginnote}
\renewcommand*{\marginfont}{\color{gray}\ttfamily\small}
\usepackage{setspace}
\newcounter{paranum}[section]
\newcommand{\Par}[1]{\vspace{10pt}\noindent\textbf{\refstepcounter{paranum}\theparanum. }\textbf{#1}~~}
\newcommand{\lec}[1]{\reversemarginpar\marginnote{{\textbf{#1}}}}
\newcommand{\mn}[1]{\marginnote{{#1}}}
\renewcommand{\algorithmcfname}{算法}
\usepackage[abspath]{currfile}

\newcommand{\incfig}[1]{\begin{center}\includegraphics[width=.4\textwidth]{figs/#1}\end{center}}
\newcommand{\incfigw}[1]{\begin{center}\includegraphics[width=.8\textwidth]{figs/#1}\end{center}}
\newcommand{\set}[1]{\{#1\}}
\newcommand{\stirling}[2]{\left\{{#1 \atop #2}\right\}}
\newcommand{\binomt}[2]{\left(\left({#1 \atop #2}\right)\right)}
\newcommand{\pf}[4]{#1_{#2}^{#3_{#4}}}
\newcommand{\pl}[4]{#1_{#2}{#3^{#4}}}
\newcommand{\ty}[3]{{#1} \equiv {#2} ~(\bmod {#3})}
\newcommand{\Z}{{\mathbb Z}}
\newcommand{\one}{\mathbf{1}}
\newcommand{\varsub}[2]{\stackrel{#1}{\stackrel{\rule{#2}{0.4pt}}{\rule{#2}{0.4pt}}}}
\newcommand{\dd}{\mathrm{d}}
\newcommand{\Ep}[1]{\mathbb E\left(#1\right)}

\renewcommand{\red}[1]{{{\color{red}#1}}}
\newcommand{\teal}[1]{{{\color{teal}#1}}}
\renewcommand{\blue}[1]{{{\color{blue}#1}}}
\newcommand{\purple}[1]{{{\color{purple}#1}}}
\DeclareMathOperator{\var}{Var}
\newcommand{\E}{\mathbb E}
\newcommand{\like}{$\blacktriangleright$}
\newcommand{\exrate}[1]{{[#1]~}}
\newcommand{\newword}[2]{{\textbf{#1(#2)}\index{#1}}}
\newcommand{\newenword}[1]{{\textbf{#1}\index{#1}}}


\title{第二章~~和式基础}
\author{具体数学阅读笔记}
\begin{document}

\maketitle

\section{和式及其基本操作}

要使得求和便于书写和分析, 我们最好(跟随傅里叶)引入如下的记号:$\sum$.

\subsection{和式与求和记号}

\begin{definition}[求和记号$\sum$]
	对于一个可数的集合$S=\set{a_1, a_2, \cdots, a_n}$, 求和实际上是将这个集合的每一个元素在某一个函数$f:S\to X$作用之后, 把对应的值相加. 记作$\sum_{i \in S} f(i)$; 表示$f\left(a_1\right)+f\left(a_2\right)+f\left(a_3\right)+\cdots+f\left(a_n\right)+\cdots$.
\end{definition}

\begin{example}
	考虑
	$$
		\begin{aligned} & \sum_{\substack{1 \leq k \leq n}} a_k=a_1+a_2+\cdots+a_n . \\ & \sum_{\substack{1 \leq k \leq 100 \\ k \text { odd }}} k^2=\sum_{0 \leq k \leq 49}(2 k+1)^2 .\end{aligned}
	$$
\end{example}

\subsubsection{换元法} 如果将 $\sum_{i \in S} f(i)$ 换为 $\sum_{k \in S} f(k)$ , 求和的表示内容将不变. 这是因为仅仅是``当前 $S$ 中的代表''用来表示的字母不同, 从而肯定不会影响整个映射$f$所表达的意思.

\begin{example}

	考虑
	$$
		\sum_{1 \leq k \leq n} a_k \varsub{k\text{代换为}s+1}{1.5cm} \sum_{1 \leq s+1 \leq n} a_{s+1} \varsub{s\text{代换为}k}{1.5cm} \sum_{1 \leq k+1 \leq n} a_{k+1}
	$$
\end{example}

\begin{remark}
	有时候$\sum_{a \leq i \leq b} f(i)$也写作$\sum_{i=a}^b f(i)$. 但是看到这样一来, 变量代换之后就得到了
	$$
		\sum_{i=1}^n a_i\varsub{k\text{代为}k+1}{1.2cm}\sum_{i=0}^{n-1} a_{i+1}
	$$
	更易于出错.

	此外, 我们用$k:=k+1$来表示把$k$代为$k+1$. 最后, 数列可以看做特殊的函数. 实际上书写$a_k$实际上就相当于$f(k)$.
\end{remark}

\subsubsection{求和的性质}

尽管我们可以对可数无限个元素进行求和操作, 但是, 那样的数列我们在高等数学的级数部分就已经学过了. 这里先讨论有限项求和的情形.

\begin{theorem}\label{thm:sum-prop}
	设$K$是某一有限个正整数的集合, 我们有如下的三条规则:

	\begin{itemize}
		\item 常数项进出求和记号: $$
			      \sum_{k \in k} c f(k)=c \sum_{k \in k} f(k);
		      $$
		\item 求和记号的拆分: $$
			      \sum_{k \in K} f(k) g(k)=\sum_{k \in K} f(k)+\sum_{k \in K} g(k);
		      $$
		\item 若$p(k)$应用于$K$中的每一个元素之后组成的集合依然是$K$的一个排列, 那么$$
			      \sum_{k \in K} f(k)=\sum_{p(k) \in K} f(p(k)).
		      $$
	\end{itemize}
\end{theorem}

上述定理的证明可以直接由定义得到.

\begin{example}
	如$K=\{-1,0,1\}, p(k)=-k$, 由于 $p(-1)=1, p(0)=0$,$p(1)=-1 , p$ 对 $k$ 中每一个元素构成集合为 $\{-1,0,1\}=K$.
\end{example}

\begin{example}[等差数列求和]
	求 $$
		S=\sum_{0 \leq k \leq n}(a+b k)
	$$的值.

	考虑
	$$
		\begin{aligned}
			 & S_2=\sum_{0 \leq k \leq n}(a+b k)                                                           \\
			 & \varsub{k:=n-k}{0.9cm} \sum_{0 \leq n-k \leq n}(a+b(n-k))=\sum_{0 \leq k \leq n}(a+b n-b k)
		\end{aligned}
	$$

	令$S+S_2$ 得

	$$
		\begin{aligned}
			S+S_2=2 S & =\sum_{0 \leq k \leq n}(a+b k)+\sum_{0 \leq k \leq n}(a+b n-b k)                   \\
			          & =\sum_{0 \leq k \leq n}(2 a+b n)=(2 a+b n) \sum_{0 \leq k \leq n} 1=(2 a+b n)(n+1)
		\end{aligned}
	$$

	那 $$
		S=\frac{(n+1)(2 a+b n)}{2}.
	$$

\end{example}

\subsection{Iverson括号}

\begin{definition}[Iverson 括号]
	假设$P$是一个命题, 定义
	$$
		[P]=\left\{\begin{array}{lr}
			1, & \text{命题}P\text{为真} \\
			0, & \text{命题}P\text{为假}
		\end{array}\right.
	$$
\end{definition}

这样的记号便于优化很复杂的求和操作, 并且可以把求和符号的下标转换为命题之间的操作.

\begin{example}
	求和式 $\sum_{0 \leq k \leq n} k$ 可被改为 $\sum_k k[0 \leq k \leq n]$.

	如果 $k$ 未指定限定条件, 我们认为 $k \in \mathbb{Z}$. 也就是说上述式子

	$$
		\begin{aligned}
			\sum_k k[0 \leq k \leq n]= & \cdots+(-3 \cdot 0)+(-2 \cdot 0)+(-1 \cdot 0)+(0 \cdot 0)+(1 \cdot 1)+(2 \cdot 1)+\cdots  +(n \cdot 1) \\
			=                          & (0 \cdot 1)+(1 \cdot 1)+\cdots+(n \cdot 1)
		\end{aligned}
	$$

\end{example}

\begin{example}
	如果$K$与$K'$是两个整数集合, 那么$\forall k$,
	$$
		[k \in K]+\left[k \in K^{\prime}\right]=\left[k \in\left(K \cap K^{\prime}\right)\right]+\left[k \in\left(K \cup K^{\prime}\right)\right]
	$$

	由此可以导出对应的和式
	$$
		\sum_{k \in k} a_k+\sum_{k \in k^{\prime}} a_k=\sum_{k \in k \cap k^{\prime}} a_k+\sum_{k \in k U k^{\prime}} a_k
	$$
\end{example}

这是一个很有用的命题. 下面我们会看到它的用途.

\begin{proposition}
	$[k \in K]+\left[k \in K^{\prime}\right]=\left[k \in\left(K \cap K^{\prime}\right)\right]+\left[k \in\left(K \cup K^{\prime}\right)\right]$对 $k , k^{\prime}$ 为可数集, $\forall k$.
\end{proposition}

\subsection{常见的求和方法}

\subsubsection{成套方法} 这个方法类似于在解答微分方程的时候, 首先求解特解, 然后求解通解. 在这里我们不再赘述.

\subsubsection{扰动法} 要计算$S_n=\sum_{0 \leq k \leq n} a_k$, 可以有两种方法改写

\begin{proposition}
	扰动法是指

	\begin{align*}
		\boxed{S_{n}+a_{n+1}}=\sum_{0\leq k\leq n+1}a_{k} & =a_{0}+\sum_{1\leq k\leq n+1}a_{k}                        \\
		                                                  & \varsub{k:=k+1}{0.9cm}a_{0}+\sum_{1\leq k+1\leq n+1}a_{k} \\
		                                                  & \boxed{=a_{0}+\sum_{0\leq k\leq n}a_{k}}
	\end{align*}

\end{proposition}

\begin{example}
	用上述的方法计算等比数列. $S_n=\sum_{0 \leq k \leq n} a x^k$.
	$$
		\begin{aligned}
			S_n+a x^{n+1} & =a x^0+\sum_{0 \leq k \leq n} a x^{k+1} \\
			              & =a x^0+x \sum_{0 \leq k \leq n} a x^k   \\
			              & =a x^0+x S_n
		\end{aligned}
	$$

	对于$x\neq 1$, 有
	$$
		S_n=\sum_{k\neq 1}^n a x^k=\frac{a-a x^{n+1}}{1-x}
	$$
\end{example}

\begin{example}[等差数列乘等比数列]
	计算
	$$
		S_n=\sum_{0 \leq k \leq n} k \cdot 2^k.
	$$

	按照上面的方法,
	$$
		\begin{aligned}
			S_n+(n+1) 2^{n+1} & =0+\sum_{0 \leq k \leq n}(k+1) 2^{k+1}                                    \\
			                  & = 0+\sum_{0 \leq k \leq n} k \cdot 2^{k+1}+\sum_{0 \leq k \leq n} 2^{k+1} \\
			                  & =2 S_n+ 2^{n+2}-2
		\end{aligned}
	$$

	解出 $S_n$, 就是

	$$
		S_n=\sum_{0 \leq k \leq n} k 2^k=(n-1) 2^{n+1}+2 \text {. }
	$$
\end{example}

\begin{example}
	从上面的推导中, 我们知道$\sum_{k=0}^n x^k=\frac{1-x^{n-1}}{1-x}$ , 两边对 $x$ 求导, 就有
	$$
		\begin{aligned}
			\sum_{k=0}^n k x^{k-1} & =\frac{(1-x)\left(-(n+1) x^n\right)+1-x^{n+1}}{\left(1-x^2\right)} \\
			                       & =\frac{1-(n+1) x^n+n x^{n+1}}{(1-x)^2} .
		\end{aligned}
	$$

	同样也可以得到上一式子的结果.
\end{example}

\subsubsection{求和因子}

\begin{example}
	由递推关系
	$$
		\begin{aligned}
			 & T_0=0           \\
			 & T_n=2 T_{n-1}+1
		\end{aligned}
	$$
	倘若两端同时除以$2^n$, 就得到
	$$
		\begin{aligned}
			 & T_0 / 2^0=0                         \\
			 & T_n / 2^n=T_{n-1} / 2^{n-1}+1 / 2^n
		\end{aligned}
	$$
	令 $S=T_n / 2^n$ ,得
	$\left\{\begin{aligned} & S_0=0 \\ & S_n=S_{n-1}+2^{-n}\end{aligned}\right.$, 即 $S_n=\sum_{k=1}^n 2^{-k}$ , 为一等比数列.
\end{example}

对于更为一般的式子, 如 $a_n T_n=b_n T_{n-1}+c_n$, 可变为 $S_n=S_{n-1}+ S_n c_n$的形式.

$1^\circ$ 方法: 使两边同时乘以求和因子 $S_n$:

\[
	\underbrace{\boxed{s_{n}a_{n}T_{n}}}_{:=S_{n}}=\underbrace{S_{n}b_{n}T_{n-1}}_{\text{必须得是}S_{n-1}=S_{n-1}a_{n-1}T_{n-1}}+S_{n}c_{n}
\]
由此 $S_n=S_0 a_0 \cdot T_0+\sum_{k=1}^n S_k c_k=S_1 b_1 T_0+\sum_{k=1}^n S_k c_k$ , 那么$$
	T_n=\frac{1}{S_n a_n}\left(S_1 b_1 T_0+\sum_{k=1}^n S_k c_k\right)
$$.

$2^\circ$ 寻找$S_n$的方法: 由于上式要满足$S_n b_n T_{n-1}=S_{n-1} a_{n-1} T_{n-1}$, 代入展开, 有

$$
	\begin{aligned}
		s_n & =\frac{s_{n-1} a_{n-1}}{b_n}                                 \\
		    & =\frac{s_{n-2} a_{n-1} a_{n-2}}{b_n b_{n-1}}=\cdots          \\
		    & =\frac{a_{n-1} a_{n-2} \cdots a_1}{b_n b_{n-1} \cdots b_2} .
	\end{aligned}
$$

因此我们就这样找到了求和因子.

\begin{example}
	解由快速排序带来的递归式:

	$$
		\begin{aligned}
			 & C_0=0                                                 \\
			 & C_n=n+1+\frac{2}{n} \sum_{k=0}^{n-1} C_k \quad, n>0 .
		\end{aligned}
	$$

	将 $C_n$ 两侧同乘以 $n$, 得
	$$
		n C_n=n^2+n+2 \sum_{k=0}^{n-1} C_k \quad(n>0).
	$$

	令 $n:=n-1$, 有 $$
		(n-1) C_{n-1}=(n-1)^2+(n-1)+2 \sum_{k=0}^{n-2} c_k.
	$$

	上面两式相减, 有$n C_n-(n-1) C_{n-1}=2 n+2 C_{n-1}$, 即

	$$
		\begin{aligned}
			C_0   & =0                   \\
			n C_n & =(n+1) C_{n-1}+2 n .
		\end{aligned}
	$$

	将上述 $a_n=n, b_n=n+1, c_n=2 n$, 得 $s=\frac{(n-1) \cdots 1}{(n+1) \cdots 3}=\frac{2}{n(n+1)}$.

	得
	$$
		C_n=2(n+1) \sum_{k=1}^n \frac{1}{k+1}=2(n+1) H_{n+1}.
	$$

\end{example}

\newpage
\section{多重和式}

\subsection{引例与启发性讨论} 考虑

$$
	\begin{aligned}
		\sum_{1 \leqslant j, k \leqslant 3} a_j b_k= & a_1 b_1+a_1 b_2+a_1 b_3    \\
		                                             & +a_1 b_1+a_2 b_2+a_2 b_3   \\
		                                             & +a_3 b_1+a_3 b_2+a_3 b_3 .
	\end{aligned}
$$

其中$j,k$是两个指标.

如果$P(j,k)$是关于$j,k$的性质, 那么$\sum_{P(j, k)} a_{j, k}=\sum a_{j, k}[P(j, k)]$, 有时候也写作$\sum_j\left(\sum_k a_{j, k}[P(j, k)]\right)$, 可以简写做作$\sum_j \sum_k a_{j, k}[P(j, k)]$.

\begin{proposition}
	如果$k$的表示不依赖于$j$的话, 那么
	$$\sum_j \sum_k a_{j, k}[P(i, k)]=\sum_k \sum_j a_{j, k}[P(j, k)]$$
\end{proposition}

\begin{example}
	上述的考量最简单的情形如下:
	$$
		\begin{aligned}
			\sum_{1 \leqslant j \leqslant 2} \sum_{1 \leqslant k \leqslant 2} a_{j, k} & =a_{1,1}+a_{1,2}+a_{2,1}+a_{2,2}                                                                              \\
			                                                                           & =a_{2,1}+a_{2,1}+a_{1,2}+a_{2,2}=\sum_{1 \leqslant k \leqslant 2} \sum_{1 \leqslant j \leqslant 2} a_{j, k} .
		\end{aligned}
	$$
\end{example}

\begin{proposition}[一般分配率]
	对于任意的整数集合$J,K$有
	$$
		\sum_{\substack{j \in J \\ k \in K}} a_j b_k=\left(\sum_{j \in J} a_j\right)\left(\sum_{k \in K} b_k\right) .
	$$
\end{proposition}

\begin{remark}
	这里$J$和$K$集合要不依赖对方就能确定. 例如,
	$$
		\sum_{i=1}^n \sum_{j=i}^m f(i, j)
	$$
	就无法使用这个方式.
\end{remark}

\begin{proof} 我们可以考虑
	$$
		\begin{aligned}
			 & \left(\sum_{j \in J} a_j\right)\left(\sum_{k \in k} b_k\right)=\sum_j a_j[j \in J] \quad \sum_{k \in k} b_k[k \in K] \\
			 & =\sum_j a_j[j \in J]\left(\sum_k b_k[k \in k]\right)                                                                 \\
			 & \varsub{j \mp k \text { 独立 }}{1.5cm} \sum_j \sum_k a_j b_k[j \in J][k \in k]                                         \\
			 & =\sum_j \sum_k a_j b_k[j \in J ,k \in K] .                                                                           \\
			 &
		\end{aligned}
	$$
\end{proof}

\begin{example}
	例如,
	$$
		\sum_{1 \leqslant j, k \leqslant 3} a_j b_k=\left(\sum_{j=1}^3 a_j\right)\left(\sum_{k=1}^3 b_k\right)
	$$

	请再次注意这里的$k$的取值构成何种集合与$j$无关. 与这个例子相对, $\sum_{i=1}^n \sum_{j=i}^m f(i, j)$就不适用这个规则. 因为$i\leq j\leq m$表示什么集合与$i$的值有关.
\end{example}

下面我们来看相互依赖的集合的情形.

\begin{proposition}
	对于依赖于$j$取值的集合$K(j),j\in J$, 可以按照如下的方式进行交换求和记号,
	$$
		\sum_{j \in J} \sum_{k \in K(j)} a_{j, k}=\sum_{k \in K^{\prime}} \sum_{j \in J^{\prime}(k)} a_{j, k}
	$$
	满足$[j \in J][k \in K(j)]=\left[k \in K^{\prime}\right]\left[j \in J^{\prime}(k)\right]$.
\end{proposition}

\begin{remark}
	可以把这个与交换二重积分次序相联系起来. 例如对于区域$D:0\leq x\leq 1, 0\leq y\leq x$的区域积分, 可以写作

	$$
		\begin{aligned}
			\iint_D f(x, y) d x d y & =\int_0^1 d x \int_0^x f(x, y) d y   & \text { (从下向上) } \\
			                        & =\int_0^1 d y \int_x^1 f(x, y) d x . & \text { (从左向右) }
		\end{aligned}
	$$

	命题中表述的事实是保证每个元素都计算了且仅仅被计算了一次.
\end{remark}

\begin{corollary}
	类似上面举例的二重积分更换$x,y$, 有
	$$
		\begin{aligned}
			{[1 \leqslant j \leqslant n][1 \leqslant k \leqslant n] } & =[1 \leqslant j \leqslant k \leqslant n]                \\
			                                                          & =[1 \leqslant k \leqslant n][1 \leqslant j \leqslant k]
		\end{aligned}
	$$
\end{corollary}

\begin{example}
	根据上述推论,
	$$
		\sum_{j=1}^n \sum_{k=j}^n a_{j, k}=\sum_{1 \leqslant j \leqslant k \leqslant n} a_{j, k}=\sum_{k=1}^n \sum_{j=1}^k a_{j, k} .
	$$
\end{example}

\begin{example}
	给出矩阵
	$$
		\left[\begin{array}{cccc}
				a_1 a_1 & a_1 a_2 & \cdots & a_1 a_n \\
				a_2 a_1 &         &        &         \\
				\vdots  &         & \ddots &         \\
				a_n a_1 &         &        & a_n a_n
			\end{array}\right]
	$$
	求
	$$
		S_{\overline\backslash|}=\sum_{1 \leqslant j \leqslant k \leqslant n} a_j a_k
	$$

	\newcommand\ssj{\overline\backslash|}
	\newcommand\xsj{\underline/|}

	由于该矩阵是对称的, 即$a_i a_j=a_j a_i$, $S_{\overline\backslash|} \approx \frac{1}{2} S_{\square}$, 那么
	$$
		S_{\ssj}=\sum_{1 \leq j \leq k \leqslant n} a_j a_k=\sum_{1 \leqslant j \leqslant k \leqslant n} a_k a_j\varsub{\text{对换}j,k}{1.0cm}\sum_{l \leqslant k \leqslant j \leqslant n} a_j a_k=S_{\xsj}
	$$

	根据 $\left[k \in k^{\prime}\right]+[k \in k]=\left[k \in\left(k \cap k^{\prime}\right)\right]+\left[k \in\left(k \cup K^{\prime}\right)\right]$, 就有
	$$
		[1 \leqslant j \leqslant k \leqslant n]+[1 \leqslant k \leqslant j \leqslant n]=[1 \leqslant j, k \leqslant n]+[1 \leqslant j=k \leqslant n] .
	$$
	所以
	$$
		\begin{aligned}
			2 S_{\ssj}=S_{\ssj}+S_{\xsj} & =\sum_{1 \leqslant i, k \leqslant n} a_j a_k+\sum_{1 \leqslant j=k \leqslant n} a_j a_k \\
			                             & =\left(\sum_{k=1}^n a_k\right)^2+\sum_{k=1}^n a_k^2
		\end{aligned}
	$$
	那么
	$$
		S_{\ssj}=\frac{1}{2}\left(\left(\sum_{k=1}^n a_k\right)^2+\sum_{k=1}^n a_k^2\right).
	$$

\end{example}

\begin{example}
	求解
	$$
		S=\sum_{1 \leqslant j<k \leqslant n}\left(a_k-a_j\right)\left(b_k-b_j\right).
	$$
	注意到$i,j$的对称性, 将$i,j$对换得:
	$$
		S=\sum_{1 \leqslant k<j \leqslant n}\left(a_j-a_k\right)\left(b_j-b_k\right)=\sum_{1 \leqslant k<j \leqslant n}\left(a_k-a_j\right)\left(b_k-b_j\right)
	$$
	由于
	$$
		[1 \leqslant j<k \leqslant n]+[1 \leqslant k<j \leqslant n]=[1 \leqslant j, k \leqslant n]-[1 \leqslant j=k \leqslant n]
	$$
	因此
	$$
		\begin{aligned}
			2 S & =\sum_{1 \leqslant j, k \leqslant n}\left(a_j-a_k\right)\left(b_j-b_k\right)-\sum_{1 \leqslant j=k \leqslant n}\left(a_j-a_k\right)\left(b_j-b_k\right) .                    \\
			    & =\sum_{1 \leqslant j \leqslant k \leqslant n}\left(a_j-a_k\right)\left(b_j-b_k\right)-0 .                                                                                    \\
			    & = \boxed{\sum_{1 \leq j, k \leqslant n} a_j b_j}-\sum_{1 \leq j, k \leq n} a_j b_k-\sum_{1 \leq j, k \leqslant n} a_k b_j+\boxed{\sum_{1\leqslant j, k \leqslant n} a_k b_k} \\
			    & = \boxed{2 n \sum_{1 \leqslant k \leqslant n} a_k b_k}-2\left(\sum_{k=1}^n a_k\right)\left(\sum_{k=1}^n b_k\right) .
		\end{aligned}
	$$

	那么
	$$
		S=n \sum_{1 \leqslant k \leqslant n} a_k b_k-\left(\sum_{k=1}^n a_k\right)\left(\sum_{k=1}^n b_k\right) .
	$$

	我们便得到
	$$
		\left(\sum_{k=1}^n a_k\right)\left(\sum_{k=1}^n b_k\right)=n \sum_{k=1}^n a_k b_k-\sum_{1 \leqslant j<k \leqslant n}\left(a_k-a_j\right)\left(b_k-b_j\right)
	$$
	这是Chebyshev单调不等式的特例.

	Chebyshev单调不等式是说, 如果$a_1 \leqslant a_2 \leqslant \cdots \leqslant a_n, b_n \leqslant b_2 \leqslant \cdots \leqslant b_n$, 那么
	$$
		\left(\sum_{k=1}^n a_k\right)\left(\sum_{k=1}^n b_k\right) \leqslant n \sum_{i=1}^n a_k b_k
	$$
	反之亦然.
\end{example}

\subsection{与一重求和中第三条的联系} 回顾\cref{thm:sum-prop}中的第三条($\sum_{k \in K} a_k=\sum_{p(k) \in K} a_{p(k)}$, $p(k)$是一个原集合的排列). 如果现在将$k$换做$f(j)$, 其中$f$是一个$J\to K$的映射, 那么
$$
	\sum_{j \in J} a_{f(j)}=\sum_{k \in K} a_k \# f^{-}(k) .
$$
其中$f^{-}(k)=\{j \mid f(j)=k\}$, $\#S$是集合$S$的元素个数. 通俗地说, 就是对 $j \in J ,\quad f(j)=k$ 的数量.

\begin{proof}
	直接展开:
	$$
		\begin{aligned} \sum_{j \in J} a_{f(j)}=\sum_{\substack{j \in J \\ k \in K}} a_k[f(j)=k] & =\sum_{k \in K} a_k \sum_{j \in J}[f(j)=k] \\ & =\sum_{k \in k} a_k \# f^{-}(k) .\end{aligned}
	$$
\end{proof}

\begin{example}
	若$f$是一个一对一的函数, 那么$\# f^{-}(k)=1$, 也就是
	$$
		\sum_{j \in J} a_{f(j)}=\sum_{f(j) \in k} a_{f(j)} \cdot 1=\sum_{k \in k} a_k .
	$$
\end{example}

\begin{example}
	求算
	$$
		S_n=\sum_{1 \leqslant j<k \leqslant n} \frac{1}{k-j}
	$$

	$$
		\begin{aligned}
			S_n & =\sum_{1 \leqslant k \leqslant n} \sum_{1 \leqslant j \leqslant k} \frac{1}{k-j}             \\
			    & \varsub{j:=k-j}{0.7cm} \sum_{1 \leqslant k \leqslant n} \sum_{1 \leqslant k-j<k} \frac{1}{j} \\
			    & =\sum_{1 \leqslant j \leqslant n} \sum_{0<j \leqslant k-1} \frac{1}{j}                       \\
			    & =\sum_{1 \leqslant k \leqslant n} H_{k-1}                                                    \\
			    & \varsub{k :=k+1}{0.7cm} \sum_{0 \leqslant k<n} H_k
		\end{aligned}
	$$

	如果在换元之前进行代换, 就有
	$$
		\begin{aligned}
			S_n & =\sum_{1 \leqslant j<k \leqslant n} \frac{1}{k-j} \varsub{k:=k+j}{0.7cm} \sum_{1 \leqslant j \leqslant k+j \leqslant n} \frac{1}{k}=\sum_{1 \leqslant k \leqslant n} \sum_{1 \leqslant j \leqslant n-k} \frac{1}{k} \\
			    & =\sum_{1 \leqslant k \leqslant n} \frac{n-k}{k}=\sum_{1 \leqslant k \leqslant n} \frac{n}{k}-\sum_{1 \leqslant k \leqslant n} 1                                                                                     \\
			    & =n\left(\sum_{1 \leqslant k \leqslant n} \frac{1}{k}\right)-n=n H_k-n .
		\end{aligned}
	$$
\end{example}

\newpage
\section{差分与微分、求和与积分}

本节的目标是, 是不是有这样的一种记号, 使得我们可以使用类似于积分的方式记录求和? 是否能够像不定积分那样引申出``不定求和''?

\subsection{回忆: 微分与差分} 数学分析中我们定义过微分算子
$$
	\mathscr D f(x)=\frac{d f(x)}{d x}=\lim _{h \rightarrow 0} \frac{f(x+h)-f(x)}{h} .
$$

同样的可以定义差分算子.

\begin{definition}[差分] 对于一个数列, 定义其差分算子$\Delta f(x)$为
	\[
		\Delta f(x):=f(x+1)-f(x) .
	\]
\end{definition}

\begin{remark}
	\begin{enumerate}
		\item 实际上这是极限定义中, $h=1$的特例.
		\item 算子作用在函数上, 给出新的函数. 其本质就是从函数到函数的映射. 例如在多项式上面求导算子给出的实际上是映射
		      \[
			      \mathbb P [x]_n\stackrel{\mathscr D}{\longrightarrow} \mathbb P[x]_{n-1}.
		      \]
	\end{enumerate}

\end{remark}

\begin{example}
	回忆$\mathscr D x^m=m x^{m-1}$, 其中$m$是给定的数, $x$是变元.

	如果对它施以差分算子, 得到$\Delta x^3=(x+1)^3-x^3=3 x^2+3 x+1$. 这表明求导算子和差分算子他们之间的性质有所不同.
	接下来考察特殊的多项式, 使得差分算子得到类似的结构.

\end{example}

\begin{example}[下降幂]
	如果我们求
	$$
		\Delta \underbrace{x(x-1) \cdots(x-m+1)}_{m \text { 项 }})
	$$
	就会得到
	$$
		\begin{aligned}
			 & \Delta(\underbrace{(x(x-1) \cdots(x-m+1))}_{m \text { 项 }} \\
			 & =(x+1) x \cdots(x-m+2)-x(x-1) \cdots(x-m+1)                \\
			 & =\underbrace{m(x-1) \cdots(x-m+2)}_{m-1 \text { 项 }}
		\end{aligned}
	$$
	说明$\Delta$算子在上述的多项式上面作用有与求导算子在$x^m$次多项式有类似的效果.
\end{example}

我们先定义
$$
	\underbrace{x(x-1) \cdots(x-m+1)}_{m \text { 项 }}
$$
为$x$的$m$次下降幂. 记作$x^{\underline m}$

\begin{proposition}
	由于上面的性质, 注意到
	$$
		\Delta x^{\underline{m}}=m x^{\underline{m-1}} \text {. }
	$$
\end{proposition}

我们发现, $\mathscr D$ 有逆运算$\int$, 意味着$\int \cdot \mathscr D=\text{id}$. 并且微积分基本定理告诉我们:
$$
	g(x)=\mathscr D f(x) \Leftrightarrow \int g(x) d x=f(x)+C .
$$

\begin{theorem}[差分基本定理] 对于差分而言,
	$$
		g(x)=\Delta f(x) \Leftrightarrow \sum g(x) \delta x=f(x)+C .
	$$
	这里$\sum g(x) \delta x$是$g(x)$的不定和式, 是差分等于$f(x)$的函数类. 其中$C$是满足$p(x+1)=p(x)$的任意一个函数$p(x)$.
\end{theorem}

正如微积分中有定积分这件事情, 有限和式也可以仿照写作``定和式''. 仿照
$\int_a^b g(x) d x=[f(x)]_a^b=f(b)-f(a)$.
我们给出下面的定义.

\begin{definition}[定和式]
	定义
	$$
		\sum_a^b g(x) \delta x=[f(x)]_a^b=f(b)-f(a) .
	$$
	如果$g(x)=\Delta f(x) .$

\end{definition}

\begin{proposition}
	对于这样的一个和式, 有
	$$
		\sum_a^b g(x) \delta x=\sum_{k=a}^{b-1} g(k)=\sum_{a \leq k<b} g(k) .
	$$

\end{proposition}

\begin{proof}
	若 $b=a , \quad \sum_a^a g(x) \delta x=f(a)-f(a)=0$ ;
	若 $b:=a+1, \quad \sum_a^b g(x) \delta x=f(a+1)-f(a)=g(a)$.
	若 $b:=b+1$,
	$$
		\begin{aligned}
			\sum_a^{b+1} g(x) \delta x-\sum_a^b g(x) \delta x & =(f(b+1)-f(a))-(f(b)-f(a)) \\
			                                                  & =f(b+1)-f(b)=g(b) .
		\end{aligned}
	$$
	根据数学归纳法可以证明.
\end{proof}

\begin{proof}
	另外的证明: 考虑裂项(伸缩法, telescoping)
	$$
		\begin{aligned}
			\sum_{a \leq k<b} f(k+1)-f(k)= & (f(a-1)-f(a))+(f(a+2)-f(a+1))+\cdots \\
			                               & +(f(b)-f(b-1))                       \\
			=                              & f(b)-f(a) .
		\end{aligned}
	$$
\end{proof}

\begin{proposition}
	像定积分那样, 不定求和有
	$$
		\sum_a^b g(x) \delta x=-\sum_b^a g(x) \delta x .
	$$
\end{proposition}

\begin{proof}
	实际上
	$$
		f(b)-f(a)=-(f(a)-f(b))=-\sum_b^a g(x) \delta x .
	$$
\end{proof}

\begin{proposition}
	像定积分那样, 不定求和有
	$$
		\sum_a^b g(x) \delta x+\sum_b^c g(x) \delta x=\sum_a^c g(x) \delta x .
	$$
\end{proposition}

\subsection{带来的好处: 有限微积分}

\subsubsection{下降幂} 下面给$m<0$的时候下降幂给出定义. 观察

\[
	\begin{array}{ll}
		x^3=x(x-1)(x-2)             & \text { 除以 } x-2   \\
		x^2=x(x-1)                  & \text { 除以 } x-1   \\
		x^2=x                       & \text { 除以 } x     \\
		x^{-0}=1                    & \text { 除以 } x+1   \\
		x^{-1}=\frac{1}{x+1}        & \text { 除以 } x+2 . \\
		x^{-2}=\frac{1}{(x+1)(x+2)} &
	\end{array}
\]

\begin{definition}[下降幂]
	我们定义
	$$
		\underbrace{x(x-1) \cdots(x-m+1)}_{m \text { 项 }}
	$$
	为$x$的$m$次下降幂. 记作$x^{\underline m}$.

	如果$m>0$,
	$$
		m>0, x^{\underline{-m}}=\frac{1}{(x+1) \cdots(x+m)}
	$$

\end{definition}

像这样的扩展定义仍然保留了对应的性质.

\begin{proposition}[下降幂的性质]
	$$
		\forall m, n \in \mathbb{Z}, \quad x^{m+n}=x^m x^n .
	$$
\end{proposition}

\begin{proposition}[下降幂的满足差分的性质]
	若$m>0, \Delta x^{\frac{-m}{n}}=-m x^{-m-1}$.
\end{proposition}

因此可得,
对于下降幂, 有
$$
	\sum_a^b x^m \delta x=\left[\frac{x^{m+1}}{m+1}\right]_a^b \quad m \neq-1 .
$$

\subsubsection{调和级数} 根据上述的描述, 如果 $m=-1 , \quad x^{-1}=\frac{1}{x+1}=\Delta f(x)=f(x+1)-f(x)$. 若 $x \in \mathbb{Z} , f(x)=\frac{1}{1}+\frac{1}{2}+\cdots+\frac{1}{x}=H_n$. 所以
\begin{proposition}
	类似与积分的情形,
	$$
		\sum_a^b x^m \delta x=\left\{\begin{array}{cl}
			{\left[\frac{x^{m+1}}{m+1}\right]_a^b} & , m \neq-1 \\
			{\left[H_x\right]_a^b,}                & m=1
		\end{array}\right.
	$$
\end{proposition}

\subsubsection{指数函数的类似物}
由于 $\mathscr D e^x=e^x$, 希望找一个 $\Delta f(x)=f(x)$. 实际上,
$$
	\begin{aligned}
		 & f(x+1)-f(x)=f(x) \Leftrightarrow f(x+1)=2 f(x), \text { 即 } \\
		 & f(x)=2^x .
	\end{aligned}
$$

对于 $\Delta c^x=c^{x+1}-c^x=(c-1) c^x$. 老 $c \neq 1$ ,那
\begin{proposition}
	等比数列的求和
	$$
		\sum_{a \leqslant k<b}=\sum_a^b c^x \delta x=\left[\frac{c^x}{c-1}\right]_a^b=\frac{c^b-c^a}{c-1}.
	$$
\end{proposition}

\subsubsection{差分表和加法, 乘法的差分}
我们希望得到如下的差分表:
$$
	\begin{array}{|ll|}
		\hline  f=\sum g       & \boxed{\Delta f}=g .                 \\
		\hline x^{\frac{m}{m}} & m x^{\frac{m-1}{}}                   \\
		2^x                    & 2^x                                  \\
		c^x                    & (c-1) c^x                            \\
		\hline c \cdot u       & c \cdot \Delta u                     \\
		u+v                    & \Delta u+\Delta v                    \\
		u v                    & u \Delta v+E v \Delta u\text{ (见下文)} \\
		\hline
	\end{array}
$$

具体地, 我们需要看加法, 乘法的情形下, 差分的变化. 注意, 复合函数差分在离散的情形没有很好的对应物.

\subsection{分部求和法}
注意到
$$
	\begin{aligned}
		 & \Delta(u(x) v(x))= u(x+1) v(x+1)-u(x) v(x) \\
		 & = u(x+1) v(x+1)-u(x) v(x+1)                \\
		 & +u(x) v(x+1)-u(x) v(x)                     \\
		 & =u(x) \Delta v(x)+v(x+1) \Delta u(x)
	\end{aligned}
$$

若定义$E f(x):=f(x+1)$, 那么
$$
	\Delta(u v)=u \Delta v+(E v) \Delta u \text {. }
$$
此时, 对两边同时取$\sum$, 有
$$
	\sum u \Delta v=u v-\sum(E \eta) \Delta u .
$$
\begin{remark}
	实际上, 这个还有另一种选择方式, 如
	$$
		\begin{aligned}
			\Delta(u v) & =u \Delta v+E v \Delta u \\
			            & =E u \Delta v+v \Delta u
		\end{aligned}
	$$
	两种形式都是正确的, 因此左右是对称的.
\end{remark}

\begin{example}仿照 $\int x e^x d x$ , 求 $\sum x 2^x \delta x$
	$$
		\begin{aligned} \sum x\boxed{2^x} \delta x=\sum x \delta\left(2^x\right) & =x 2^x-\sum 2^{\boxed{x+1}} \delta x \\ & =x 2^x-2^{x+1}+c .\end{aligned}
	$$
	那么
	$$
		\begin{aligned}
			\sum_{k=0}^n k 2^k & =\sum_0^{n+1} x 2^x \delta x=\left[x 2^x-2^{x+1}\right]_0^{n+1} \\
			                   & =(n-1) 2^{n+1}+2 .
		\end{aligned}
	$$
\end{example}

\begin{example}仿照$\int x \ln x d x$, 求 $\sum k H_k \delta k$,
	$$
		\begin{aligned}
			\sum \boxed{x} H_x \delta x & = \sum H_x \delta\left(\frac{x^{\underline{2}}}{2}\right)                                \\
			                            & = \frac{x^2}{2} H_x - \sum \frac{(x+1)^{\underline{2}}}{{2}} x^{\underline{-1}} \delta x \\
			                            & = \frac{x^2}{2} H_x - \frac{1}{2} \sum x^{\underline{1}} \delta x                        \\
			                            & = \frac{x^2}{2} H_x - \frac{x^2}{4} + C.
		\end{aligned}
	$$

	那么
	$$
		\sum_{0 \leqslant k<n} k H_k=\left[\frac{x^2}{2} H_x-\frac{x^2}{4}+C\right]_0^n=\frac{n^{\underline{2}}}{2}\left(H_n-\frac{1}{2}\right) .
	$$

\end{example}

\end{document}
