\documentclass{ctexart}
\usepackage[utf8]{inputenc}

% \usepackage[
% HomeHTMLFilename=index, % Filename of the homepage.
% HTMLFilename={node-}, % Filename prefix of other pages.
% % IndexLanguage=english, % Language for xindy index, glossary.
% % latexmk, % Use latexmk to compile.
% % mathjax, % Use MathJax to display math.
% ]{lwarp}

\usepackage{amsmath, amsthm, amssymb, amsfonts}
\usepackage{thmtools}
\usepackage{graphicx}
\usepackage{setspace}
\usepackage{geometry}

\usepackage{float}
\usepackage{amsthm}
\usepackage{hyperref}


\usepackage{cleveref}
\crefname{equation}{式}{式}
\crefname{figure}{图}{图}
\crefname{table}{表}{表}
\crefname{page}{页}{页}
\crefname{chapter}{章}{章}
\crefname{section}{节}{节}
\crefname{appendix}{附录}{附录}
\crefname{theorem}{定理}{定理}
\crefname{lemma}{引理}{引理}
\crefname{corollary}{推论}{推论}
\crefname{proposition}{命题}{命题}
\crefname{definition}{定义}{定义}
\crefname{example}{例}{例}
\crefname{algorithm}{算法}{算法}
\crefname{listing}{列表}{列表}
\crefname{line}{行}{行}

\crefformat{chapter}{#2第#1章#3}
\crefformat{section}{#2第#1节#3}
\crefformat{subsection}{#2第#1节#3}
\crefformat{subsubsection}{#2第#1节#3}

\crefrangeformat{chapter}{#3第#1章#4至#5第#2章#6}
\crefrangeformat{section}{#3第#1节#4至#5第#2节#6}
\crefrangeformat{subsection}{#3第#1节#4至#5第#2节#6}
\crefrangeformat{subsubsection}{#3第#1节#4至#5第#2节#6}

\crefmultiformat{chapter}{#2第#1章#3}{和#2第#1章#3}{, #2第#1章#3}{和#2第#1章#3}
\crefmultiformat{section}{#2第#1节#3}{和#2第#1节#3}{, #2第#1节#3}{和#2第#1节#3}
\crefmultiformat{subsection}{#2第#1节#3}{和#2第#1节#3}{, #2第#1节#3}{和#2第#1节#3}
\crefmultiformat{subsubsection}{#2第#1节#3}{和#2第#1节#3}{, #2第#1节#3}{和#2第#1节#3}

\crefrangemultiformat{chapter}{#3第#1章#4至#5第#2章#6}{和#3第#1章#4至#5第#2章#6}{, #3第#1章#4至#5第#2章#6}{和#3第#1章#4至#5第#2章#6}
\crefrangemultiformat{section}{#3第#1节#4至#5第#2节#6}{和#3第#1节#4至#5第#2节#6}{, #3第#1节#4至#5第#2节#6}{和#3第#1节#4至#5第#2节#6}
\crefrangemultiformat{subsection}{#3第#1节#4至#5第#2节#6}{和#3第#1节#4至#5第#2节#6}{, #3第#1节#4至#5第#2节#6}{和#3第#1节#4至#5第#2节#6}
\crefrangemultiformat{subsubsection}{#3第#1节#4至#5第#2节#6}{和#3第#1节#4至#5第#2节#6}{, #3第#1节#4至#5第#2节#6}{和#3第#1节#4至#5第#2节#6}

\newcommand{\crefpairconjunction}{~和~}
\newcommand{\crefmiddleconjunction}{, }
\newcommand{\creflastconjunction}{~和~}
\newcommand{\crefpairgroupconjunction}{~和~}
\newcommand{\crefmiddlegroupconjunction}{, }
\newcommand{\creflastgroupconjunction}{~和~}
\newcommand{\crefrangeconjunction}{~至~}


% \usepackage{mathabx}

\usepackage[english]{babel}
\usepackage{framed}
\usepackage[dvipsnames]{xcolor}
\usepackage[skins,breakable]{tcolorbox}
\usepackage{awesomebox}
\usepackage{mathrsfs}  
\usepackage{xcolor}
\usepackage{wrapfig}
% \usepackage{algorithm2e}
\usepackage{algorithm}
\usepackage[noend]{algpseudocode}
% \RestyleAlgo{ruled}

\usepackage{pstricks-add}
\usepackage{epsfig}
\usepackage{pst-grad} % For gradients
\usepackage{pst-plot} % For axes
\usepackage[space]{grffile} % For spaces in paths
\usepackage{etoolbox} % For spaces in paths
\makeatletter % For spaces in paths
\patchcmd\Gread@eps{\@inputcheck#1 }{\@inputcheck"#1"\relax}{}{}
\makeatother

% Make SS at the beginning of a section

\makeatletter
%% See pp. 26f. of 'The LaTeX Companion,' 2nd. ed.
\def\@seccntformat#1{\@ifundefined{#1@cntformat}%
    {\csname the#1\endcsname\quad}%      default
    {\csname #1@cntformat\endcsname}}%   individual control
\newcommand{\section@cntformat}{\S\thesection\quad}
\newcommand{\subsection@cntformat}{\S\thesubsection\quad}
\makeatother % changes @ back to a special character

\usepackage{titlesec}

\CTEXsetup[format={\raggedright\large\bfseries}]{section}
\titleformat{\subsection}[runin]{\normalfont\bfseries}{\thesubsection.}{0.5em}{}[.]
\titleformat{\subsubsection}[runin]{\normalfont\bfseries}{\alph{subsubsection})}{0.5em}{}


\usepackage{annotate-equations}



\theoremstyle{definition}
\newtheorem{example}{例子}[section]
\newtheorem{definition}{定义}[section]
\newtheorem{theorem}{定理}[section]
\newtheorem{proposition}[theorem]{命题}
\newtheorem{prop}[theorem]{性质}
\newtheorem{corollary}[theorem]{推论}

\newenvironment{remark}{%
  \par\medskip
  \noindent
  \textbf{注:}
}{%
  \par\medskip
}

\newenvironment{solution}{%
  \par\medskip
  \noindent
  \textbf{解答:}
}{%
  \par\medskip
}

\newenvironment{solution*}{%
  \par\medskip
  \noindent 
  \color{gray}\small\textbf{提示或解答:}
}{%
  \par\medskip
}

\newenvironment{definition*}{%
  \par\medskip
  \noindent
  \textbf{定义:}
}{%
  \par\medskip
}

\newenvironment{lemma}{%
  \par\medskip
  \noindent
  \textbf{引理:}
}{%
  \par\medskip
}

\newenvironment{proposition*}{%
  \par\medskip
  \noindent
  \textbf{性质: }
}{%
  \par\medskip
}


\newtcolorbox{asidebox}{
  colback=gray!10,
  colframe=gray!60,
  fonttitle=\bfseries,
  title={Aside},
  breakable=true
}

\newtcolorbox{webaside}{
  colback=cyan!10,
  colframe=cyan!60,
  fonttitle=\bfseries,
  title={Web Demonstrate Aside},
  breakable=true
}

\usepackage{enumitem}

\setlist{nosep}

\setstretch{1.2}
\geometry{
    textheight=9in,
    textwidth=5.5in,
    top=1in,
    headheight=12pt,
    headsep=25pt,
    footskip=30pt
}

\usepackage{environ}
\usepackage[tikz]{bclogo}
\usepackage{tikz}
\usetikzlibrary{calc}
\NewEnviron{takeaway}
  {\par\medskip\noindent
  \begin{tikzpicture}
    \node[inner sep=0pt] (box) {\parbox[t]{.99\textwidth}{%
      \begin{minipage}{.3\textwidth}
      \centering\tikz[scale=5]\node[scale=3,rotate=30]{\bclampe};
      \end{minipage}%
      \begin{minipage}{.65\textwidth}
      \textbf{Takeaway Message}\par\smallskip
      \BODY
      \end{minipage}\hfill}%
    };
    \draw[red!75!black,line width=3pt] 
      ( $ (box.north east) + (-5pt,3pt) $ ) -- ( $ (box.north east) + (0,3pt) $ ) -- ( $ (box.south east) + (0,-3pt) $ ) -- + (-5pt,0);
    \draw[red!75!black,line width=3pt] 
      ( $ (box.north west) + (5pt,3pt) $ ) -- ( $ (box.north west) + (0,3pt) $ ) -- ( $ (box.south west) + (0,-3pt) $ ) -- + (5pt,0);
  \end{tikzpicture}\par\medskip%
}


\newcommand{\lecture}[6]{
   \begin{center}
   \framebox{
      \vbox{\vspace{2mm}
    \hbox to 6.28in { {\bf #1
		\hfill #6} }
       \vspace{4mm}
       \hbox to 6.28in { {\Large \hfill 第 #2 节: #3  \hfill} }
       \vspace{2mm}
       \hbox to 6.28in { {\it Lecturer: #4 \hfill Scribes: #5} }
      \vspace{2mm}}
   }
   \end{center}
   
   \vspace*{4mm}
}

\usepackage{marginnote}
\renewcommand*{\marginfont}{\color{gray}\ttfamily\small}
\usepackage{setspace}
\newcounter{paranum}[section]
\newcommand{\Par}[1]{\vspace{10pt}\noindent\textbf{\refstepcounter{paranum}\theparanum. }\textbf{#1}~~}
\newcommand{\lec}[1]{\reversemarginpar\marginnote{{\textbf{#1}}}}
\newcommand{\mn}[1]{\marginnote{{#1}}}
% \renewcommand{\algorithmcfname}{算法}
\usepackage[abspath]{currfile}

\newcommand{\incfig}[1]{\begin{center}\includegraphics[width=.4\textwidth]{figs/\currfilename/#1}\end{center}}
\newcommand{\incfigw}[1]{\begin{center}\includegraphics[width=.8\textwidth]{figs/\currfilename/#1}\end{center}}



\newcommand{\set}[1]{\{#1\}}
\newcommand{\stirling}[2]{\left\{{#1 \atop #2}\right\}}
\newcommand{\binomt}[2]{\left(\left({#1 \atop #2}\right)\right)}
\newcommand{\pf}[4]{#1_{#2}^{#3_{#4}}}
\newcommand{\pl}[4]{#1_{#2}{#3^{#4}}}
\newcommand{\ty}[3]{{#1} \equiv {#2} ~(\bmod {#3})}
\newcommand{\Z}{{\mathbb Z}}
\newcommand{\one}{\mathbf{1}}
\newcommand{\varsub}[2]{\stackrel{#1}{\stackrel{\rule{#2}{0.4pt}}{\rule{#2}{0.4pt}}}}
\newcommand{\dd}{\mathrm{d}}
\newcommand{\Ep}[1]{\mathbb E\left(#1\right)}

\renewcommand{\red}[1]{{{\color{red}#1}}}
\newcommand{\teal}[1]{{{\color{teal}#1}}}
\renewcommand{\blue}[1]{{{\color{blue}#1}}}
\newcommand{\purple}[1]{{{\color{purple}#1}}}
\DeclareMathOperator{\var}{Var}
\newcommand{\E}{\mathbb E}
\newcommand{\like}{$\blacktriangleright$}
\newcommand{\exrate}[1]{{[#1]~}}
\newcommand{\newword}[2]{{\textbf{#1(#2)}\index{#1}}}
\newcommand{\newenword}[1]{{\textbf{#1}\index{#1}}}

\newcommand{\bzc}{\not\backslash}
\newcommand{\zc}{\backslash}
\begin{document}

\section{整除性}

\subsection{整除记号}

\begin{definition}
    我们称 $m$ 整除 $n$ (或 $n$ 可以被 $m$ 整除),当且仅当$m>0$ ,并且 $\frac{n}{m}$ 是一个整数。可记作 $m \backslash n$. 即
$$
m \backslash n \Leftrightarrow(m>0) \wedge(\exists k \in \mathbb{Z}, n=m k) .
$$

如果 $m$ 不整除 $n$ ,记作 $m \bzc n$ 。

\end{definition}

\begin{definition}[最大公约数]
    两个整数$m, n$的最大公约数是最大的可以整除$m$和$n$的那个数. 即

    $$
\operatorname{gcd}(m, n)=\max \{k: k \backslash m \text { 且 } k \backslash n\} .
$$
\end{definition}

比如$\operatorname{gcd}(12,18)=6$ ,那么可以进行分数进行化简, 即$\frac{12}{18}=\frac{12 / 6}{18/6}=\frac{2}{3}$.

\begin{remark}
    若 $n>0$ ,那么 $\operatorname{gcd}(0, n)=n$ ,因为任何一个整数都整除 0 .
\end{remark}

\begin{definition}[最小公倍数]
    两个整数 $m,n$ 的最小公倍数是最小的可以被$m,n$整除的那个数. 即
    $$
\operatorname{lcm}(m, n)=\min \{k>0 ; m \backslash k \wedge n \backslash k\}
$$
\end{definition}

\subsection{求最大公约数的算法}

算法描述: $\gcd(m,n)$是
\begin{itemize}
    \item 若$m=0$成立, 那么$\gcd(m,n)=n$.
    \item 否则, $\gcd(m,n)=\gcd(n \bmod m, m)$.
\end{itemize}

这个递归式是有效的, 因为从公因式的角度来讲, 任何一个$m$和$n$的公因式都一定是$m$和$n \bmod m$(即$n-\left\lfloor\frac{n}{m}\right\rfloor m$) 的公因式. 下面只要考虑求出来的这个公因数一定是最大的即可. 

可以构造一个方程, 来辅助说明这为什么是对的: 考虑找到$m', n'$, 满足$m'm+n'n=\gcd(m,n)$, 其中, $m', n'$由下面的内容给出: 
\begin{itemize}
    \item 如果$m=0$, 那么置$m' \leftarrow 0, n' \leftarrow 1$. 
    \item 否则, 置$r\leftarrow n \bmod m$, 并使用$r, m$代替$m, n$(这是在倒着做上面的递归算法). 这时候我们需要知道他们前面的系数是什么. 因此需要找到$r,m$前面的系数, 记作$\overline{r}, \overline{m}$, 满足$\overline{r}r+\overline{m}m=\gcd(r,m)$.
        \begin{itemize}
            \item 由于$r=n-\left\lfloor\frac{n}{m}\right\rfloor m$, 并且$\operatorname{gcd}(r, m)=\operatorname{gcd}(m, n)$, 代入得$$\bar{r}\left(n-\left\lfloor\frac{n}{m}\right\rfloor m\right)+\bar{m} m=\operatorname{gcd}(m, n).$$
            \item 重写上述项, 收集$m, n$项, 就有了$$\underbrace{\left(\bar{m}-\left\lfloor\frac{n}{m}\right\rfloor \bar{r}\right)}_{m^{\prime}} m+\underset{n^{\prime}}{\bar{r}} n=\operatorname{gcd}(m, n).$$
            \item 因此我们可以将$m' \leftarrow \bar{m}-\left\lfloor\frac{n}{m}\right\rfloor \bar{r}$, $n' \leftarrow \overline{r}$.
        \end{itemize}
\end{itemize}

对于上述的过程, 给出一个例子, 如在求$\gcd(12, 18)$的时候表示出6: 
\[
\begin{aligned}
6 & =0.0+1 \cdot 6 \\
& =(1-0) \cdot 6+0.12 \\
& =(-1) \cdot 12+1 \cdot 18
\end{aligned}
\]

接下来解释为什么是最大的: 假设上述算法得到$\gcd(m,n)=d$, 以及$m^{\prime} m+n^{\prime} n=d$, 但是存在一个$d'>d$, 满足$d\zc m \land d\zc n$. 
由于任何$m, n$的公因子都要整除$m'm+n'n$, 它也要整除$d$, 因此它一定要小于$d$. 这就达到了矛盾. 

下面介绍几个常用的性质. 

\begin{prop}
    $k \backslash m \wedge k \backslash n \Leftrightarrow k \backslash \operatorname{gcd}(m, n)$.
\end{prop}

\begin{proof}
    若 $k \backslash m, k \backslash n, k \backslash\left(m^{\prime} m+n^{\prime} n\right) \Rightarrow k \backslash \operatorname{gcd}(m, n)$.
\end{proof}

\begin{prop}[gcd与公因子的关系] 每一个公因子是他们最大公因数的因子
    
\end{prop}

\subsection{对因子的求和及其常见变形}

\begin{prop}[因子的对偶性]

    $$
\sum_{m \backslash n} a_m=\sum_{m \backslash n} a_{n / m} \quad, n>0, n \in \mathbb{Z} .
$$
    
\end{prop}
    
\begin{prop}[求和记号的交换]
    $$
\sum_{m \backslash n} \sum_{k \backslash m} a_{k, m}=\sum_{k \backslash n} \sum_{l \backslash(m / k)} a_{k, k l}
$$
\end{prop}

\begin{proof}
    这是因为可以将求和记号的整除用Iverson括号来表达. 即
    $$
\sum_{m \mid n} a_m=\sum_k a_m[n=m k]
$$
从而, 如果把等式的左手边记作LHS, 右手边记作RHS, 有
$$
\begin{aligned}
& \text { LHS }=\sum_{j, l} \sum_{k, m>0} a_{k, m}[n=j m][m=k l]=\sum_j \sum_{k, l>0} a_{k, k l}[n=j k l] \\
& \text { RHS }=\sum_{j, m} \sum_{k, l>0} a_{k, k l}[n=j k]\left[\frac{n}{k}=m l\right]=\sum_m \sum_{k, l>0} a_{k, k l}[n=m l k]
\end{aligned}
$$
从而等式的左手边等于等式的右手边. 
\end{proof}

这同样可以使用直观的方式: 把要求的和式写成三角形, 然后从两个不同的方面进行描述即可. 

\section{质数}

下文中, 若无特殊说明, 总是用$p$代表一个质数. 

\subsection{基本的定义}

\begin{definition}[质数] 如果一个数$p$有且仅有两个正因数, 也就是1和$p$, 我们称它为质数. 
    特别地, 1不是质数. 
\end{definition}

\begin{definition}[合数] 有大于2个正因子的数叫做合数. 特别地, 1不是合数. 
\end{definition}

实际上, 大于2的数之间, 一个数要么是质数, 要么是合数. 但不可能同时是两者. 

实际上, 质数是构成自然数的骨架. 我们给出如下的定理: 

\begin{theorem}[唯一分解定理] 任何一个数$n$都可以分解为若干个质数的乘积. 
    $$
n=p_1 \cdots p_m=\prod_{k=1}^m p_k, \quad p_1 \leqslant \cdots \leqslant p_m .
$$
其中, $p_1, p_2, \cdots, p_m$都是质数. 并且这个分解式唯一. 
    
\end{theorem}

\begin{proof}
    [存在性] 使用数学归纳法. 

    [唯一性] 考虑反证法. 假设同一个数存在两个不同的分解方法
    $$
\begin{aligned}
    n & =p_1 \cdots p_m & p_1\leq p_2\leq ...\leq p_m\\
      & =q_1 \cdots q_{k} & q_1 \leq q_2 \leq  ... \leq  q_k
\end{aligned}
$$
以及$p_i, q_i$都是质数. 


\motivation{$p_1=q_1$} 假设$p_1<q_1$, 根据他们都是质数, 故$\gcd(p_1, q_1)=1.$ 根据Euclid算法, 存在$a, b$, 使得
$$
a p_1 q_2 \cdots q_k+b q_1 q_i \cdots q_k=q_2 \cdots q_k .
$$
$$
\text { 现在 } p_1\left|a p_1 q_2 \cdots q_k, p_1\right| b q_1 q_2 \cdots q_k, \Rightarrow p_1 \mid q_2 \cdots q_k
$$
而这就意味着$\frac{q_2 \cdots q_k}{p_1} \in \mathbb{Z}$ , 且 $q_2 \cdots q_k$ 有一分解使 $q_1$ 出现.
但是$q_2 ... q_n < n$, 根据归纳法, 其必有唯一分解. 矛盾! 
因此, $p_1=q_1$. 此时可以在等两端同时除以$p_1, q_1$. 重复刚才的过程, 可以证明$p_2=q_2$等. 

\end{proof}

\begin{remark}
    此式子有时候也可以记作
    $$
n=\prod_p p^{n_p}, \quad n_p \geqslant 0 .
$$
其中$n_p \geq 0$, $n_p$是质数$p$出现的次数. 
\end{remark}

由此我们便可以得到整数类似于``坐标''的表示. 只不过这里的维数是无穷维的. 

例如, 下面方框里面的就可以认为是整数的``坐标''. 

\begin{align*}
12 & =2^{\text{\fbox{2}}}\times3^{\text{\fbox{1}}}\times5^{\text{\fbox{0}}}\times7^{\text{\fbox{0}}}\times\cdots\\
18 & =2^{\text{\fbox{1}}}\times3^{\text{\fbox{2}}}\times5^{\text{\fbox{0}}}\times7^{\text{\fbox{0}}}\times\cdots
\end{align*}

从而, 12和18的坐标可以记作
$$
12=\langle 2,1,0,0\rangle, 18=\langle 1,2,0,0, \cdots\rangle
$$

在这个坐标系下, 对于一个数的坐标形式, 即$\left\langle n_2, n_3, n_5, n_7, \cdots\right\rangle$, 我们发现
$$\begin{aligned} & k=m n &\Leftrightarrow& k_p=m_p+n_p, &\forall p . \\ & m \backslash n & \Leftrightarrow& m_p\leq n_p\quad, &\forall p \\ & k=\operatorname{gcd}(m, n) &\Leftrightarrow& k_p=\min \left(m_p, n_p\right), &\forall p \\ & k=\operatorname{lcm}(m, n) &\Leftrightarrow& k_p=\max \left(m_p, n_p\right), &\forall p .\end{aligned}$$

\end{document}
