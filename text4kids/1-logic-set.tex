\documentclass{ctexart}

\usepackage{amsmath, amsthm, amssymb, amsfonts}
\usepackage{thmtools}
\usepackage{graphicx}
\usepackage{setspace}
\usepackage{geometry}

\usepackage{float}
\usepackage{amsthm}
\usepackage{hyperref}
\usepackage{cleveref}
% \usepackage{mathabx}
\usepackage[utf8]{inputenc}
\usepackage[english]{babel}
\usepackage{framed}
\usepackage[dvipsnames]{xcolor}
\usepackage[skins,breakable]{tcolorbox}
\usepackage{awesomebox}
\usepackage{mathrsfs}  
\usepackage{xcolor}
\usepackage{wrapfig}
\usepackage{algorithm2e}
\RestyleAlgo{ruled}

\usepackage{pstricks-add}
\usepackage{epsfig}
\usepackage{pst-grad} % For gradients
\usepackage{pst-plot} % For axes
\usepackage[space]{grffile} % For spaces in paths
\usepackage{etoolbox} % For spaces in paths
\makeatletter % For spaces in paths
\patchcmd\Gread@eps{\@inputcheck#1 }{\@inputcheck"#1"\relax}{}{}
\makeatother

% Make SS at the beginning of a section

\makeatletter
%% See pp. 26f. of 'The LaTeX Companion,' 2nd. ed.
\def\@seccntformat#1{\@ifundefined{#1@cntformat}%
    {\csname the#1\endcsname\quad}%      default
    {\csname #1@cntformat\endcsname}}%   individual control
\newcommand{\section@cntformat}{\S\thesection\quad}
\newcommand{\subsection@cntformat}{\S\thesubsection\quad}
\makeatother % changes @ back to a special character

\usepackage{titlesec}

\CTEXsetup[format={\raggedright\large\bfseries}]{section}
\titleformat{\subsection}[runin]{\normalfont\bfseries}{\thesubsection.}{0.5em}{}[.]
\titleformat{\subsubsection}[runin]{\normalfont\bfseries}{\alph{subsubsection})}{0.5em}{}





\theoremstyle{definition}
\newtheorem{example}{例子}[section]
\newtheorem{definition}{定义}[section]
\newtheorem{theorem}{定理}[section]
\newtheorem{proposition}[theorem]{命题}
\newtheorem{prop}[theorem]{性质}
\newtheorem{corollary}[theorem]{推论}

\newenvironment{remark}{%
  \par\medskip
  \noindent
  \textbf{注:}
}{%
  \par\medskip
}

\newenvironment{solution}{%
  \par\medskip
  \noindent
  \textbf{解答:}
}{%
  \par\medskip
}

\newenvironment{solution*}{%
  \par\medskip
  \noindent 
  \color{gray}\small\textbf{提示或解答:}
}{%
  \par\medskip
}

\newenvironment{definition*}{%
  \par\medskip
  \noindent
  \textbf{定义:}
}{%
  \par\medskip
}

\newenvironment{lemma}{%
  \par\medskip
  \noindent
  \textbf{引理:}
}{%
  \par\medskip
}

\newenvironment{proposition*}{%
  \par\medskip
  \noindent
  \textbf{性质: }
}{%
  \par\medskip
}


\newtcolorbox{asidebox}{
  colback=gray!10,
  colframe=gray!60,
  fonttitle=\bfseries,
  title={Aside},
  breakable=true
}

\newtcolorbox{webaside}{
  colback=cyan!10,
  colframe=cyan!60,
  fonttitle=\bfseries,
  title={Web Demonstrate Aside},
  breakable=true
}

\usepackage{enumitem}

\setlist{nosep}

\setstretch{1.2}
\geometry{
    textheight=9in,
    textwidth=5.5in,
    top=1in,
    headheight=12pt,
    headsep=25pt,
    footskip=30pt
}

\usepackage{environ}
\usepackage[tikz]{bclogo}
\usepackage{tikz}
\usetikzlibrary{calc}
\NewEnviron{takeaway}
  {\par\medskip\noindent
  \begin{tikzpicture}
    \node[inner sep=0pt] (box) {\parbox[t]{.99\textwidth}{%
      \begin{minipage}{.3\textwidth}
      \centering\tikz[scale=5]\node[scale=3,rotate=30]{\bclampe};
      \end{minipage}%
      \begin{minipage}{.65\textwidth}
      \textbf{Takeaway Message}\par\smallskip
      \BODY
      \end{minipage}\hfill}%
    };
    \draw[red!75!black,line width=3pt] 
      ( $ (box.north east) + (-5pt,3pt) $ ) -- ( $ (box.north east) + (0,3pt) $ ) -- ( $ (box.south east) + (0,-3pt) $ ) -- + (-5pt,0);
    \draw[red!75!black,line width=3pt] 
      ( $ (box.north west) + (5pt,3pt) $ ) -- ( $ (box.north west) + (0,3pt) $ ) -- ( $ (box.south west) + (0,-3pt) $ ) -- + (5pt,0);
  \end{tikzpicture}\par\medskip%
}

\usepackage{marginnote}
\renewcommand*{\marginfont}{\color{gray}\ttfamily\small}
\usepackage{setspace}
\newcounter{paranum}[section]
\newcommand{\Par}[1]{\vspace{10pt}\noindent\textbf{\refstepcounter{paranum}\theparanum. }\textbf{#1}~~}
\newcommand{\lec}[1]{\reversemarginpar\marginnote{{\textbf{#1}}}}
\newcommand{\mn}[1]{\marginnote{{#1}}}
\renewcommand{\algorithmcfname}{算法}
\usepackage[abspath]{currfile}

\newcommand{\incfig}[1]{\begin{center}\includegraphics[width=.4\textwidth]{figs/#1}\end{center}}
\newcommand{\incfigw}[1]{\begin{center}\includegraphics[width=.8\textwidth]{figs/#1}\end{center}}
\newcommand{\set}[1]{\{#1\}}
\newcommand{\stirling}[2]{\left\{{#1 \atop #2}\right\}}
\newcommand{\binomt}[2]{\left(\left({#1 \atop #2}\right)\right)}
\newcommand{\pf}[4]{#1_{#2}^{#3_{#4}}}
\newcommand{\pl}[4]{#1_{#2}{#3^{#4}}}
\newcommand{\ty}[3]{{#1} \equiv {#2} ~(\bmod {#3})}
\newcommand{\Z}{{\mathbb Z}}
\newcommand{\one}{\mathbf{1}}
\newcommand{\varsub}[2]{\stackrel{#1}{\stackrel{\rule{#2}{0.4pt}}{\rule{#2}{0.4pt}}}}
\newcommand{\dd}{\mathrm{d}}
\newcommand{\Ep}[1]{\mathbb E\left(#1\right)}

\renewcommand{\red}[1]{{{\color{red}#1}}}
\newcommand{\teal}[1]{{{\color{teal}#1}}}
\renewcommand{\blue}[1]{{{\color{blue}#1}}}
\newcommand{\purple}[1]{{{\color{purple}#1}}}
\DeclareMathOperator{\var}{Var}
\newcommand{\E}{\mathbb E}
\newcommand{\like}{$\blacktriangleright$}
\newcommand{\exrate}[1]{{[#1]~}}
\newcommand{\newword}[2]{{\textbf{#1(#2)}\index{#1}}}
\newcommand{\newenword}[1]{{\textbf{#1}\index{#1}}}

% 定义一个新的计数器
\newcounter{judgement}[subsubsection]

% 定义一个名为 judgement 的环境
\newenvironment{exc}
  {
    \par\addvspace{-\parskip}
    \refstepcounter{judgement}
    \fbox{\thejudgement}
  }
  {\par\addvspace{-\parskip} \ignorespacesafterend}



\begin{document}

\lecture{数学基础}{1}{逻辑符号, 集合, 函数}{张桄玮}{\underline{~~~~~~~~~~}}{Summer 2024}

本系列文本主要阐明一些基础的数学概念. 由于其基础性, 放在课件上面未免也太难阅读了. 于是统一放在这里. 请大家按需索取. 

如何得到的文本: 打开卓里奇《数学分析》第一章, 把文本照抄, 例子改为初高中例子. 我们也鼓励有志向报考数学和(或)计算机专业的同学首先预习大学数学(计算机科学)的内容.

\section{逻辑符号}

在中学的时候, 我们学习了二次方程. 我们会说``$x^2-3x+2=0$, 等价于$x=1$或$x=2$''. 数学中有相当多类似的表达. 为了表达简洁起见, 最好把它们用专门的符号代指. 

我们使用如下的记号: 

\begin{itemize}
    \item $\lnot$: 否定词\nwd{非}(not)
    \item $\land$: \nwd{与}(and)
    \item $\lor$: \nwd{或}(or)
    \item $\implies$: \nwd{蕴含}(implies)
    \item $\iff$: \nwd{等价}(if and only if, iff)
\end{itemize}



为了节约括号, 我们规定逻辑符号的优先级如$\lnot, \land, \lor, \implies, \iff$. 也就是说语句``$\lnot$ 我计算很好 $\land$ 他喜欢计算''
表示的是``($\lnot$我计算很好)$\land$(他喜欢计算) ''. 


\subsection{蕴含记号} $A\implies B $表示从$A$中可以推出$B$. 我们也常说做``$A$蕴含$B$'', ``$B$是$A$的必要特征(\nwd{必要条件})'', ``$A$是$B$的充分特征(\nwd{充分条件})''. 特别注意, 如果$A\implies B$的情况下, 如果前提条件为假, 那么$A \implies B$为真. 这可以大致理解为: 前提为假的情况下, 无论如何做都没有说谎, 故整体来看, 说的话是真的.  

对于$A \iff B$, 可以理解为$(A\implies B) \land (B\implies A)$. 也就是$A$既是$B$的充分条件, 也是必要条件. 简称\nwd{充要条件}. 通常还会用``当且仅当'', ``等价''描述他们. 


\subsection{记号的相互作用}

我们列举几个常见的表述. 请大家找一些实际的例子把$A$和$B$换掉, 感受一下这些陈述为什么是对的:

\begin{itemize}
    \item $\neg(A \wedge B) \Leftrightarrow \neg A \vee \neg B$;
    \item $\neg(A \vee B) \Leftrightarrow \neg A \wedge \neg B$;
    \item $(A \Rightarrow B) \Leftrightarrow(\neg B \Rightarrow \neg A)$;
    \item $(A \Rightarrow B) \Leftrightarrow \neg A \vee B$;
    \item $\neg(A \Rightarrow B) \Leftrightarrow A \wedge \neg B$
\end{itemize}

\subsection{数学证明} 一般而言, 证明的形式如同$A \implies B$. 其中$A$是\nwd{前提}, $B$是\nwd{结论}. 要证明这个命题, 就是建立一连串蕴含关系: 
\[
    A\implies C_1 \implies C_2 \implies ... \implies C_n \implies B
\]
其中每一个蕴含关系要么是公理, 要么是已经被证明的命题. 

证明的时候, 采用经典的推导法则(三段论): 如果$A$成立, 而且有$A\implies B$, 那么$B$也成立. 

此外, 我们还是用\nwd{排中律}. 即无论命题$A$的具体内容是什么, $A \lor \lnot A$总是成立的. 我们还认为$\lnot (\lnot A)=A$. 

\subsection{某些特殊的记号} 我们约定, 可以用专门的符号``:=''(根据定义等于)表示某件事情就是这样定义的. 其中冒号的位置位于被定义的对象的一侧. 例如
\[
    x:=2
\]
表示$x$根据定义等于2. 同样, 对于已有定义的表达式, 可以用此记号引入一个缩写. 例如
\[
    x_1+x_2+x_3=:f(x_1,x_2,x_3)
\]

需要注意的是, 我们这里做了一个关于记号的表面的说明, 根本没有讨论数理逻辑中有效性, 完备性等深刻的内容. 因为我们所掌握的, 总是比这时能够总结成的一般理论多一些. 

\section{集合及基本运算} 从19世纪末20世纪初开始, 集合论成为了最通用的数学语言. 在数学的一种定义中, 甚至提到``数学是研究集合上的各种结构(关系)的科学.''

首先来看朴素集合论. 朴素集合论的前提有三个: 

\begin{itemize}
    \item 集合由有区别的对象组成;
    \item 集合由其组成对象唯一确定; 
    \item 任何性质都确定一个具有该性质的集合.
\end{itemize}

如果$x$是一个对象, $P$是一种性质, $P(x)$表示$x$具有性质$P$, 那么用$\{ x:P(x) \}$表示具有性质$P$的整整一类的对象. 他们构成一个\nwd{集合}. \footnote{有时候也写作$\{ x | P(x) \}$.}
组成集合的对象称为集合的\nwd{元素}. 

由元素$x_1, x_2, ..., x_n$组成的集合可以写作$\{ x_1, x_2, ..., x_n \}$, 为了方便书写, 可以在不引起误会的场合用$a$代表单元素集合$\{ a \}$. \footnote{高中范围内不要用这个记号 -- 到处都是误会!}

\subsection{包含关系} 通常习惯为, 用大写的拉丁字母表示集合, 对应的小写拉丁字母表示集合中的元素. 

我们说``$x$是$X$的元素'', 或``集合$X$有一个元素$x$'', 或``$x$\nwd{属于}集合$X$'', 用符号记为
\[
x \in X ( \text{或} X \ni x)
\]
其否定记作$x \notin X$. 即``$x$\nwd{不属于}集合$X$''.

在考虑集合相关的问题时, 会经常使用``任意''和``存在''两个逻辑符号. 
\begin{itemize}
    \item $\forall$: \nwd{任何}, 对于任何的...(全称量词)
    \item $\exists$: \nwd{存在}, 可以找到... (特称量词)
\end{itemize} 

$\forall x((x \in A) \Leftrightarrow(x \in B))$ 表示, 对于任何对象 $x$, 关系 $x \in A$ 与 $x \in B$是等价的. 这是因为一个集合完全由其元素所定义, 可以简单记作$A=B$. 

如果集合 $A$ 的任何元素都是集合 $B$ 的元素, 我们就采用记号 $A \subset B$ 或 $B \supset A$, 读作 “集合 $A$ 是集合 $B$ 的\nwd{子集}”, 或者 “ $A$ {包含于} $B$ ”, 或者 “ $B$ 包含 (含有) $A$ ”. 使用符号表达就是
$$
(A \subset B):=\forall x((x \in A) \Rightarrow(x \in B))
$$

如果 $A \subset B$ 且 $A \neq B$, 我们就说, 包含关系 $A \subset B$ 是严格的, 或者 $A$ 是 $B$ 的\nwd{真子集}.

给出上述的定义, 可以得出结论
$$
(A=B) \Leftrightarrow(A \subset B) \wedge(B \subset A) \text {. }
$$

如果 $M$ 是一个集合, 则任何一个性质 $P$ 都可在 $M$ 中分离出一个子集
$$
\{x \in M : P(x)\},
$$

其元素 $M$ 具有这个性质.
例如, 显然
$$
M=\{x \in M : x \in M\} .
$$

另外, 如果取集合 $M$ 中任何元素都不具有的一个性质作为 $P$, 例如 $P(x):=(x \neq x)$,我们就得到集合
$$
\varnothing:=\{x \in M : x \neq x\},
$$
称为集合 $M$ 的\nwd{空子集}.

\subsection{最简单的集合运算} 下面考察最简单的集合运算.

a) 集合 $A$ 与 $B$ 的\nwd{并集}是指集合
$$
A \cup B:=\{x \in M :(x \in A) \vee(x \in B)\},
$$

它由全部至少属于集合 $A, B$ 之一的元素组成 .

b) 集合 $A$ 与 $B$ 的\nwd{交集}是指集合
$$
A \cap B:=\{x \in M :(x \in A) \wedge(x \in B)\},
$$

它由全部同时属于集合 $A$ 和 $B$ 的元素组成.

c) 集合 $A$ 与 $B$ 的\nwd{差集}是指集合
$$
A \backslash B:=\{x \in M :(x \in A) \wedge(x \notin B)\},
$$

它由全部属于 $A$ 但不属于 $B$ 的元素组成.
集合 $M$ 与其子集 $A$ 的差集通常称为 $A$ 在 $M$ 中的\nwd{补集},记为 $C_M A$ 或 $\overline{A}$, 后者用于从上下文显然知道在哪一个集合中求 $A$ 的补集的情况.

d) 集合的\nwd{直积} (笛卡儿积). 对于任何两个集合 $A, B$, 还以组成一个新的集合 -- $\{A, B\}=\{B, A\}$, 其元素是且仅是集合 $A$ 和 $B$. 这个集合在 $A \neq B$ 时由两个元素组成, 而在 $A=B$ 时由一个元素组成.

上述新集合称为集合 $A, B$ 的无序偶, 以区别于\nwd{序偶} $(A, B)$, 序偶的元素 $A, B$ 能够区别 $\{A, B\}$ 中的第一个元素和第二个元素. 按照定义,序偶等式
$$ 
(A, B)=(C, D)
$$
表示 $A=C$ 且 $B=D$. 特别地, 如果 $A \neq B$, 则 $(A, B) \neq(B, A)$.
现在设 $X, Y$ 是任意集合. 集合
$$
X \times Y:=\{(x, y) : (x \in X) \wedge(y \in Y)\}
$$
称为集合 $X, Y$ (按这样的顺序!) 的直积或笛卡儿积, 它是由第一项属于 $X$ 而第二项属于 $Y$ 的全部序偶 $(x, y)$ 组成的.\footnote{这类似于pair或有序对, 在第一个位置和第二个位置是不一样的}

从直积的定义和关于序偶的上述说明可以看出, 一般而言, $X \times Y \neq Y \times X$.等式仅当 $X=Y$ 时才成立, 这时 $X \times X$ 简写为 $X^2$.


\section{函数} 

\subsection{映射的概念} \nwd{映射}是十分基础的概念, 在日常生活中到处都有体现.  

设 $X$ 与 $Y$ 是某两个集合.

如果集合 $X$ 的每一个元素 $x$ 都按照某规律 $f$ 与集合 $Y$ 的元素 $y$ 相对应, 我们就说有一个\nwd{函数}, 它定义于 $X$ 并取值于 $Y$.

这时, 集合 $X$ 称为函数的\nwd{定义域}, 其元素 $x$ 称为函数的变元或{自变量}, 而与\nwd{自变量} $x$ 的具体值 $x_0 \in X$ 相对应的元素 $y_0 \in Y$ 称为元素 $x_0$ 上的或自变量 $x=x_0$时的\nwd{函数值}, 并表示为 $f\left(x_0\right)$. 当自变量 $x \in X$ 变化时, 一般而言, 值 $y=f(x) \in Y$随 $x$ 的值而变化. 因此, 量 $y=f(x)$ 经常称为\nwd{因变量}.

函数在集合 $X$ 各元素上的全部函数值的集合
$$
f(X):=\{y \in Y : \exists x((x \in X) \wedge(y=f(x)))\}
$$
称为函数的值集或\nwd{值域}.

通常使用以下记号来表示函数 (映射):
$$
f: X \rightarrow Y, \quad X \xrightarrow{f} Y .
$$

当函数的定义域和值域从上下文看很明显时, 也用记号 $x \mapsto f(x)$ 或 $y=f(x)$来表示函数, 更常见的则是只用一个字母 $f$ 来表示函数.

如果两个函数 $f_1, f_2$ 具有相同的定义域 $X$, 并且在每个元素 $x \in X$ 上的值 $f_1(x), f_2(x)$ 相同, 就认为这两个函数相同或相等, 记作 $f_1=f_2$.

如果 $A \subset X$, 而 $f: X \rightarrow Y$ 是某函数, 就用 $f \mid A$ 或 $\left.f\right|_A$ 来表示在集合 $A$.上与 $f$ 相等的函数 $\varphi: A \rightarrow Y$. 更确切地, 如果 $x \in A$, 则 $\left.f\right|_A(x):=\varphi(x)$. 函数 $\left.f\right|_A$ 称为函数 $f$ 在集合 $A$ 上的收缩或\nwd{限制}, 而相对于函数 $\varphi=\left.f\right|_A: A \rightarrow Y$ 来说, 函数 $f: X \rightarrow Y$ 称为函数 $\varphi$ 在集合 $X$ 上的扩展或\nwd{延拓}.

我们看到, 有时必须研究在某集合 $X$ 的子集 $A$ 上定义的函数 $\varphi: A \rightarrow Y$, 并且函数 $\varphi$ 的值域 $\varphi(A)$ 也可能是 $Y$ 的一个与之不等的子集. 因此, 有时使用术语 “函数的出发域” 来表示包含函数定义域在内的任何一个集合 $X$, 而包含函数值域在内的任何一个集合 $Y$ 则称为 “函数的到达域”.

于是, 为了给出一个函数 (映射), 就要指出它的三要素 $(X, f, Y)$ :
\begin{itemize}
    \item  $X$ 是被映射的集合或函数的定义域,
    \item $Y$ 是映射所到达的集合或函数的到达域,
    \item $f$ 是让每一个元素 $x \in X$ 与确定元素 $y \in Y$ 相对应的规律.
\end{itemize}

我们看到, 这里的 $X$ 与 $Y$ 并不对称, 这表明映射的方向恰恰是从 $X$ 到 $Y$.

\subsection{映射的分类}

当函数 $f: X \rightarrow Y$ 称为映射时, 它在元素 $x \in X$ 上的值$f(x) \in Y$ 通常称为元素 $x$ 的\nwd{像}.
对于映射 $f: X \rightarrow Y$, 集合 $Y$ 中作为集合 $A \subset X$ 中各元素的像的集合
$$
f(A):=\{y \in Y : \exists x((x \in A) \wedge(y=f(x)))\}
$$
称为集合 $A$ 的像,而集合 $X$ 中以集合 $B \subset Y$ 中各元素为像的元素的集合
$$
f^{-1}(B):=\{x \in X : f(x) \in B\}
$$
称为集合 $B$ 的\nwd{原像}.

映射 $f: X \rightarrow Y$ 分为以下几类:
\begin{itemize}
    \item \nwd{满射}, 这时 $f(X)=Y$;
    \item \nwd{单射} (或称为嵌入), 这时对于集合 $X$ 的任何元素 $x_1, x_2$ 有
$$
\left(f\left(x_1\right)=f\left(x_2\right)\right) \Rightarrow\left(x_1=x_2\right),
$$
\item \nwd{双射} (或称为一一映射), 这时它既是满射又是单射.
\end{itemize}

如果映射 $f: X \rightarrow Y$ 是双射, 即如果它给出集合 $X$ 与 $Y$ 的元素之间的一一对应关系, 自然就存在一个映射
$$
f^{-1}: Y \rightarrow X,
$$

其定义方法如下: 如果 $f(x)=y$, 则 $f^{-1}(y)=x$, 即与元素 $y \in Y$ 相对应的是在映射 $f$ 下以 $y$ 为像的元素 $x \in X$. 因为 $f$ 是满射, 所以这样的元素 $x \in X$ 存在, 又因为 $f$ 是单射, 所以该元素是唯一的. 因此, 映射 $f^{-1}$ 的定义是良好的. 这个映射称为原映射 $f$ 的\nwd{逆映射}.

从逆映射的构造方式可以看出, $f^{-1}: Y \rightarrow X$ 本身也是双射, 并且它的逆映射 $\left(f^{-1}\right)^{-1}: X \rightarrow Y$ 就是 $f: X \rightarrow Y$.

因此, 两个映射具有逆映射关系的性质是相互的: 如果 $f^{-1}$ 是 $f$ 的逆映射, 则 $f$ 同样也是 $f^{-1}$ 的逆映射.

我们指出, 尽管集合 $B \subset Y$ 的原像 $f^{-1}(B)$ 与反函数 $f^{-1}$ 共用同样的符号, 但是应该注意, 集合的原像对于任何映射 $f: X \rightarrow Y$ 都有定义, 即使它不是双射, 从而没有逆映射, 原像的定义仍然成立.

\subsection{映射的复合, 互逆映射} 映射的复合运算, 一方面是生成新函数的丰富源泉, 另一方面是把复杂函数分解为简单函数的一种方法. 

如果在映射 $f: X \rightarrow Y$ 和 $g: Y \rightarrow Z$中, 一个映射 (这里是 $g$ ) 定义于另一个映射 $(f)$ 的值域, 就可以构造一个新的映射
$$
g \circ f: X \rightarrow Z,
$$

它在集合 $X$ 的元素上的值由公式
$$
(g \circ f)(x):=g(f(x))
$$
给出. 这样构造出来的映射 $g \circ f$ 称为映射 $f$ 与 $g$ (按这种顺序!) 的\nwd{复合映射}.

鉴于复合运算有时需要连续进行若干次, 指出该运算满足结合律是有益的, 即
$$
h \circ(g \circ f)=(h \circ g) \circ f .
$$
这是因为
\[
    (h \circ(g \circ f))(x)=h((g \circ f)(x))=h(g(f(x)))=(h \circ g)(f(x))=((h \circ g) \circ f)(x).
\]

我们还指出, 即使两种复合 $g \circ f$ 与 $f \circ g$ 都有定义, 它们一般也不相等:
$$
g \circ f \neq f \circ g .
$$

比如, 取二元素集合 $\{a, b\}$ 和映射 $f:\{a, b\} \rightarrow a, g:\{a, b\} \rightarrow b$, 则显然有 $g \circ f:\{a, b\} \rightarrow b$, 同时 $f \circ g:\{a, b\} \rightarrow a$.

使集合 $X$ 的每个元素与自身相对应的映射 $f: X \rightarrow X$, 即映射 $x \stackrel{f}{\longrightarrow} x$, 记作 $e_X$ 并称为集合 $X$ 的\nwd{恒等映射}.

实际上我们看到, 映射 $f: X \rightarrow Y, g: Y \rightarrow X$ 当且仅当 $g \circ f=e_X, f \circ g=e_Y$ 时才是互逆的双射. 

\subsection{作为关系的函数. 函数的图像} 本节最前面对函数概念的描述是一种反映其本质的相当动态的描述, 但从现代标准来看, 还不能称之为定义, 因为它使用了一个与函数等价的概念 --- 对应.为了让读者有所了解, 我们在这里指出用集合论语言给出函数定义的方法. (有趣的是, 我们现在就要介绍的关系的概念, 在莱布尼茨的著作中也出现在函数的概念之前.)

\subsubsection{关系} 

序偶 $(x, y)$ 的任何集合称为\nwd{关系} $\mathcal{R}$. 组成 $\mathcal{R}$ 的所有序偶的第一个元素的集合 $X$ 称为关系 $\mathcal{R}$ 的定义域,而第二个元素的集合 $Y$ 称为关系 $\mathcal{R}$ 的值域.

因此, 可以把关系 $\mathcal{R}$ 解释为直积 $X \times Y$ 的子集 $\mathcal{R}$. 如果 $X \subset X^{\prime}$ 且 $Y \subset Y^{\prime}$,则显然 $\mathcal{R} \subset X \times Y \subset X^{\prime} \times Y^{\prime}$, 所以同一个关系可以作为不同集合的子集给出.

包含某关系的定义域的集合, 称为这个关系的\nwd{出发域}. 包含某关系的值域的集合, 称为这个关系的\nwd{到达域}.
常常把 $(x, y) \in \mathcal{R}$ 写为 $x \mathcal{R} y$, 并说 $x$ 与 $y$ 之间的关系为 $\mathcal{R}$.
如果 $\mathcal{R} \subset X^2$, 就说在 $X$ 上给定了关系 $\mathcal{R}$.

\begin{example}
    设 $X$ 是平面上的直线的集合. 如果直线 $b \in X$ 平行于直线 $a \in X$, 我们就认为这两条直线之间的关系为 $\mathcal{R}$, 记为 $a \mathcal{R} b$. 显然, 这就从 $X^2$ 中划分出满足 $a \mathcal{R} b$ 的序偶 $(a, b)$ 的集合 $\mathcal{R}$. 从几何课程可知, 直线之间的平行关系具有下列性质:

\begin{itemize}
    \item $a \mathcal{R} a$ (\nwd{自反性});
    \item $a \mathcal{R} b \Rightarrow b \mathcal{R} a$ (\nwd{对称性});
    \item $(a \mathcal{R} b) \wedge(b \mathcal{R} c) \Rightarrow a \mathcal{R} c$ (\nwd{传递性}).
\end{itemize}

任何具有上述三个性质的关系 $\mathcal{R}$, 即任何自反的(1)对称的和传递的关系 $\mathcal{R}$,通常称为\nwd{等价关系}. 等价关系由专用符号 $\sim$ 表示, 它这时代替表示关系的字母 $\mathcal{R}$.于是, 对于等价关系, 我们把 $a \mathcal{R} b$ 写为 $a \sim b$, 并说 $a$ 与 $b$ 等价.
    
\end{example}

\begin{example}
    设 $M$ 是某集合, $X=\mathcal{P}(M)$ 是其一切子集的全体. 对于集合 $X=\mathcal{P}(M)$的任意两个元素 $a$ 和 $b$, 即集合 $M$ 的任意两个子集 $a$ 和 $b$, 下列三种可能之一总是成立的: $a$ 包含于 $b ; b$ 包含于 $a ; a$ 不是 $b$ 的子集, $b$ 也不是 $a$ 的子集.
作为 $X^2$ 中的关系 $\mathcal{R}$, 我们来考虑 $X$ 的子集之间的包含关系, 即按照定义令
$$
a \mathcal{R} b:=(a \subset b) \text {. }
$$

这个关系显然具有下列性质:
$$
\begin{aligned}
& a \mathcal{R} a \text { (自反性); } \\
& (a \mathcal{R} b) \wedge(b \mathcal{R} c) \Rightarrow a \mathcal{R} c \text { (传递性); } \\
& (a \mathcal{R} b) \wedge(b \mathcal{R} a) \Rightarrow a=b \text { (\nwd{反对称性}). }
\end{aligned}
$$

如果某集合 $X$ 的任意两个元素之间的关系具有上述三个性质, 则该关系称为集合 $X$ 上的偏序关系. 对于偏序关系, 经常把 $a \mathcal{R} b$ 写为 $a \preccurlyeq b$, 并说 $b$ 在 $a$ 之后.
如果除了偏序关系定义中的三个性质, 还成立条件
$$
\forall a \forall b((a \mathcal{R} b) \vee(b \mathcal{R} a)),
$$

即集合 $X$ 的任何两个元素都是可比的, 则关系 $\mathcal{R}$ 称为\nwd{序关系}, 而定义了序关系的集合 $X$ 称为\nwd{线性序集}.

这个术语的来源与数轴的直观形态有关, 因为数轴上任何一对实数之间的关系都具有 $a \leqslant b$ 的形式.
    
\end{example}

\subsubsection{函数与函数的图像} 满足
$$
\left(x \mathcal{R} y_1\right) \wedge\left(x \mathcal{R} y_2\right) \Rightarrow\left(y_1=y_2\right)
$$
的关系 $\mathcal{R}$ 称为函数关系.
函数关系称为函数.

特别地, 设 $X$ 与 $Y$ 是两个集合 (不一定不同), $\mathcal{R} \subset X \times Y$ 是定义在 $X$.上的关系, 即 $X$ 的元素 $x$ 与 $Y$ 的元素 $y$ 之间的关系. 如果对于任何 $x \in X$, 都存在唯一的元素 $y \in Y$, 使得 $y$ 与 $x$ 满足以上关系, 即 $x \mathcal{R} y$, 则关系 $\mathcal{R}$ 是函数关系.
这样的函数关系 $\mathcal{R} \subset X \times Y$ 也是 $X$ 到 $Y$ 上的映射, 或 $X$ 到 $Y$ 上的函数.
我们常用符号 $f$ 来表示函数. 设 $f$ 是函数, 我们将像前面那样用记号 $y=f(x)$或 $x \stackrel{f}{\longmapsto} y$ 来代替 $x f y$, 并把 $y=f(x)$ 称为函数 $f$ 在元素 $x$ 上的值, 或元素 $x$ 在映射 $f$ 下的像.

我们在最初描述函数的概念时曾经说, 元素 $x \in X$ 按照 “规律” $f$ 与元素 $y \in Y$相对应. 我们看到, 这样的对应就是, 对于每一个元素 $x \in X$, 均可指出唯一的元素 $y \in Y$, 使得 $x f y$, 即 $(x, y) \in f \subset X \times Y$.

对于按照最初描述来理解的函数 $f: X \rightarrow Y$, 由一切形如 $(x, f(x))$ 的元素组成的集合 $\Gamma$ 称为该函数的图像, 它是直积 $X \times Y$ 的子集. 于是,
$$
\Gamma:=\{(x, y) \in X \times Y \mid y=f(x)\} .
$$

在关于函数概念的新描述下,我们用子集 $f \subset X \times Y$ 的形式给出函数, 这时函数与它的图像当然已经没有区别.

% \section{例子} 

% \subsection{集合及其运算} 常见的数集的符号: 

% \begin{tabular}{|c|c|c|}
%     \hline
%     \textbf{数集} & \textbf{符号} & \textbf{描述} \\
%     \hline
%     自然数(Natrual number) & $\mathbb{N}$ & 包括所有正整数\{0, 1, 2, 3, ...\} \\
%     \hline
%     整数(Integer, 德语Zahlen) & $\mathbb{Z}$ & 包括所有正整数、负整数和0 \\
%     \hline
%     有理数(Quotient) & $\mathbb{Q}$ & 包括所有可以表示为两个整数比例的数 \\
%     \hline
%     实数(Rational) & $\mathbb{R}$ & 包括所有的有理数和无理数 \\
%     \hline
%     复数(Complex) & $\mathbb{C}$ & 包括所有形式为 \(a + bi\) 的数,其中 \(a\) 和 \(b\) 是实数 \\
%     \hline
%     \end{tabular}

% \begin{exc}
%     判断: 很小的实数可以构成集合; 集合$\{1,2,3\}$与集合$\{(1,2,3)\}$是同一个集合; 
% \end{exc}

% \begin{exc}
%     填入符号$\in$或者$\notin$: $\sqrt{2-\sqrt{3}}+\sqrt{2+\sqrt{3}} $\underline{\quad}$ \{x : x=a+\sqrt{6} b, a \in \mathbb{Q},  b \in \mathbb{Q}\}$
% \end{exc}

% \begin{exc}
%     设  $S=\{x \mid x=m+\sqrt{2} n, m, n \in \mathbb{Z}\}$

% (1)若  $a \in \mathbb{Z}$ , 则  $a$  是否是集合  $S$  的元素?

% (2)对于  $S$  中任意两个元素  $x_{1} , x_{2} , 则  x_{1}+x_{2} , x_{1} \cdot x_{2}$  是否属于  $S$  ?
% \end{exc}

\end{document}
