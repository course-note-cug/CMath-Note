\documentclass{ctexart}

\usepackage{amsmath, amsthm, amssymb, amsfonts}
\usepackage{thmtools}
\usepackage{graphicx}
\usepackage{setspace}
\usepackage{geometry}

\usepackage{float}
\usepackage{amsthm}
\usepackage{hyperref}
\usepackage{cleveref}
% \usepackage{mathabx}
\usepackage[utf8]{inputenc}
\usepackage[english]{babel}
\usepackage{framed}
\usepackage[dvipsnames]{xcolor}
\usepackage[skins,breakable]{tcolorbox}
\usepackage{awesomebox}
\usepackage{mathrsfs}  
\usepackage{xcolor}
\usepackage{wrapfig}
\usepackage{algorithm2e}
\RestyleAlgo{ruled}

\usepackage{pstricks-add}
\usepackage{epsfig}
\usepackage{pst-grad} % For gradients
\usepackage{pst-plot} % For axes
\usepackage[space]{grffile} % For spaces in paths
\usepackage{etoolbox} % For spaces in paths
\makeatletter % For spaces in paths
\patchcmd\Gread@eps{\@inputcheck#1 }{\@inputcheck"#1"\relax}{}{}
\makeatother

% Make SS at the beginning of a section

\makeatletter
%% See pp. 26f. of 'The LaTeX Companion,' 2nd. ed.
\def\@seccntformat#1{\@ifundefined{#1@cntformat}%
    {\csname the#1\endcsname\quad}%      default
    {\csname #1@cntformat\endcsname}}%   individual control
\newcommand{\section@cntformat}{\S\thesection\quad}
\newcommand{\subsection@cntformat}{\S\thesubsection\quad}
\makeatother % changes @ back to a special character

\usepackage{titlesec}

\CTEXsetup[format={\raggedright\large\bfseries}]{section}
\titleformat{\subsection}[runin]{\normalfont\bfseries}{\thesubsection.}{0.5em}{}[.]
\titleformat{\subsubsection}[runin]{\normalfont\bfseries}{\alph{subsubsection})}{0.5em}{}





\theoremstyle{definition}
\newtheorem{example}{例子}[section]
\newtheorem{definition}{定义}[section]
\newtheorem{theorem}{定理}[section]
\newtheorem{proposition}[theorem]{命题}
\newtheorem{prop}[theorem]{性质}
\newtheorem{corollary}[theorem]{推论}

\newenvironment{remark}{%
  \par\medskip
  \noindent
  \textbf{注:}
}{%
  \par\medskip
}

\newenvironment{solution}{%
  \par\medskip
  \noindent
  \textbf{解答:}
}{%
  \par\medskip
}

\newenvironment{solution*}{%
  \par\medskip
  \noindent 
  \color{gray}\small\textbf{提示或解答:}
}{%
  \par\medskip
}

\newenvironment{definition*}{%
  \par\medskip
  \noindent
  \textbf{定义:}
}{%
  \par\medskip
}

\newenvironment{lemma}{%
  \par\medskip
  \noindent
  \textbf{引理:}
}{%
  \par\medskip
}

\newenvironment{proposition*}{%
  \par\medskip
  \noindent
  \textbf{性质: }
}{%
  \par\medskip
}


\newtcolorbox{asidebox}{
  colback=gray!10,
  colframe=gray!60,
  fonttitle=\bfseries,
  title={Aside},
  breakable=true
}

\newtcolorbox{webaside}{
  colback=cyan!10,
  colframe=cyan!60,
  fonttitle=\bfseries,
  title={Web Demonstrate Aside},
  breakable=true
}

\usepackage{enumitem}

\setlist{nosep}

\setstretch{1.2}
\geometry{
    textheight=9in,
    textwidth=5.5in,
    top=1in,
    headheight=12pt,
    headsep=25pt,
    footskip=30pt
}

\usepackage{environ}
\usepackage[tikz]{bclogo}
\usepackage{tikz}
\usetikzlibrary{calc}
\NewEnviron{takeaway}
  {\par\medskip\noindent
  \begin{tikzpicture}
    \node[inner sep=0pt] (box) {\parbox[t]{.99\textwidth}{%
      \begin{minipage}{.3\textwidth}
      \centering\tikz[scale=5]\node[scale=3,rotate=30]{\bclampe};
      \end{minipage}%
      \begin{minipage}{.65\textwidth}
      \textbf{Takeaway Message}\par\smallskip
      \BODY
      \end{minipage}\hfill}%
    };
    \draw[red!75!black,line width=3pt] 
      ( $ (box.north east) + (-5pt,3pt) $ ) -- ( $ (box.north east) + (0,3pt) $ ) -- ( $ (box.south east) + (0,-3pt) $ ) -- + (-5pt,0);
    \draw[red!75!black,line width=3pt] 
      ( $ (box.north west) + (5pt,3pt) $ ) -- ( $ (box.north west) + (0,3pt) $ ) -- ( $ (box.south west) + (0,-3pt) $ ) -- + (5pt,0);
  \end{tikzpicture}\par\medskip%
}

\usepackage{marginnote}
\renewcommand*{\marginfont}{\color{gray}\ttfamily\small}
\usepackage{setspace}
\newcounter{paranum}[section]
\newcommand{\Par}[1]{\vspace{10pt}\noindent\textbf{\refstepcounter{paranum}\theparanum. }\textbf{#1}~~}
\newcommand{\lec}[1]{\reversemarginpar\marginnote{{\textbf{#1}}}}
\newcommand{\mn}[1]{\marginnote{{#1}}}
\renewcommand{\algorithmcfname}{算法}
\usepackage[abspath]{currfile}

\newcommand{\incfig}[1]{\begin{center}\includegraphics[width=.4\textwidth]{figs/#1}\end{center}}
\newcommand{\incfigw}[1]{\begin{center}\includegraphics[width=.8\textwidth]{figs/#1}\end{center}}
\newcommand{\set}[1]{\{#1\}}
\newcommand{\stirling}[2]{\left\{{#1 \atop #2}\right\}}
\newcommand{\binomt}[2]{\left(\left({#1 \atop #2}\right)\right)}
\newcommand{\pf}[4]{#1_{#2}^{#3_{#4}}}
\newcommand{\pl}[4]{#1_{#2}{#3^{#4}}}
\newcommand{\ty}[3]{{#1} \equiv {#2} ~(\bmod {#3})}
\newcommand{\Z}{{\mathbb Z}}
\newcommand{\one}{\mathbf{1}}
\newcommand{\varsub}[2]{\stackrel{#1}{\stackrel{\rule{#2}{0.4pt}}{\rule{#2}{0.4pt}}}}
\newcommand{\dd}{\mathrm{d}}
\newcommand{\Ep}[1]{\mathbb E\left(#1\right)}

\renewcommand{\red}[1]{{{\color{red}#1}}}
\newcommand{\teal}[1]{{{\color{teal}#1}}}
\renewcommand{\blue}[1]{{{\color{blue}#1}}}
\newcommand{\purple}[1]{{{\color{purple}#1}}}
\DeclareMathOperator{\var}{Var}
\newcommand{\E}{\mathbb E}
\newcommand{\like}{$\blacktriangleright$}
\newcommand{\exrate}[1]{{[#1]~}}
\newcommand{\newword}[2]{{\textbf{#1(#2)}\index{#1}}}
\newcommand{\newenword}[1]{{\textbf{#1}\index{#1}}}

\begin{document}

(写在前面: 本文的 Markdown 版本由计算机程序自动生成并投稿. 因此可能会有错误. 这系列是《具体数学》浅显的阅读笔记. )

\section{和式及其基本操作}

要使得求和便于书写和分析, 我们最好(跟随傅里叶)引入如下的记号:$\sum$.

\subsection{和式与求和记号}

\begin{definition}[求和记号$\sum$]
	对于一个可数的集合$S=\set{a_1, a_2, \cdots, a_n}$, 求和实际上是将这个集合的每一个元素在某一个函数$f:S\to X$作用之后, 把对应的值相加. 记作$\sum_{i \in S} f(i)$; 表示$f\left(a_1\right)+f\left(a_2\right)+f\left(a_3\right)+\cdots+f\left(a_n\right)+\cdots$.
\end{definition}

\begin{example}
	考虑
	$$
		\begin{aligned} & \sum_{\substack{1 \leq k \leq n}} a_k=a_1+a_2+\cdots+a_n . \\ & \sum_{\substack{1 \leq k \leq 100 \\ k \text { odd }}} k^2=\sum_{0 \leq k \leq 49}(2 k+1)^2 .\end{aligned}
	$$
\end{example}

\subsubsection{换元法} 如果将 $\sum_{i \in S} f(i)$ 换为 $\sum_{k \in S} f(k)$ , 求和的表示内容将不变. 这是因为仅仅是``当前 $S$ 中的代表''用来表示的字母不同, 从而肯定不会影响整个映射$f$所表达的意思.

\begin{example}

	考虑
	$$
		\sum_{1 \leq k \leq n} a_k \varsub{k\text{代换为}s+1}{1.5cm} \sum_{1 \leq s+1 \leq n} a_{s+1} \varsub{s\text{代换为}k}{1.5cm} \sum_{1 \leq k+1 \leq n} a_{k+1}
	$$
\end{example}

\begin{remark}
	有时候$\sum_{a \leq i \leq b} f(i)$也写作$\sum_{i=a}^b f(i)$. 但是看到这样一来, 变量代换之后就得到了
	$$
		\sum_{i=1}^n a_i\varsub{k\text{代为}k+1}{1.2cm}\sum_{i=0}^{n-1} a_{i+1}
	$$
	更易于出错.

	此外, 我们用$k:=k+1$来表示把$k$代为$k+1$. 最后, 数列可以看做特殊的函数. 实际上书写$a_k$实际上就相当于$f(k)$.
\end{remark}

\subsubsection{求和的性质}

尽管我们可以对可数无限个元素进行求和操作, 但是, 那样的数列我们在高等数学的级数部分就已经学过了. 这里先讨论有限项求和的情形.

\begin{theorem}
	设$K$是某一有限个正整数的集合, 我们有如下的三条规则:

	\begin{itemize}
		\item 常数项进出求和记号: $$
			      \sum_{k \in k} c f(k)=c \sum_{k \in k} f(k);
		      $$
		\item 求和记号的拆分: $$
			      \sum_{k \in K} f(k) g(k)=\sum_{k \in K} f(k)+\sum_{k \in K} g(k);
		      $$
		\item 若$p(k)$应用于$K$中的每一个元素之后组成的集合依然是$K$的一个排列, 那么$$
			      \sum_{k \in K} f(k)=\sum_{p(k) \in K} f(p(k)).
		      $$
	\end{itemize}
\end{theorem}

上述定理的证明可以直接由定义得到.

\begin{example}
	如$K=\{-1,0,1\}, p(k)=-k$, 由于 $p(-1)=1, p(0)=0$,$p(1)=-1 , p$ 对 $k$ 中每一个元素构成集合为 $\{-1,0,1\}=K$.
\end{example}

\begin{example}[等差数列求和]
	求 $$
		S=\sum_{0 \leq k \leq n}(a+b k)
	$$的值.

	考虑
	$$
		\begin{aligned}
			 & S_2=\sum_{0 \leq k \leq n}(a+b k)                                                           \\
			 & \varsub{k:=n-k}{0.9cm} \sum_{0 \leq n-k \leq n}(a+b(n-k))=\sum_{0 \leq k \leq n}(a+b n-b k)
		\end{aligned}
	$$

	令$S+S_2$ 得

	$$
		\begin{aligned}
			S+S_2=2 S & =\sum_{0 \leq k \leq n}(a+b k)+\sum_{0 \leq k \leq n}(a+b n-b k)                   \\
			          & =\sum_{0 \leq k \leq n}(2 a+b n)=(2 a+b n) \sum_{0 \leq k \leq n} 1=(2 a+b n)(n+1)
		\end{aligned}
	$$

	那 $$
		S=\frac{(n+1)(2 a+b n)}{2}.
	$$

\end{example}

\subsection{Iverson括号}

\begin{definition}[Iverson 括号]
	假设$P$是一个命题, 定义
	$$
		[P]=\left\{\begin{array}{lr}
			1, & \text{命题}P\text{为真} \\
			0, & \text{命题}P\text{为假}
		\end{array}\right.
	$$
\end{definition}

这样的记号便于优化很复杂的求和操作, 并且可以把求和符号的下标转换为命题之间的操作.

\begin{example}
	求和式 $\sum_{0 \leq k \leq n} k$ 可被改为 $\sum_k k[0 \leq k \leq n]$.

	如果 $k$ 未指定限定条件, 我们认为 $k \in \mathbb{Z}$. 也就是说上述式子

	$$
		\begin{aligned}
			\sum_k k[0 \leq k \leq n]= & \cdots+(-3 \cdot 0)+(-2 \cdot 0)+(-1 \cdot 0)+(0 \cdot 0)+(1 \cdot 1)+(2 \cdot 1)+\cdots  +(n \cdot 1) \\
			=                          & (0 \cdot 1)+(1 \cdot 1)+\cdots+(n \cdot 1)
		\end{aligned}
	$$

\end{example}

\begin{example}
	如果$K$与$K'$是两个整数集合, 那么$\forall k$,
	$$
		[k \in K]+\left[k \in K^{\prime}\right]=\left[k \in\left(K \cap K^{\prime}\right)\right]+\left[k \in\left(K \cup K^{\prime}\right)\right]
	$$

	由此可以导出对应的和式
	$$
		\sum_{k \in k} a_k+\sum_{k \in k^{\prime}} a_k=\sum_{k \in k \cap k^{\prime}} a_k+\sum_{k \in k U k^{\prime}} a_k
	$$
\end{example}

这是一个很有用的命题. 下面我们会看到它的用途.

\begin{proposition}
	$[k \in K]+\left[k \in K^{\prime}\right]=\left[k \in\left(K \cap K^{\prime}\right)\right]+\left[k \in\left(K \cup K^{\prime}\right)\right]$对 $k , k^{\prime}$ 为可数集, $\forall k$.
\end{proposition}

\subsection{常见的求和方法}

\subsubsection{成套方法} 这个方法类似于在解答微分方程的时候, 首先求解特解, 然后求解通解. 在这里我们不再赘述.

\subsubsection{扰动法} 要计算$S_n=\sum_{0 \leq k \leq n} a_k$, 可以有两种方法改写

\begin{proposition}
	扰动法是指

	
		\begin{align*}
			\boxed{S_{n}+a_{n+1}}=\sum_{0\leq k\leq n+1}a_{k} & =a_{0}+\sum_{1\leq k\leq n+1}a_{k}                        \\
			                                                  & \varsub{k:=k+1}{0.9cm}a_{0}+\sum_{1\leq k+1\leq n+1}a_{k} \\
			                                                  & \boxed{=a_{0}+\sum_{0\leq k\leq n}a_{k}}
		\end{align*}
	

\end{proposition}

\begin{example}
	用上述的方法计算等比数列. $S_n=\sum_{0 \leq k \leq n} a x^k$.
	$$
		\begin{aligned}
			S_n+a x^{n+1} & =a x^0+\sum_{0 \leq k \leq n} a x^{k+1} \\
			              & =a x^0+x \sum_{0 \leq k \leq n} a x^k   \\
			              & =a x^0+x S_n
		\end{aligned}
	$$

	对于$x\neq 1$, 有
	$$
		S_n=\sum_{k\neq 1}^n a x^k=\frac{a-a x^{n+1}}{1-x}
	$$
\end{example}

\begin{example}[等差数列乘等比数列]
	计算
	$$
		S_n=\sum_{0 \leq k \leq n} k \cdot 2^k.
	$$

	按照上面的方法,
	$$
		\begin{aligned}
			S_n+(n+1) 2^{n+1} & =0+\sum_{0 \leq k \leq n}(k+1) 2^{k+1}                                    \\
			                  & = 0+\sum_{0 \leq k \leq n} k \cdot 2^{k+1}+\sum_{0 \leq k \leq n} 2^{k+1} \\
			                  & =2 S_n+ 2^{n+2}-2
		\end{aligned}
	$$

	解出 $S_n$, 就是

	$$
		S_n=\sum_{0 \leq k \leq n} k 2^k=(n-1) 2^{n+1}+2 \text {. }
	$$
\end{example}

\begin{example}
	从上面的推导中, 我们知道$\sum_{k=0}^n x^k=\frac{1-x^{n-1}}{1-x}$ , 两边对 $x$ 求导, 就有
	$$
		\begin{aligned}
			\sum_{k=0}^n k x^{k-1} & =\frac{(1-x)\left(-(n+1) x^n\right)+1-x^{n+1}}{\left(1-x^2\right)} \\
			                       & =\frac{1-(n+1) x^n+n x^{n+1}}{(1-x)^2} .
		\end{aligned}
	$$

	同样也可以得到上一式子的结果.
\end{example}

\subsubsection{求和因子}

\begin{example}
	由递推关系
	$$
		\begin{aligned}
			 & T_0=0           \\
			 & T_n=2 T_{n-1}+1
		\end{aligned}
	$$
	倘若两端同时除以$2^n$, 就得到
	$$
		\begin{aligned}
			 & T_0 / 2^0=0                         \\
			 & T_n / 2^n=T_{n-1} / 2^{n-1}+1 / 2^n
		\end{aligned}
	$$
	令 $S=T_n / 2^n$ ,得
	$\left\{\begin{aligned} & S_0=0 \\ & S_n=S_{n-1}+2^{-n}\end{aligned}\right.$, 即 $S_n=\sum_{k=1}^n 2^{-k}$ , 为一等比数列.
\end{example}

对于更为一般的式子, 如 $a_n T_n=b_n T_{n-1}+c_n$, 可变为 $S_n=S_{n-1}+ S_n c_n$的形式.

$1^\circ$ 方法: 使两边同时乘以求和因子 $S_n$:

\[
	\underbrace{\boxed{s_{n}a_{n}T_{n}}}_{:=S_{n}}=\underbrace{S_{n}b_{n}T_{n-1}}_{\text{必须得是}S_{n-1}=S_{n-1}a_{n-1}T_{n-1}}+S_{n}c_{n}
\]
由此 $S_n=S_0 a_0 \cdot T_0+\sum_{k=1}^n S_k c_k=S_1 b_1 T_0+\sum_{k=1}^n S_k c_k$ , 那么$$
	T_n=\frac{1}{S_n a_n}\left(S_1 b_1 T_0+\sum_{k=1}^n S_k c_k\right)
$$.

$2^\circ$ 寻找$S_n$的方法: 由于上式要满足$S_n b_n T_{n-1}=S_{n-1} a_{n-1} T_{n-1}$, 代入展开, 有

$$
	\begin{aligned}
		s_n & =\frac{s_{n-1} a_{n-1}}{b_n}                                 \\
		    & =\frac{s_{n-2} a_{n-1} a_{n-2}}{b_n b_{n-1}}=\cdots          \\
		    & =\frac{a_{n-1} a_{n-2} \cdots a_1}{b_n b_{n-1} \cdots b_2} .
	\end{aligned}
$$

因此我们就这样找到了求和因子.

\begin{example}
	解由快速排序带来的递归式:

	$$
		\begin{aligned}
			 & C_0=0                                                 \\
			 & C_n=n+1+\frac{2}{n} \sum_{k=0}^{n-1} C_k \quad, n>0 .
		\end{aligned}
	$$

	将 $C_n$ 两侧同乘以 $n$, 得
	$$
		n C_n=n^2+n+2 \sum_{k=0}^{n-1} C_k \quad(n>0).
	$$

	令 $n:=n-1$, 有 $$
		(n-1) C_{n-1}=(n-1)^2+(n-1)+2 \sum_{k=0}^{n-2} c_k.
	$$

	上面两式相减, 有$n C_n-(n-1) C_{n-1}=2 n+2 C_{n-1}$, 即

	$$
		\begin{aligned}
			C_0   & =0                   \\
			n C_n & =(n+1) C_{n-1}+2 n .
		\end{aligned}
	$$

	将上述 $a_n=n, b_n=n+1, c_n=2 n$, 得 $s=\frac{(n-1) \cdots 1}{(n+1) \cdots 3}=\frac{2}{n(n+1)}$.

	得
	$$
		C_n=2(n+1) \sum_{k=1}^n \frac{1}{k+1}=2(n+1) H_{n+1}.
	$$

\end{example}

\end{document}