%!TEX TS-program = xelatex
\documentclass{ctexart}

\usepackage{amsmath, amsthm, amssymb, amsfonts}
\usepackage{thmtools}
\usepackage{graphicx}
\usepackage{setspace}
\usepackage{geometry}

\usepackage{float}
\usepackage{amsthm}
\usepackage{hyperref}
\usepackage{cleveref}
% \usepackage{mathabx}
\usepackage[utf8]{inputenc}
\usepackage[english]{babel}
\usepackage{framed}
\usepackage[dvipsnames]{xcolor}
\usepackage[skins,breakable]{tcolorbox}
\usepackage{awesomebox}
\usepackage{mathrsfs}  
\usepackage{xcolor}
\usepackage{wrapfig}
\usepackage{algorithm2e}
\RestyleAlgo{ruled}

\usepackage{pstricks-add}
\usepackage{epsfig}
\usepackage{pst-grad} % For gradients
\usepackage{pst-plot} % For axes
\usepackage[space]{grffile} % For spaces in paths
\usepackage{etoolbox} % For spaces in paths
\makeatletter % For spaces in paths
\patchcmd\Gread@eps{\@inputcheck#1 }{\@inputcheck"#1"\relax}{}{}
\makeatother

% Make SS at the beginning of a section

\makeatletter
%% See pp. 26f. of 'The LaTeX Companion,' 2nd. ed.
\def\@seccntformat#1{\@ifundefined{#1@cntformat}%
    {\csname the#1\endcsname\quad}%      default
    {\csname #1@cntformat\endcsname}}%   individual control
\newcommand{\section@cntformat}{\S\thesection\quad}
\newcommand{\subsection@cntformat}{\S\thesubsection\quad}
\makeatother % changes @ back to a special character

\usepackage{titlesec}

\CTEXsetup[format={\raggedright\large\bfseries}]{section}
\titleformat{\subsection}[runin]{\normalfont\bfseries}{\thesubsection.}{0.5em}{}[.]
\titleformat{\subsubsection}[runin]{\normalfont\bfseries}{\alph{subsubsection})}{0.5em}{}





\theoremstyle{definition}
\newtheorem{example}{例子}[section]
\newtheorem{definition}{定义}[section]
\newtheorem{theorem}{定理}[section]
\newtheorem{proposition}[theorem]{命题}
\newtheorem{prop}[theorem]{性质}
\newtheorem{corollary}[theorem]{推论}

\newenvironment{remark}{%
  \par\medskip
  \noindent
  \textbf{注:}
}{%
  \par\medskip
}

\newenvironment{solution}{%
  \par\medskip
  \noindent
  \textbf{解答:}
}{%
  \par\medskip
}

\newenvironment{solution*}{%
  \par\medskip
  \noindent 
  \color{gray}\small\textbf{提示或解答:}
}{%
  \par\medskip
}

\newenvironment{definition*}{%
  \par\medskip
  \noindent
  \textbf{定义:}
}{%
  \par\medskip
}

\newenvironment{lemma}{%
  \par\medskip
  \noindent
  \textbf{引理:}
}{%
  \par\medskip
}

\newenvironment{proposition*}{%
  \par\medskip
  \noindent
  \textbf{性质: }
}{%
  \par\medskip
}


\newtcolorbox{asidebox}{
  colback=gray!10,
  colframe=gray!60,
  fonttitle=\bfseries,
  title={Aside},
  breakable=true
}

\newtcolorbox{webaside}{
  colback=cyan!10,
  colframe=cyan!60,
  fonttitle=\bfseries,
  title={Web Demonstrate Aside},
  breakable=true
}

\usepackage{enumitem}

\setlist{nosep}

\setstretch{1.2}
\geometry{
    textheight=9in,
    textwidth=5.5in,
    top=1in,
    headheight=12pt,
    headsep=25pt,
    footskip=30pt
}

\usepackage{environ}
\usepackage[tikz]{bclogo}
\usepackage{tikz}
\usetikzlibrary{calc}
\NewEnviron{takeaway}
  {\par\medskip\noindent
  \begin{tikzpicture}
    \node[inner sep=0pt] (box) {\parbox[t]{.99\textwidth}{%
      \begin{minipage}{.3\textwidth}
      \centering\tikz[scale=5]\node[scale=3,rotate=30]{\bclampe};
      \end{minipage}%
      \begin{minipage}{.65\textwidth}
      \textbf{Takeaway Message}\par\smallskip
      \BODY
      \end{minipage}\hfill}%
    };
    \draw[red!75!black,line width=3pt] 
      ( $ (box.north east) + (-5pt,3pt) $ ) -- ( $ (box.north east) + (0,3pt) $ ) -- ( $ (box.south east) + (0,-3pt) $ ) -- + (-5pt,0);
    \draw[red!75!black,line width=3pt] 
      ( $ (box.north west) + (5pt,3pt) $ ) -- ( $ (box.north west) + (0,3pt) $ ) -- ( $ (box.south west) + (0,-3pt) $ ) -- + (5pt,0);
  \end{tikzpicture}\par\medskip%
}

\usepackage{marginnote}
\renewcommand*{\marginfont}{\color{gray}\ttfamily\small}
\usepackage{setspace}
\newcounter{paranum}[section]
\newcommand{\Par}[1]{\vspace{10pt}\noindent\textbf{\refstepcounter{paranum}\theparanum. }\textbf{#1}~~}
\newcommand{\lec}[1]{\reversemarginpar\marginnote{{\textbf{#1}}}}
\newcommand{\mn}[1]{\marginnote{{#1}}}
\renewcommand{\algorithmcfname}{算法}
\usepackage[abspath]{currfile}

\newcommand{\incfig}[1]{\begin{center}\includegraphics[width=.4\textwidth]{figs/#1}\end{center}}
\newcommand{\incfigw}[1]{\begin{center}\includegraphics[width=.8\textwidth]{figs/#1}\end{center}}
\newcommand{\set}[1]{\{#1\}}
\newcommand{\stirling}[2]{\left\{{#1 \atop #2}\right\}}
\newcommand{\binomt}[2]{\left(\left({#1 \atop #2}\right)\right)}
\newcommand{\pf}[4]{#1_{#2}^{#3_{#4}}}
\newcommand{\pl}[4]{#1_{#2}{#3^{#4}}}
\newcommand{\ty}[3]{{#1} \equiv {#2} ~(\bmod {#3})}
\newcommand{\Z}{{\mathbb Z}}
\newcommand{\one}{\mathbf{1}}
\newcommand{\varsub}[2]{\stackrel{#1}{\stackrel{\rule{#2}{0.4pt}}{\rule{#2}{0.4pt}}}}
\newcommand{\dd}{\mathrm{d}}
\newcommand{\Ep}[1]{\mathbb E\left(#1\right)}

\renewcommand{\red}[1]{{{\color{red}#1}}}
\newcommand{\teal}[1]{{{\color{teal}#1}}}
\renewcommand{\blue}[1]{{{\color{blue}#1}}}
\newcommand{\purple}[1]{{{\color{purple}#1}}}
\DeclareMathOperator{\var}{Var}
\newcommand{\E}{\mathbb E}
\newcommand{\like}{$\blacktriangleright$}
\newcommand{\exrate}[1]{{[#1]~}}
\newcommand{\newword}[2]{{\textbf{#1(#2)}\index{#1}}}
\newcommand{\newenword}[1]{{\textbf{#1}\index{#1}}}


\begin{document}

\section{词法分析}

\subsection{语言}

语言是字符串的集合. 

\begin{definition}[字母表]
    字母表 $\Sigma$ 是一个有限的符号集合. 
\end{definition}

现在, 符号没有任何的含义. 符号是什么意思是语义干的事情. 

\begin{definition}[字母表]
    字母表 $\Sigma$ 上的串 $(s)$ 是由 $\Sigma$ 中符号构成的一个有穷序列.
\end{definition}

其中特殊的串是空串, $|\epsilon|=0$

\begin{definition}[串上的连接运算]
    例如$x=\operatorname{dog}, y=$ house $\quad x y=$ doghouse, $s \epsilon=\epsilon S=s$
\end{definition}

\begin{definition}[串上的指数运算]
    $$
\begin{aligned}
&s^0 \triangleq \epsilon\\
&s^i \triangleq s s^{i-1}, i>0
\end{aligned}
$$
\end{definition}

\begin{definition}
    语言是给定字母表 $\Sigma$ 上一个任意的可数的串集合。
\end{definition}

\begin{remark}
    \begin{enumerate}
        \item 注意$\emptyset$和$\{\epsilon\}$. 后者是只有空串的语言. 
        \item 例如ws : \{blank, tab, newline $\}$
        \item 这就可以通过集合操作构造新的语言.
    \end{enumerate}
    
\end{remark}

\begin{definition}
    我们可以通过如下的方式构造新语言.

    $$
\begin{array}{|c|c|}
\hline \text { 运算 } & \text { 定义和表示 } \\
\hline L \text { 和 } M \text { 的并 } & L \cup M :=\{s \mid s \text { 属于 } L \text { 或者 } s \text { 属于 } M\} \\
\hline L \text { 和 } M \text { 的连接 } & L M:=\{s t \mid s \text { 属于 } L \text { 且 } t \text { 属于 } M\} \\
\hline L \text { 的 Kleene 闭包 } & L^*:=\cup_{i=0}^{\infty} L^i \\
\hline L \text { 的正闭包 } & L^{+}:=\cup_{i=1}^{\infty} L^i \\
\hline
\end{array}
$$

\end{definition}



\begin{remark}
    $L^*\left(L^{+}\right)$允许我们构造无穷集合
\end{remark}

\begin{example}
    假设$L=\{A, B, \ldots, Z, a, b, \ldots, z\} , D=\{0,1, \ldots, 9\}$. 那么
    \begin{itemize}
        \item $L \cup D$ = 所有的大小写字母和所有的一位数码构成的集合.
        \item $|L D|=52\times 10=520$.
        \item $L^4$ = 所有长度为4的由字母构成的语言的集合.
        \item $L^*$=所有字母构成的字符串(包含空串)构成的集合. 
        \item $D^+$ = 所有数字构成的字符串(不包含空串)构成的集合.
        \item $L(L \cup D)^*$ = 以字母开头, 跟上0个或者若干个字母或者数字的字符串构成的集合. (也就是类似于id). 
    \end{itemize}
\end{example}

\subsection{正则表达式} 注意区分语法和语义. 例如正则表达式中, 每个正则表达式 $r$ 对应一个正则语言 $L(r)$. 正则表达式是语法, 正则语言是语义. 

\begin{definition}[正则表达式]
    给定字母表 $\Sigma, \Sigma$ 上的正则表达式由且仅由以下规则定义:
\begin{enumerate}
    \item $\epsilon$ 是正则表达式;
    \item $\forall a \in \Sigma, a$ 是正则表达式;
    \item 如果 $r$ 是正则表达式, 则 $(r)$ 是正则表达式;
    \item 如果 $r$ 与 $s$ 是正则表达式, 则 $\red{r \mid s, r s, r^*}$ 也是正则表达式.
\end{enumerate}
运算优先级: ()$\blue{\succ} * \blue{\succ}$ 连接 $\blue{\succ} |$
\end{definition}

例如
$$
(a)\left|\left((b)^*(c)\right) \equiv a\right| b^* c.
$$

每个正则表达式 $r$ 对应一个正则语言 $L(r)$, 这代表了它的语义. 

\begin{definition}[正则表达式对应的正则语言]
$$
\begin{aligned}
L(\epsilon)&=\{\epsilon\} \\
L(a)&=\{a\}, \forall a \in \Sigma \\
L((r))&=L(r) \\
L(r \mid s)&=L(r) \cup L(s) \quad L(r s)=L(r) L(s) \quad L\left(r^*\right)=(L(r))^*
\end{aligned}
$$
\end{definition}

\begin{example}
    如果$\Sigma=\{a, b\}$, $L(a \mid b)=\{a, b\}$. 那么
    \begin{itemize}
        \item $L\left(a^*\right)=\{\epsilon, a, aa, aaa, \cdots, \}$.
        \item $L\left((a \mid b)^*\right)$=由$a$和$b$构成的任意长度的字符串(包括空串). 
        \item $ L\left(a \mid a^* b\right)=\{a, b, a b, a a b,  aaab, \cdots\} $. 
    \end{itemize}
\end{example}

实际生活中, 我们的正则表达式的列表如下: 

\begin{tabular}{|c|c|c|}
    \hline 表达式 & 匹配 & 例子 \\
    \hline $c$ & 单个非运算符字符 $c$ & $ a  $ \\
    \hline$\backslash c$ & 字符 $c$ 的字面值 & $\backslash *$ \\
    \hline$" s$ & 串 $s$ 的字面值 & $" ** "$ \\
    \hline . & 除换行符以外的任何字符(看环境, ANTLR4中可以匹配换行符) & $a.*b$ \\
    \hline$\hat{ }$ & 一行的开始 & $\hat{ }abc$ \\
    \hline$\$$ & 行的结尾 & $a b c \$$ \\
    \hline$[s]$ & 字符串 $s$ 中的任何一个字符 & {$[\mathrm{abc}]$} \\
    \hline$[\hat{~}{s}]$ & 不在串 $s$ 中的任何一个字符 & {$\hat{~}[a b c]$} \\
    \hline$r^*$ & 和 $r$ 匹配的零个或多个串连接成的串 & $a *$ \\
    \hline$r+$ & 和 $r$ 匹配的一个或多个串连接成的串 & $a+$ \\
    \hline$r ?$ & 零个或一个 $r$ & $a$ ? \\
    \hline$r\{m, n\}$ & 最少 $m$ 个, 最多 $n$ 个 $r$ 的重复出现 & $a\{1,5\}$ \\
    \hline$r_1 r_2$ & $r_1$ 后加上 $r_2$ & $a b$ \\
    \hline$r_1 \mid r_2$ & $r_1$ 或 $r_2$ & $a \mid b$ \\
    \hline$(r)$ & 与 $r$ 相同 & $(a \mid b)$ \\
    \hline$r_1 / r_2$ & 后面跟有 $r_2$ 时的 $r_1$ & $abc / 123$ \\
    \hline
    \end{tabular}

    有一些简单记录方法(在Vim中)
    \begin{itemize}
        \item $\backslash w$表示所有大小写字母, 数字, 以及下划线
        \item $\backslash W$表示\red{除去}所有大小写字母, 数字, 以及下划线
        \item $\backslash d$表示所有数码 
        \item $\backslash D$表示\red{除去}所有数码 
    \end{itemize}

    \begin{example}
        $$
\left(0 \mid\left(1\left(01^* 0\right)^* 1\right)\right)^*
$$
表示的二进制的3的倍数. 可以从\url{regex.com}中观察一些例子. 
    \end{example}

    接下来介绍自动机. 

    \begin{definition}[NFA]
        非确定性有穷自动机 $\mathcal{A}$ 是一个五元组 $\mathcal{A}=\left(\Sigma, S, s_0, \delta, F\right)$ :
        \begin{enumerate}
            \item 字母表 $\Sigma(\epsilon \notin \Sigma)$;
            \item 有穷的状态集合 $S$;
            \item (唯一)的初始状态 $s_0 \in S$;
            \item 状态转移函数 $\delta$.$$
            \delta: S \times(\Sigma \cup\{\epsilon\}) \rightarrow 2^S
            $$
            \item 接受状态集合 $F \subseteq S$
        \end{enumerate}

    \end{definition}

    \begin{remark}
        \begin{enumerate}
            \item 非确定性指的是其``出路''可能不唯一. 此外, 可以接受空串然后进行转移(称为$\epsilon$转移). 
            \item 所有没有对应出边的字符默认指向“空状态” $\emptyset$. 此状态无论接受什么字符都到自身. 一旦进入将无法出去.
        \end{enumerate}
    \end{remark}

    \incfigw{automata-eg}

    下面来看NFA的语义. (非确定性) 有穷自动机是一类极其简单的计算装置, 它可以识别 (接受/拒绝) $\Sigma$ 上的字符串. 

    \begin{definition}[接受(Accept)]
        (非确定性) 有穷自动机 $\mathcal{A}$ 接受字符串 $x$, 当且仅当存在一条从开始状态 $s_0$ 到某个接受状态 $f \in F$ 、标号为 $x$ 的路径.
    \end{definition}

    \begin{example}考虑下面的自动机: 

        \incfig{auto-another}

        对于上面的一个状态图, $L(\mathcal{A})=L\left(\left(a a^* \mid b b^*\right)\right)$. 
    \end{example}

    \begin{example}
        考虑下面的自动机: 

        \incfig{aufofig13}

        实际上$L(\mathcal{A})=\{$ 包含偶数个 1 或偶数个 0 的 01 串 $\}$. 

    \end{example}
    

    因此, $\mathcal{A}$ 定义了一种语言 $L(\mathcal{A})$ : 它能接受的所有字符串构成的集合.

    关于自动机 $\mathcal{A}$ 的两个基本问题:
    \begin{enumerate}
        \item Membership 问题: 给定字符串 $x, x \in L(\mathcal{A})$ ?
        \item $L(\mathcal{A})$ 究竟是什么?
    \end{enumerate}

    \begin{definition}[DFA]
        确定性有穷自动机 $\mathcal{A}$ 是一个五元组 $\mathcal{A}=\left(\Sigma, S, s_0, \delta, F\right)$ :
        \begin{itemize}
            \item 字母表 $\Sigma(\epsilon \notin \Sigma)$
            \item 有穷的状态集合 $S$
            \item 唯一的初始状态 $s_0 \in S$
            \item 状态转移函数 $\delta$ 
            $$
            \delta: S \times \Sigma \rightarrow S
            $$
            \item 接受状态集合 $F \subseteq S$
        \end{itemize} 
        
    \end{definition}

    上述约定(所有没有对应出边的字符默认指向一个 “死状态”)同样适用于这里. 

    NFA 简洁易于理解, 便于描述语言 $L(\mathcal{A})$ DFA 易于判断 $x \in L(\mathcal{A})$, 适合产生词法分析器. 下面我们来走$\mathrm{RE} \Longrightarrow \mathrm{NFA} \Longrightarrow \mathrm{DFA} \Longrightarrow$ 词法分析器的流程. 

    \subsubsection{正则表达式到NFA}

    要求$L(N(r))=L(r)$. 可以使用Tompson构造法. 使用四种基础的归纳. 

    \incfigw{tompson}
        
    $N(r)$ 的性质以及 Thompson 构造法复杂度分析
\begin{enumerate}
    \item  $N(r)$ 的开始状态与接受状态均唯一。
    \item  开始状态没有人边, 接受状态没有出边。
    \item  $N(r)$ 的状态数 $|S| \leq 2 \times|r|$ 。
    \item  每个状态最多有两个 $\epsilon$ - 入边与两个 $\epsilon$-出边。
    \item  $\forall a \in \Sigma$, 每个状态最多有一个 $a$-人边与一个 $a$-出边。
    $(|r|: r$ 中运算符与运算分量的总和)
\end{enumerate}

\subsubsection{从NFA到DFA的转换} 主要思想: 使用DFA模拟NFA. 



\end{document}