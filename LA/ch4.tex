%!TEX TS-program = xelatex
\documentclass{ctexart}

\usepackage{amsmath, amsthm, amssymb, amsfonts}
\usepackage{thmtools}
\usepackage{graphicx}
\usepackage{setspace}
\usepackage{geometry}

\usepackage{float}
\usepackage{amsthm}
\usepackage{hyperref}
\usepackage{cleveref}
% \usepackage{mathabx}
\usepackage[utf8]{inputenc}
\usepackage[english]{babel}
\usepackage{framed}
\usepackage[dvipsnames]{xcolor}
\usepackage[skins,breakable]{tcolorbox}
\usepackage{awesomebox}
\usepackage{mathrsfs}  
\usepackage{xcolor}
\usepackage{wrapfig}
\usepackage{algorithm2e}
\RestyleAlgo{ruled}

\usepackage{pstricks-add}
\usepackage{epsfig}
\usepackage{pst-grad} % For gradients
\usepackage{pst-plot} % For axes
\usepackage[space]{grffile} % For spaces in paths
\usepackage{etoolbox} % For spaces in paths
\makeatletter % For spaces in paths
\patchcmd\Gread@eps{\@inputcheck#1 }{\@inputcheck"#1"\relax}{}{}
\makeatother

% Make SS at the beginning of a section

\makeatletter
%% See pp. 26f. of 'The LaTeX Companion,' 2nd. ed.
\def\@seccntformat#1{\@ifundefined{#1@cntformat}%
    {\csname the#1\endcsname\quad}%      default
    {\csname #1@cntformat\endcsname}}%   individual control
\newcommand{\section@cntformat}{\S\thesection\quad}
\newcommand{\subsection@cntformat}{\S\thesubsection\quad}
\makeatother % changes @ back to a special character

\usepackage{titlesec}

\CTEXsetup[format={\raggedright\large\bfseries}]{section}
\titleformat{\subsection}[runin]{\normalfont\bfseries}{\thesubsection.}{0.5em}{}[.]
\titleformat{\subsubsection}[runin]{\normalfont\bfseries}{\alph{subsubsection})}{0.5em}{}





\theoremstyle{definition}
\newtheorem{example}{例子}[section]
\newtheorem{definition}{定义}[section]
\newtheorem{theorem}{定理}[section]
\newtheorem{proposition}[theorem]{命题}
\newtheorem{prop}[theorem]{性质}
\newtheorem{corollary}[theorem]{推论}

\newenvironment{remark}{%
  \par\medskip
  \noindent
  \textbf{注:}
}{%
  \par\medskip
}

\newenvironment{solution}{%
  \par\medskip
  \noindent
  \textbf{解答:}
}{%
  \par\medskip
}

\newenvironment{solution*}{%
  \par\medskip
  \noindent 
  \color{gray}\small\textbf{提示或解答:}
}{%
  \par\medskip
}

\newenvironment{definition*}{%
  \par\medskip
  \noindent
  \textbf{定义:}
}{%
  \par\medskip
}

\newenvironment{lemma}{%
  \par\medskip
  \noindent
  \textbf{引理:}
}{%
  \par\medskip
}

\newenvironment{proposition*}{%
  \par\medskip
  \noindent
  \textbf{性质: }
}{%
  \par\medskip
}


\newtcolorbox{asidebox}{
  colback=gray!10,
  colframe=gray!60,
  fonttitle=\bfseries,
  title={Aside},
  breakable=true
}

\newtcolorbox{webaside}{
  colback=cyan!10,
  colframe=cyan!60,
  fonttitle=\bfseries,
  title={Web Demonstrate Aside},
  breakable=true
}

\usepackage{enumitem}

\setlist{nosep}

\setstretch{1.2}
\geometry{
    textheight=9in,
    textwidth=5.5in,
    top=1in,
    headheight=12pt,
    headsep=25pt,
    footskip=30pt
}

\usepackage{environ}
\usepackage[tikz]{bclogo}
\usepackage{tikz}
\usetikzlibrary{calc}
\NewEnviron{takeaway}
  {\par\medskip\noindent
  \begin{tikzpicture}
    \node[inner sep=0pt] (box) {\parbox[t]{.99\textwidth}{%
      \begin{minipage}{.3\textwidth}
      \centering\tikz[scale=5]\node[scale=3,rotate=30]{\bclampe};
      \end{minipage}%
      \begin{minipage}{.65\textwidth}
      \textbf{Takeaway Message}\par\smallskip
      \BODY
      \end{minipage}\hfill}%
    };
    \draw[red!75!black,line width=3pt] 
      ( $ (box.north east) + (-5pt,3pt) $ ) -- ( $ (box.north east) + (0,3pt) $ ) -- ( $ (box.south east) + (0,-3pt) $ ) -- + (-5pt,0);
    \draw[red!75!black,line width=3pt] 
      ( $ (box.north west) + (5pt,3pt) $ ) -- ( $ (box.north west) + (0,3pt) $ ) -- ( $ (box.south west) + (0,-3pt) $ ) -- + (5pt,0);
  \end{tikzpicture}\par\medskip%
}

\usepackage{marginnote}
\renewcommand*{\marginfont}{\color{gray}\ttfamily\small}
\usepackage{setspace}
\newcounter{paranum}[section]
\newcommand{\Par}[1]{\vspace{10pt}\noindent\textbf{\refstepcounter{paranum}\theparanum. }\textbf{#1}~~}
\newcommand{\lec}[1]{\reversemarginpar\marginnote{{\textbf{#1}}}}
\newcommand{\mn}[1]{\marginnote{{#1}}}
\renewcommand{\algorithmcfname}{算法}
\usepackage[abspath]{currfile}

\newcommand{\incfig}[1]{\begin{center}\includegraphics[width=.4\textwidth]{figs/#1}\end{center}}
\newcommand{\incfigw}[1]{\begin{center}\includegraphics[width=.8\textwidth]{figs/#1}\end{center}}
\newcommand{\set}[1]{\{#1\}}
\newcommand{\stirling}[2]{\left\{{#1 \atop #2}\right\}}
\newcommand{\binomt}[2]{\left(\left({#1 \atop #2}\right)\right)}
\newcommand{\pf}[4]{#1_{#2}^{#3_{#4}}}
\newcommand{\pl}[4]{#1_{#2}{#3^{#4}}}
\newcommand{\ty}[3]{{#1} \equiv {#2} ~(\bmod {#3})}
\newcommand{\Z}{{\mathbb Z}}
\newcommand{\one}{\mathbf{1}}
\newcommand{\varsub}[2]{\stackrel{#1}{\stackrel{\rule{#2}{0.4pt}}{\rule{#2}{0.4pt}}}}
\newcommand{\dd}{\mathrm{d}}
\newcommand{\Ep}[1]{\mathbb E\left(#1\right)}

\renewcommand{\red}[1]{{{\color{red}#1}}}
\newcommand{\teal}[1]{{{\color{teal}#1}}}
\renewcommand{\blue}[1]{{{\color{blue}#1}}}
\newcommand{\purple}[1]{{{\color{purple}#1}}}
\DeclareMathOperator{\var}{Var}
\newcommand{\E}{\mathbb E}
\newcommand{\like}{$\blacktriangleright$}
\newcommand{\exrate}[1]{{[#1]~}}
\newcommand{\newword}[2]{{\textbf{#1(#2)}\index{#1}}}
\newcommand{\newenword}[1]{{\textbf{#1}\index{#1}}}

\begin{document}
\section{线性空间的定义}

\begin{definition}[线性空间]
    设$P$是一个数域, $V$是一个非空集合, 对$V$中的任意两个元素$\alpha,\beta$, 有唯一的元素与之对应, 记作$\alpha$与$\beta$的和. 我们称在$V$中定义了加法``+''. 

    又对于$P$中任意一个数$k$与$V$中的任意一个元素$\alpha$, 有唯一的$V$中的元素与他们对应. 我们称为$k$与$\alpha$的积. 记作$k\alpha$. 我们称在$V$中定义了数量积(纯量积). 

    并且这两种运算满足如下的八条性质: 

    \begin{enumerate}
        \item 加法部分
        \begin{enumerate}
            \item 加法交换律: $\forall \alpha, \beta \in V. \alpha+\beta=\beta+\alpha$.
            \item 加法结合律: $\forall \alpha,\beta,\gamma \in V. (\alpha+\beta)+\gamma=\alpha+(\beta+\gamma)$
            \item 存在零元: $\exists 0\in V. 0+\alpha=\alpha, \forall \alpha\in V$. 
            \item 存在负元: $\forall \alpha\in V. \exists -\alpha\in V. \alpha+(-\alpha)=0$.
        \end{enumerate}
    \item 乘法部分
        \begin{enumerate}
            \item 存在单位元: $1\cdot \alpha=\alpha, \forall \alpha\in V$. 
            \item 数乘结合律: $k(l\alpha)=(kl)\alpha$
        \end{enumerate}
    \item 加法与数乘
        \begin{enumerate}
            \item $(k+l)\alpha=k\alpha+l\alpha, \forall k,l \in P, \alpha\in V$
            \item $k(\alpha+\beta)=k\alpha+k\beta, \forall k\in P, \alpha,\beta\in V$.
        \end{enumerate}
    \end{enumerate}
    则称$V$是属于$P$上的线性空间(向量空间), $V$中的元素称为向量, $P$为$V$的基域.  
\end{definition}

接下来考察线性空间的常见性质. 

\begin{prop}
    \begin{enumerate} 对于这个定义, 立即有以下的性质. 
        \item 零元是唯一的. 
\item 对于$\alpha\in V$, 负元唯一. 
\item 有消去律, 也就是如果$\alpha, \beta, \gamma\in V$, $\alpha+\beta=\alpha+\gamma\implies \beta=\gamma$.
\item $\forall k\in P, k\cdot 0=0; \forall \alpha\in V, 0\cdot \alpha=0, (-1)\cdot \alpha=-\alpha$. 
    \end{enumerate}
    \end{prop}
\begin{proof}
    1. 设若有$0' \in V$, 使得$0'+\alpha=\alpha, \forall \alpha\in V$, 取$\alpha=0$, 那就是说$0'=0'+0=0+0'=0$. 因此$0$和$0'$实际上是一个东西. 

    2. 若$\beta\in V$, 取$\alpha$使得$\alpha+\beta=0$, 则
    \[
        \beta=0+\beta=(\alpha+(-\alpha))+\beta=(-\alpha)+(\alpha+\beta)=-\alpha.
    \]

    3. 可以通过如下证明
    \[
        \beta=0+\beta=-\alpha+\alpha+\beta=-\alpha+\alpha+\gamma=\gamma .
    \]

    4. 使用分配率, 有$0+k \cdot 0=k \cdot 0=k \cdot(0+0)=k \cdot 0+k \cdot 0$. 这就意味着
    $k \cdot 0=0$. 

    另一方面, $0+0 \cdot \alpha=0 \cdot \alpha=(0+0) \alpha=0 \cdot \alpha+0 \cdot \alpha$. 因此得到$0 \cdot \alpha=0$. 

    由于$0\cdot \alpha=0$知$\alpha+(-1) \alpha=(1+(-1)) \alpha=0=\alpha+(-\alpha)$. 这就说明$(-1)\alpha=-\alpha$. 


    5. 如果$k\neq 0, \alpha=(k \frac 1k)\alpha=\frac 1k (k\alpha)=\frac 1k \cdot 0=0$.另一方面类似可以证明.
    
\end{proof}

\begin{example}
    \begin{enumerate}
        \item 行向量空间$[x_1, x_2, \cdots, x_n]$和列向量空间$[x_1, x_2, \cdots, x_n]'$. 
        \item $P$上的一元多项式$P[x]$对多项式和多项式的加法, 多项式和数的乘法封闭. 构成$P$上的线性空间. 
        \item 设$V$是所有收敛实数序列的集合 $V=\left\{\alpha=\left(a_1, a_2, \cdots, a_n, \cdots\right):  \underset{n \rightarrow \infty}\lim a_n \text { 存在 }\right\}$, 则对数列的加法, 数列与实数的乘法封闭. 也就是$V$构成$\R$上的线性空间. 
    \end{enumerate}
    
\end{example}

我们在给出一个定义之后, 自然要考察其初步的性质, 与它关联的结构, 以及结构与结构之间有何种联系. 下面我们来看这个结构和它满足某种条件的子集有什么联系. 我们称为子空间. 

设$W$是$\mathbb P$上的线性空间的子集, 如果对于任意$\alpha, \beta, $都有$\alpha+\beta \in W$($W$对加法封闭), 以及$\forall k \in \mathbb{P}, \alpha \in W$, 有$k {\alpha} \in W$($W$对纯量乘法封闭), 我们把这样的子集称为$W$的子空间. 

\begin{definition}
    设 $W$ 且数域 $\mathbb{P} $ 上线性空间 $V$的非空子集, 如果对于$V$的加法, 纯量乘法$W$也构成一个线性空间, 则称$W$是$V$的子空间. 简称子空间. 
\end{definition}

下面来考察初步的性质. 

\begin{prop}
    若$W$是线性空间$V$上的非空子集, 下列的三个命题等价
    \begin{enumerate}
        \item $W$是$V$的子空间;
        \item $W$对$V$的加法, 数乘封闭; 
        \item $\forall k, l \in \mathbb{P}, \alpha, \beta \in W,  k \alpha+l \beta \in W$.
    \end{enumerate}
\end{prop}





\section{子空间}

\subsection{子空间的更多性质}

\paragraph{1. 基的扩充} 设$W$是线性空间$V$的子空间, 则$W$的基底$\alpha_1, \alpha_2,\cdots, \alpha_r$可以扩充为$V$的基底$\alpha_1, \alpha_2,\cdots, \alpha_r,\alpha_{r+1},\cdots, \alpha_n$. 
这是因为在$V$中取一组基$\vecgrp\beta n$, 根据替换定理, 有$\vecgrp\alpha r,\beta_{j_{r+1}},\cdots, \beta_{j_n}$与$\vecgrp \beta n$等价并且线性无关. 

\paragraph{2. 子空间的交与和} 如果$V_1, V_2, V_3$都是$V$的子空间, 那么对于交运算有
\begin{itemize}
    \item 交换律: $V_1\cap V_2 = V_2 \cap V_1$;
    \item 结合律: $V_1 \cap (V_2 \cap V_3)=(V_1\cap V_2)\cap V_3$;
    \item 封闭性: $V_1\cap V_2\cap \cdots \cup V_s =\bigcap_{i=1}^s V_i$也是$V$的子空间
\end{itemize}
对于和运算有
\begin{itemize}
    \item 交换律: $V_1+V_2=V_2+V_1$;
    \item 结合律: $V_1 + (V_2 + V_3)=(V_1 + V_2)+ V_3$;
    \item 封闭性: $V_1+V_2+ \cdots + V_s =\sum_{i=1}^s V_i$也是$V$的子空间
\end{itemize}

\paragraph{3. 子空间的等价条件} 设$V_1, V_2$都是$V$的子空间, 那么下面三个条件是等价的:
\begin{itemize}
    \item $V_1\subset V_2$;
    \item $V_1+V_2=V_2$;
    \item $V_1\cup V_2=V_1$.
\end{itemize}

\paragraph{4. 子空间之间的关系} 设$V_1, V_2$都是$V$的子空间, 那么
\begin{itemize}
    \item 如果$W\subset V_1, W\subset V_2$, 那么$W \subset V_1 \cap V_2$;
    \item 如果$W\supset V_1,W\supset V_2$, 那么$W\supset V_1+V_2$. 
\end{itemize}

\subsection{维数定理}

\begin{theorem}
    设$V_1,V_2$都是$V$的子空间, 那么有
    \[
    \dim(V_{1}\cap V_{2})+\dim(V_{1}+V_{2})=\dim V_{1}+\dim V_{2}.
    \]
\end{theorem}

\begin{proof}
    假设$\dim V_1=s,\dim V_2=t, \dim (V_1\cup V_2)=r$. 先在$V_1 \cap V_2$中取基底$\vecgrp\alpha r$. 然后把它扩充为$V_1$的基底$\vecgrp \alpha r \beta{r+1},\cdots, \beta$. 然后对$\vecgrp\alpha r$又扩充为$V_2$的基底$\vecgrp \alpha r \gamma_{r+1},\cdots, \gamma_t$. 

    因为$\alpha\in V_1+V_2$, 那么有$\beta\in V_1,\gamma\in V_2$, 使得$\alpha=\beta+\gamma$. 由于$\beta,\gamma$可以被上面的两个扩充的基底表示, 自然, $\alpha$可以被$\vecgrp \alpha r, \beta_{r+1},\cdots, \beta_{t},\gamma_{r+1},\cdots, \gamma_t$表示, 也就是
    \[
        V_1+V_2=L(\alpha_1,\cdots, \alpha_r, \beta_{r+1},\cdots, \beta_s,\gamma_{r+1},\gamma_t).
    \]

    设有$x_1, \cdots, x_r, y_{r+1}, \cdots, y_s, z_{r+1},\cdots,z_t\in P$使得
    \[
        {\sum_{i=1}^{r}x_i\alpha_i} + \sum_{j=r+1}^{s} y_j\beta_j + \sum_{k=r+1}^{t}z_k\gamma_k=0
    \]
    
    因此
    \[
       \sum_{j=r+1}^{s} y_j\beta_j =- \sum_{i=1}^{r}x_i\alpha_i-  \sum_{k=r+1}^{t}z_k\gamma_k
    \]
    由于等式左侧是$V_1$中的元素, 右侧是$V_1\cap V_2$的元素减去$V_2$中的元素, 所以我们断定,  $\sum_{j=r+1}^{s} y_j\beta_j$一定属于$V_1\cap V_2$. 又因为$V_1\cap V_2$的基为$\alpha_1,\cdots, \alpha_r$, 于是一定有
    \[
        \sum_{j=r+1}^{s}y_j\beta_j=\sum_{i=1}^{r}x_i'\alpha_i.
    \]
    由于$\alpha_1,\cdots, \alpha_r,\beta_{r+1},\cdots, \beta_s$线性无关, 直到
    \[
        y_{r+1}=\cdots=y_s=-x_1'=\cdots=-x_r'=0.
    \]
    于是
    \[
        \sum_{i=1}^{r}x_i\alpha_i+\sum_{k=r+1}^{t}z_k\gamma_k=0.
    \]
    因此$\alpha_1,\cdots,\alpha_r,\beta_{r+1}\cdots,\beta_s,\gamma_{r+1},\cdots,\gamma_t$线性无关且为$V_1+V_2$的基. 于是
    \[
        \dim (V_1+V_2)=\dim V_1+\dim V_2-\dim(V_1\cap V_2).
    \]
\end{proof}

\begin{corollary}
    如果$\dim V_1+\dim V_2>\dim V$, 那么$V_1\cup V_2 \neq \set{0}$.
\end{corollary}

\subsection{直和} 

\begin{theorem}[直和的等价定义]
    设$V_{1},V_{2}$都是$V$的子空间, 则下列条件等价:
    \begin{enumerate}
        \item $V_{1}\cap V_{2}=\{0\}$;
        \item $\dim(V_{1}+V_{2})=\dim V_{1}+\dim V_{2}$;
        \item $\forall\alpha\in V_{1}+V_{2}$, $\alpha_{1}\in V_{1},\alpha_{2}\in V_{2}$,$\alpha$的分解式唯一. 
        \item $\beta\in V_1,\gamma\in V_2,\alpha$若$\beta+\gamma=0,$那么$\beta=\gamma=0$.
    \end{enumerate}
\end{theorem}

若$V_{1},V_{2}$满足上述五条的任何一条, 则称$V_{1}+V_{2}$是$V_{1}+V_{2}$的直和, 通常记作$V_{1}\oplus V_{2}$.

\begin{proof}
    $(1)\iff (2)$根据上述定理可知. 

    $(1)\implies (3)$ 设$\alpha=\beta+\gamma=\beta_1+\gamma_1,\beta,\beta_1\in V_1,\gamma,\gamma_1\in V_2$. 两边相减就有
    \[
        \beta-\beta_1=\gamma-\gamma_1\in V_1\cap V_2=\set 0.
    \]
    由此说明$\beta=\beta_1,\gamma=\gamma_1$, 分解唯一. 

    $(3)\implies (4)$ $0=\beta+\gamma,\beta\in V_1, \gamma\in V_2$, 又因为$0=0+0,0\in V_1,0\in V_2$. 由于分解的唯一性知$\beta=\gamma=0$.

    $(4)\implies (1)$ 设$\beta=V_1\cap V_2$, 因此$-\beta\in V_1\cup V_2$, 而$0=\beta+(-\beta), \beta\in V_1, -\beta\in V_2$, 于是$\beta=-\beta=0$, 因此$V_1\cup V_2=\set 0$. 



\end{proof}

\begin{definition}[直和]
    设 $V_1, V_2, \ldots, V_s$ 都是线性空间 $V$ 的子空间. 又 $W=V_1+V_2+\cdots+V_s$. 如果 $\alpha \in W$ 的分解
$$
\alpha=\alpha_1+\alpha_2+\cdots+\alpha_s, \alpha_i \in V_i, 1 \leq i \leq s
$$

是唯一的, 则称 $W$ 是 $V_1, V_2 \ldots, V_s$ 的直和. 记为
$$
W=V_1 \oplus V_2 \oplus \cdots \oplus V_s
$$
\end{definition}

\begin{theorem}
     设 $V_1, V_2, \ldots, V_s$ 为线性空间 $V$ 的子空间. 又 $W=$ $V_1+V_2+\cdots+V_s$ 则下面四个条件等价.
     \begin{enumerate}
         \item 1) $W=V_1 \dot{+} V_2 \dot{+} \cdots \dot{+} V_s$.
         \item $\alpha_i \in V_i, 1 \leq i \leq s$, 且 $\sum_{i=1}^s \alpha_i=0$. 则 $\alpha_i=0,1 \leq i \leq s$.
         \item $V_j \cap \sum_{i \neq j} V_j=(0), 1 \leq i \leq s$.
         \item  $\operatorname{dim} W=\sum_{i=1}^s \operatorname{dim} V_i$.
     \end{enumerate}
\end{theorem}

\section{商空间}

\begin{definition}[商空间]
设 $V$ 是数域 $P$ 上的线性空间. $W$是 $V$ 的子空间.设 $\alpha, \beta \in V$, 且 $\alpha-\beta \in W$; 则称 $\alpha, \beta$ 模 $W$ 同余, 记为
$$
\alpha \equiv \beta(\bmod W) .
$$
$$
\bar{\alpha}=\{\beta\}, \beta \equiv \alpha(\bmod W)\}
$$

称为 $\alpha$ 模 $W$ 的 同余类. 类中的任一向量称为此类的代表.
\end{definition}

\begin{example}[多项式同余]
    对于多项式的同余, 
    设 $V={P}[x], g(x) \neq 0$. 令 $W=\langle g(x)\rangle=\{h(x) \in$ ${P}[x]|g(x)| h(x)\}$ 模 $V$ 的子空间. 自然 $\alpha, \beta \in V, \alpha \equiv \beta(\bmod g(x))$当且仅当 $\alpha \equiv \beta\left(\bmod W\right)$, 且 $\{\beta \mid \beta \equiv \alpha(\bmod g(x))\}=\{\beta \mid \beta \equiv$ $\left.\left.\alpha(\bmod W)\right)\right\}$
\end{example}

\begin{example}[空间坐标的同余]
    在空间中取定标架 $\{O ; \alpha, \beta, \gamma\}$. 于是 $X O Y$ 平面 $\pi$可看成 $\mathbf{R}^{3 \times 1}$ 中子空间 $W=\left\{\left(\begin{array}{l}x \\ y \\ 0\end{array}\right)\right\}$. 则
$$
\left(\begin{array}{l}
a \\
b \\
c
\end{array}\right) \equiv\left(\begin{array}{l}
a_1 \\
b_1 \\
c_1
\end{array}\right)(\bmod W)
$$

当且仅当 $c=c_1$. 因而 $\alpha=\left(\begin{array}{l}a \\ b \\ c\end{array}\right)$ 的模 $W$ 的同余类 $\bar{\alpha}$ 的图形是通过 $\left(\begin{array}{l}0 \\ 0 \\ c\end{array}\right)$ 平行 $\pi$ 的平面 $\pi_1$.
\end{example}

我们构造这样的商空间, 主要是用来进行更好地对研究对象进行分类. 例如, 我们可以把相同余数的多项式分为一类, 亦可以将相同的高度的坐标分为一类. 实际上, 同余也是一个等价关系. 

\begin{enumerate}

    \item (自反性). $\alpha \equiv \alpha(\bmod W)$
    \item (对称性). 若 $\alpha \equiv \beta(\bmod W)$, 则 $\beta \equiv \alpha(\bmod W)$.
    \item (传递性). 若 $\alpha \equiv \beta(\bmod W), \beta \equiv \gamma(\bmod W)$, 则$\alpha \equiv \gamma(\bmod W)$.
\end{enumerate}

实际上, 我们还会发现, 我们的两个同余类或者相等, 或者不相交.

\begin{itemize}
    \item [4.] $\alpha, \beta \in V . \bar{\alpha}=\bar{\beta}$ 当且仅当 $\bar{\alpha} \cap \bar{\beta} \neq \emptyset$ 当且仅当 $\alpha \equiv$ $\beta(\bmod W)$.
\end{itemize}

我们接下来说明商空间也是线性空间. 

\begin{theorem}
设 $W$ 是 $V$ 的子空间. 又 $\alpha_1, \beta_1, \alpha_2, \beta_2 \in V, k \in P$.且 $\alpha_i \equiv \beta_i(\bmod W), i=1,2$. 则
$$
\begin{aligned}
\alpha_1+\alpha_2 & \equiv \beta_1+\beta_2(\bmod W) . \\
k \alpha_1 & \equiv k \beta_1(\bmod W) .
\end{aligned}
$$
    
\end{theorem}

\begin{theorem}
 设 $V$ 是数域 $P$ 上的线性空间. W是 V的一个子空间. 以 $V / W$ 表示 $V$ 中元素模 $W$ 的同余类的集合. 在: $V / W$ 中定义加法和纯量乘法如下:
$$
\begin{aligned}
& \bar{\alpha}+\bar{\beta}=\overline{\alpha+\beta}, \forall \bar{\alpha}, \bar{\beta} \in V / W, \\
& k \cdot \bar{\alpha}=\overline{k \alpha}, \forall \bar{\alpha} \in V / W, k \in P .
\end{aligned}
$$

则 $V / W$ 构成数域 $P$ 上的线性空间,称为 $V$ 对 $W$ 的商空间.
\end{theorem}

\end{document}
